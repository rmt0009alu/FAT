% ORIGINAL:
% \documentclass[a4paper,12pt,twoside]{memoir}
\documentclass[a4paper,12pt,oneside]{memoir}

% Castellano
\usepackage[spanish,es-tabla]{babel}
\selectlanguage{spanish}
\usepackage[utf8]{inputenc}
\usepackage[T1]{fontenc}
\usepackage{lmodern} % Scalable font
\usepackage{microtype}
\usepackage{placeins}
% Para imágenes flotantes
\usepackage{float}

\RequirePackage{booktabs}
\RequirePackage[table]{xcolor}
\RequirePackage{xtab}
\RequirePackage{multirow}


% Links
\PassOptionsToPackage{hyphens}{url}\usepackage[colorlinks]{hyperref}
\hypersetup{
	allcolors = {red}
}

% Ecuaciones
\usepackage{amsmath}

% Bibliografía (mejor natbib que biblatex. Con [numbers, sort] indico que
% se numeren las referencias y que se ordenen según el orden de citación)
\usepackage[numbers]{natbib}

% Rutas de fichero / paquete
\newcommand{\ruta}[1]{{\sffamily #1}}

% Párrafos
\nonzeroparskip

% Huérfanas y viudas
\widowpenalty100000
\clubpenalty100000

% Imágenes

% Comando para insertar una imagen en un lugar concreto.
% Los parámetros son:
% 1 --> Ruta absoluta/relativa de la figura
% 2 --> Texto a pie de figura
% 3 --> Tamaño en tanto por uno relativo al ancho de página
\usepackage{graphicx}
\newcommand{\imagen}[3]{
	\begin{figure}[!h]
		\centering
		\includegraphics[width=#3\textwidth]{#1}
		\caption{#2}\label{fig:#1}
	\end{figure}
	\FloatBarrier
}

% Comando para insertar una imagen sin posición.
% Los parámetros son:
% 1 --> Ruta absoluta/relativa de la figura
% 2 --> Texto a pie de figura
% 3 --> Tamaño en tanto por uno relativo al ancho de página
\newcommand{\imagenflotante}[3]{
	\begin{figure}
		\centering
		\includegraphics[width=#3\textwidth]{#1}
		\caption{#2}\label{fig:#1}
	\end{figure}
}

% El comando \figura nos permite insertar figuras comodamente, y utilizando
% siempre el mismo formato. Los parametros son:
% 1 --> Porcentaje del ancho de página que ocupará la figura (de 0 a 1)
% 2 --> Fichero de la imagen
% 3 --> Texto a pie de imagen
% 4 --> Etiqueta (label) para referencias
% 5 --> Opciones que queramos pasarle al \includegraphics
% 6 --> Opciones de posicionamiento a pasarle a \begin{figure}
\newcommand{\figuraConPosicion}[6]{%
  \setlength{\anchoFloat}{#1\textwidth}%
  \addtolength{\anchoFloat}{-4\fboxsep}%
  \setlength{\anchoFigura}{\anchoFloat}%
  \begin{figure}[#6]
    \begin{center}%
      \Ovalbox{%
        \begin{minipage}{\anchoFloat}%
          \begin{center}%
            \includegraphics[width=\anchoFigura,#5]{#2}%
            \caption{#3}%
            \label{#4}%
          \end{center}%
        \end{minipage}
      }%
    \end{center}%
  \end{figure}%
}

%
% Comando para incluir imágenes en formato apaisado (sin marco).
\newcommand{\figuraApaisadaSinMarco}[5]{%
  \begin{figure}%
    \begin{center}%
    \includegraphics[angle=90,height=#1\textheight,#5]{#2}%
    \caption{#3}%
    \label{#4}%
    \end{center}%
  \end{figure}%
}
% Para las tablas
\newcommand{\otoprule}{\midrule [\heavyrulewidth]}
%
% Nuevo comando para tablas pequeñas (menos de una página).
\newcommand{\tablaSmall}[5]{%
 \begin{table}
  \begin{center}
   \rowcolors {2}{gray!35}{}
   \begin{tabular}{#2}
    \toprule
    #4
    \otoprule
    #5
    \bottomrule
   \end{tabular}
   \caption{#1}
   \label{tabla:#3}
  \end{center}
 \end{table}
}

%
% Nuevo comando para tablas pequeñas (menos de una página).
\newcommand{\tablaSmallSinColores}[5]{%
 \begin{table}[H]
  \begin{center}
   \begin{tabular}{#2}
    \toprule
    #4
    \otoprule
    #5
    \bottomrule
   \end{tabular}
   \caption{#1}
   \label{tabla:#3}
  \end{center}
 \end{table}
}

\newcommand{\tablaApaisadaSmall}[5]{%
\begin{landscape}
  \begin{table}
   \begin{center}
    \rowcolors {2}{gray!35}{}
    \begin{tabular}{#2}
     \toprule
     #4
     \otoprule
     #5
     \bottomrule
    \end{tabular}
    \caption{#1}
    \label{tabla:#3}
   \end{center}
  \end{table}
\end{landscape}
}

%
% Nuevo comando para tablas grandes con cabecera y filas alternas coloreadas en gris.
\newcommand{\tabla}[6]{%
  \begin{center}
    \tablefirsthead{
      \toprule
      #5
      \otoprule
    }
    \tablehead{
      \multicolumn{#3}{l}{\small\sl continúa desde la página anterior}\\
      \toprule
      #5
      \otoprule
    }
    \tabletail{
      \hline
      \multicolumn{#3}{r}{\small\sl continúa en la página siguiente}\\
    }
    \tablelasttail{
      \hline
    }
    \bottomcaption{#1}
    \rowcolors {2}{gray!35}{}
    \begin{xtabular}{#2}
      #6
      \bottomrule
    \end{xtabular}
    \label{tabla:#4}
  \end{center}
}

%
% Nuevo comando para tablas grandes con cabecera.
\newcommand{\tablaSinColores}[6]{%
  \begin{center}
    \tablefirsthead{
      \toprule
      #5
      \otoprule
    }
    \tablehead{
      \multicolumn{#3}{l}{\small\sl continúa desde la página anterior}\\
      \toprule
      #5
      \otoprule
    }
    \tabletail{
      \hline
      \multicolumn{#3}{r}{\small\sl continúa en la página siguiente}\\
    }
    \tablelasttail{
      \hline
    }
    \bottomcaption{#1}
    \begin{xtabular}{#2}
      #6
      \bottomrule
    \end{xtabular}
    \label{tabla:#4}
  \end{center}
}

%
% Nuevo comando para tablas grandes sin cabecera.
\newcommand{\tablaSinCabecera}[5]{%
  \begin{center}
    \tablefirsthead{
      \toprule
    }
    \tablehead{
      \multicolumn{#3}{l}{\small\sl continúa desde la página anterior}\\
      \hline
    }
    \tabletail{
      \hline
      \multicolumn{#3}{r}{\small\sl continúa en la página siguiente}\\
    }
    \tablelasttail{
      \hline
    }
    \bottomcaption{#1}
  \begin{xtabular}{#2}
    #5
   \bottomrule
  \end{xtabular}
  \label{tabla:#4}
  \end{center}
}



\definecolor{cgoLight}{HTML}{EEEEEE}
\definecolor{cgoExtralight}{HTML}{FFFFFF}

%
% Nuevo comando para tablas grandes sin cabecera.
\newcommand{\tablaSinCabeceraConBandas}[5]{%
  \begin{center}
    \tablefirsthead{
      \toprule
    }
    \tablehead{
      \multicolumn{#3}{l}{\small\sl continúa desde la página anterior}\\
      \hline
    }
    \tabletail{
      \hline
      \multicolumn{#3}{r}{\small\sl continúa en la página siguiente}\\
    }
    \tablelasttail{
      \hline
    }
    \bottomcaption{#1}
    \rowcolors[]{1}{cgoExtralight}{cgoLight}

  \begin{xtabular}{#2}
    #5
   \bottomrule
  \end{xtabular}
  \label{tabla:#4}
  \end{center}
}



\graphicspath{ {./img/} }

% Capítulos
\chapterstyle{bianchi}
\newcommand{\capitulo}[2]{
	\setcounter{chapter}{#1}
	\setcounter{section}{0}
	\setcounter{figure}{0}
	\setcounter{table}{0}
	\chapter*{\thechapter.\enskip #2}
	\addcontentsline{toc}{chapter}{\thechapter.\enskip #2}
	\markboth{#2}{#2}
}

% Apéndices
\renewcommand{\appendixname}{Apéndice}
\renewcommand*\cftappendixname{\appendixname}

\newcommand{\apendice}[1]{
	%\renewcommand{\thechapter}{A}
	\chapter{#1}
}

\renewcommand*\cftappendixname{\appendixname\ }

% Formato de portada
\makeatletter
\usepackage{xcolor}
\newcommand{\tutor}[1]{\def\@tutor{#1}}
\newcommand{\course}[1]{\def\@course{#1}}
\definecolor{cpardoBox}{HTML}{E6E6FF}
\def\maketitle{
  \null
  \thispagestyle{empty}
  % Cabecera ----------------
\noindent\includegraphics[width=\textwidth]{cabecera}\vspace{1cm}%
  \vfill
  % Título proyecto y escudo informática ----------------
  \colorbox{cpardoBox}{%
    \begin{minipage}{.8\textwidth}
      \vspace{.5cm}\Large
      \begin{center}
      \textbf{TFG del Grado en Ingeniería Informática}\vspace{.6cm}\\
      \textbf{\LARGE\@title{}}
      \end{center}
      \vspace{.2cm}
    \end{minipage}

  }%
  \hfill\begin{minipage}{.20\textwidth}
    \includegraphics[width=\textwidth]{escudoInfor}
  \end{minipage}
  \vfill
  % Datos de alumno, curso y tutores ------------------
  \begin{center}%
  {%
    \noindent\LARGE
    Presentado por \@author{}\\ 
    en Universidad de Burgos --- \@date{}\\
    Tutores: \@tutor{}\\
  }%
  \end{center}%
  \null
  \cleardoublepage
  }
\makeatother

\newcommand{\nombre}{Rodrigo Merino Tovar} %%% cambio de comando

% Datos de portada
\title{{\Huge FAT}\\[0.5cm]Financial Analysis Tool. Herramienta de análisis financiero.}
\author{\nombre}
\tutor{Dra. Virginia Ahedo García y\\ Dr. José Ignacio Santos Martín}
\date{\today}

\begin{document}

\maketitle


\newpage\null\thispagestyle{empty}\newpage


%%%%%%%%%%%%%%%%%%%%%%%%%%%%%%%%%%%%%%%%%%%%%%%%%%%%%%%%%%%%%%%%%%%%%%%%%%%%%%%%%%%%%%%%
\thispagestyle{empty}


\noindent\includegraphics[width=\textwidth]{cabecera}\vspace{1cm}

\noindent Dña. Virginia Ahedo García y D. José Ignacio Santos Martín, profesores del departamento de Ingeniería de Organización, área de Organización de Empresas.

\noindent Exponen:

\noindent Que el alumno D. \nombre, con DNI 71286910C, ha realizado el Trabajo final de Grado en Ingeniería Informática titulado ''FAT: Financial Analysis Tool. Herramienta de análisis financiero.''. 

\noindent Y que dicho trabajo ha sido realizado por el alumno bajo la dirección de los que suscriben, en virtud de lo cual se autoriza su presentación y defensa.

\begin{center} %\large
En Burgos, {\large \today}
\end{center}

\vfill\vfill\vfill

% Author and supervisor
\begin{minipage}{0.45\textwidth}
\begin{flushleft} %\large
Vº. Bº. del Tutor:\\[2cm]
Dña. Virginia Ahedo García
\end{flushleft}
\end{minipage}
\hfill
\begin{minipage}{0.45\textwidth}
\begin{flushleft} %\large
Vº. Bº. del co-tutor:\\[2cm]
D. José Ignacio Santos Martín
\end{flushleft}
\end{minipage}
\hfill

\vfill

% para casos con solo un tutor comentar lo anterior
% y descomentar lo siguiente
%Vº. Bº. del Tutor:\\[2cm]
%D. nombre tutor


\newpage\null\thispagestyle{empty}\newpage




\frontmatter

% Abstract en castellano
\renewcommand*\abstractname{Resumen}
\begin{abstract}
Cuando buscamos maximizar el rendimiento de nuestro patrimonio neto, nos enfrentamos a desafíos significativos, como la falta de información de calidad, la proliferación de rumores sesgados y nuestras propias limitaciones en educación financiera.

Para abordar estos retos, este trabajo propone una herramienta digital que recopila información técnica y fundamental de empresas y sus productos cotizados. Esta herramienta presenta datos en gráficos y tablas de fácil comprensión, lo que permite tomar decisiones de inversión informadas y fiables, especialmente orientadas a inversiones a medio o largo plazo.
\end{abstract}

\renewcommand*\abstractname{Descriptores}
\begin{abstract}
Servidor web, Python, SQLite, Django, finanzas, análisis técnico, análisis fundamental.
\end{abstract}

\clearpage

% Abstract en inglés
\renewcommand*\abstractname{Abstract}
\begin{abstract}
When we seek to maximize the return of our net worth, we face significant challenges, such as a lack of high-quality information, the proliferation of biased rumors, and our own limitations in financial literacy.

To address these challenges, this work proposes a digital tool that gathers technical and fundamental information about companies and their listed products. This tool presents data in easily understandable charts and tables, enabling informed and reliable investment decisions, particularly tailored to medium or long-term investments.
\end{abstract}

\renewcommand*\abstractname{Keywords}
\begin{abstract}
Web server, Python, SQLite, Django, finance, technical analysis, fundamental analysis.
\end{abstract}

\clearpage

% Indices
\tableofcontents

\clearpage

\listoffigures

\clearpage

\listoftables
\clearpage

\mainmatter
\capitulo{1}{Introducción}

Vivimos en una sociedad en la que diferentes conflictos de interés generan desinformación en los medios de comunicación, lo cual se hace notable a través de la difusión de noticias falsas o deliberadamente incompletas. En los mercados financieros este problema cobra especial relevancia, con la difícilmente demostrable manipulación de precios, la falta de transparencia en determinados productos cotizados y la continua propagación de rumores. 

A lo anterior debemos sumar que, según el Plan de Educación Financiera 2022-2025\citep{cnmv-informe} de la CNMV\citep{cnmv-portal} y del Banco de España\citep{bde}, existe un consenso generalizado sobre la necesidad de mejorar el nivel de cultura financiera, independientemente del país y las circunstancias de los ciudadanos. 

Además, no debemos de olvidar que cada vez vivimos más conectados y que, en los últimos años, hemos cambiado nuestra forma de acceder y gestionar los activos financieros. La constante digitalización ha traído consigo problemas estructurales como la falta de acceso a servicios para personas mayores; pero también ha facilitado la accesibilidad a los datos. Y el análisis de estos datos puede ofrecer una perspectiva razonablemente buena de lo que está ocurriendo en realidad. 

Por lo tanto, son necesarias herramientas digitales que permitan realizar una toma de decisiones informada, que sean transparentes y que generen confianza en el usuario final. Además, necesitamos poder planificar nuestro ahorro e inversión de forma coherente con nuestro nivel de tolerancia al riesgo, tratando de diversificar nuestro capital en base a una información confiable. 

La herramienta propuesta en este trabajo tiene como objetivo recopilar los aspectos técnicos y fundamentales más relevantes de algunas empresas y sus productos cotizados. La información final, en gráficos y tablas, están explicados para que se pueda entender, de forma sencilla, el análisis que se hace sobre un determinado conjunto de datos. Esta herramienta proporciona acceso a la información de una manera diferente a lo que suelen ofrecer sitios web y aplicaciones de bolsa y finanzas, porque aquí podremos entender qué está ocurriendo con la evolución de las cuentas de una empresa y compararlo con sus precios cotizados y rendimientos. Además, se ofrece al usuario la posibilidad de ver los resultados de aplicar algunos modelos y algoritmos de trading, pero siempre desde la perspectiva de inversiones a medio o largo plazo y utilizando datos al cierre de los mercados. 


\section{Estructura de la memoria}\label{estructura-de-la-memoria}

La memoria sigue la siguiente estructura:

\begin{itemize}
\tightlist
\item
  \textbf{Introducción:} breve descripción del problema a resolver y la
  solución propuesta. Estructura de la memoria y listado de materiales
  adjuntos.
\item
  \textbf{Objetivos del proyecto:} exposición de los objetivos que
  persigue el proyecto.
\item
  \textbf{Conceptos teóricos:} breve explicación de los conceptos
  teóricos clave para la comprensión de la solución propuesta.
\item
  \textbf{Técnicas y herramientas:} listado de técnicas metodológicas y
  herramientas utilizadas para gestión y desarrollo del proyecto.
\item
  \textbf{Aspectos relevantes del desarrollo:} exposición de aspectos
  destacables que tuvieron lugar durante la realización del proyecto.
\item
  \textbf{Trabajos relacionados:} estado del arte en las aplicaciones y sitios web de bolsa y finanzas.
\item
  \textbf{Conclusiones y líneas de trabajo futuras:} conclusiones
  obtenidas tras la realización del proyecto y posibilidades de mejora o
  expansión de la solución aportada.
\end{itemize}

Junto a la memoria se proporcionan los siguientes anexos:

\begin{itemize}
\tightlist
\item
  \textbf{Plan del proyecto software:} planificación temporal y estudio
  de viabilidad del proyecto.
\item
  \textbf{Especificación de requisitos del software:} se describe la
  fase de análisis; los objetivos generales, el catálogo de requisitos
  del sistema y la especificación de requisitos funcionales y no
  funcionales.
\item
  \textbf{Especificación de diseño:} se describe la fase de diseño; el
  ámbito del software, el diseño de datos, el diseño procedimental y el
  diseño arquitectónico.
\item
  \textbf{Manual del programador:} recoge los aspectos más relevantes
  relacionados con el código fuente (estructura, compilación,
  instalación, ejecución, pruebas, etc.).
\item
  \textbf{Manual de usuario:} guía de usuario para el correcto manejo de
  la aplicación.
\end{itemize}

\section{Materiales adjuntos}\label{materiales-adjuntos}

Los materiales que se adjuntan con la memoria son: 

\begin{itemize}
\tightlist
\item
	\textbf{Aplicación FAT}: Financial Analysis Tool.
\item	
	\emph{Dataset} de \textbf{vídeos de prueba}.
\end{itemize}

Además, los siguientes recursos están accesibles a través de internet:

\begin{itemize}
\tightlist
\item
  \textbf{Página web} del proyecto \citep{FAT:web}.
\item
  \textbf{Repositorio} del proyecto \citep{FAT:repo}.
\end{itemize}



\capitulo{2}{Objetivos del proyecto}

A continuación se detallan los objetivos que se persiguen con la realización de este
 proyecto:


\section{Objetivos generales}\label{objetivos_generales}

\begin{itemize}
\tightlist
\item
Desarrollar una aplicación \emph{web} que permita a un usuario la composición
 de una cartera de valores cotizados bien diversificada. 
\item
Ofrecer información agregada sobre la evolución de un valor y su sector
 de referencia. 
\item
Permitir la comparación relativa entre valores cotizados. 
\item
Aportar valor añadido a través del análisis de correlaciones entre valores. 
\item
Facilitar la interpretación de los datos recogidos mediante
 representaciones gráficas.
\item
Dar acceso a información extra mediante el análisis de series temporales con
 modelos y redes neuronales. 
\end{itemize}

\section{Objetivos de carácter técnico}\label{objetivos_tecnicos}

\begin{itemize}
\tightlist
\item
Desarrollar una plataforma \emph{web} con \emph{Django} que permita
 realizar el seguimiento de valores cotizados en algunos de los principales
 índices de referencia mundiales. 
\item
Crear bases de datos \emph{SQLite} cuya actualización sea automática a través
 de un administrador de procesos como \emph{cron} en un servidor \emph{web} remoto
 o de forma semiautomática en un servidor local. 
\item
Aplicar la arquitectura MVC (\emph{Model-View-Controler}), más conocida en
 \emph{Django} como MVT (\emph{Model-View-Template}).
\item
Diseñar formularios que permitan la interacción con el usuario para realizar 
 operaciones \emph{CRUD} en la base de datos principal y operaciones de
 lectura en las bases de datos de los valores cotizados. 
\item
Utilizar Git como sistema de control de versiones distribuido junto
 con la plataforma GitHub.
\item
Hacer uso de herramientas CI/CD integradas en el repositorio con
 \emph{GitHub actions}. Por ejemplo, utilizar \emph{pyllint} como
 herramienta de control de calidad de código, o \emph{coverage}
 para testear de forma continuada el desarrollo del proyecto. 
\item
Aplicar la metodología ágil Scrum junto con TDD (\emph{Test Driven
 Development}) en los apartados que sea posible a lo largo del desarrollo
 del software. 
\item
Realizar test unitarios, de integración y de interfaz.
\item
Utilizar Zube como herramienta de gestión de proyectos.
\item
Utilizar un sistema de documentación como Sphinx con el estilo de Read The Docs
 y con la posibilidad de subir la documentación de forma continua. 
\end{itemize}

\section{Objetivos personales}\label{objetivos-personales}

\begin{itemize}
\tightlist
\item
Crear una herramienta sencilla - y no habitual - para el público general en el 
 ecosistema de las \emph{webs} de los mercados de valores. 
\item  
Abarcar el máximo número posible de conocimientos adquiridos durante el grado.
\item
Explorar metodologías, herramientas y estándares utilizados en el mercado 
 laboral.
\item
Introducirme en el mundo del análisis y el \emph{forecasting} de series
 temporales de datos. 
\end{itemize}

\capitulo{3}{Conceptos teóricos}

Los conceptos teóricos más destacables de este proyecto residen en el estudio del modelo de Markowitz para la formación de una cartera bien diversificada - así como la correlación entre valores cotizados - y  en el análisis de \emph{forecasting} de series temporales con modelos ARIMA y con redes LSTM. 


\section{Diversificación de una cartera de valores cotizados}\label{diversificar_cartera}

Esta sección puede empezar con la idea básica de que el riesgo en las inversiones es perjudicial y tener una cartera diversificada reduce el riesgo, por lo tanto, diversificar una cartera es una buena idea. 

La diversificación de una cartera de valores cotizados es una estrategia fundamental para reducir el riesgo y mejorar la rentabilidad a largo plazo. Esta estrategia consiste en distribuir el capital entre diferentes activos, como acciones, bonos y activos de diferentes sectores y regiones geográficas. Al hacerlo, se reduce la dependencia del rendimiento de una sola empresa o sector, lo que protege a la cartera de las fluctuaciones del mercado y minimiza las pérdidas potenciales.

Una manera de caracterizar una cartera es a través del retorno medio de los activos que la componen y su varianza. En esta sección se verá cómo se realiza una optimización de varianza-media, o más conocida como \emph{Modern Portfolio Theory (MPT)} \citep{wiki:mpt}. Es decir, se va a demostrar cómo se busca una cartera con la mejor media y la mejor varianza posibles dados unos valores en dicha cartera y la ponderación de esos valores en la misma. 


\subsection{Disminuir la varianza para minorar el riesgo}

El retorno - o variación diaria porcentual - de un valor viene dado por:

\begin{equation}
	R = P_{t}/P_{t-1} - 1 = (P_{t} - P_{t-1})/P_{t}
\end{equation}

Donde $P_{t}$ es el último precio de cierre de mercado disponible y $P_{t-1}$ es el precio de cierre previo. 

Los retornos de un valor son aleatorios y podemos asumir, de forma general, que hay una distribución normal subyacente en ellos:

\imagen{img_01_distribución_retornos.png}{Distribuciones de retornos de diferentes valores. Fuente: elaboración propia}{1}

Haciendo esta asunción de distribución normal en los retornos (algo que no siempre se cumple) podemos pensar en una cartera con dos valores, A y B, cuyas distribuciones de retornos son iguales: 

\begin{align} \label{eq:1}
	A: R_{1} \sim  \mathcal{N}(\mu,\,\sigma^{2})\\ 
	B: R_{2} \sim  \mathcal{N}(\mu,\,\sigma^{2})
\end{align}

Entonces, el retorno esperado será el mismo, $\mu$ , tengamos el 100\% de A en cartera, el 100\% de B o con diferentes ponderaciones. Sin embargo, la varianza sí es distinta, porque si calculamos la desviación estándar de los retornos de A y B tenemos lo siguiente: 

\begin{equation}
	sd(R_{1}) = sd(R_{2}) = \sigma
\end{equation}

Es decir, si invertimos todo en A o todo en B, tendremos la misma varianza, pero si hacemos un reparto de, por ejemplo, 50\%/50\%, veremos que la varianza es menor. 

Para ello, dadas las distribuciones de \ref{eq:1}, asumiremos - por ahora - que son independientes. Además, supongamos que existe una variable Y con $1/2$ de $R_{1}$ y $1/2$ de $R_{2}$:

\begin{equation}
	Y = 1/2R_{1} + 1/2R_{2}
\end{equation} 

Entonces, lo que hay que calcular es la varianza de Y:

\begin{equation}
	var(Y) = var(1/2R_{1} + 1/2R_{2})
\end{equation}

Una de las maneras de calcularlo es teniendo en cuenta lo siguiente:

\begin{align}
	var(cX) &= c^{2}var(X)\\
	var(A + B) &= var(A) + var(B)
\end{align}

Y, por tanto:

\begin{equation}
	var(1/2R_{1} + 1/2R_{2}) = (1/2)^{2}\sigma^{2} + (1/2)^{2}\sigma^{2} = 1/2\sigma^{2}
\end{equation}

Es decir: 

\begin{equation}
	var(1/2R_{1} + 1/2R_{2}) = 1/2\sigma^{2} \rightarrow sd = 1/\sqrt{2}\sigma
\end{equation}


Esto nos lleva a pensar que podemos obtener los mismos retornos medios pero disminuyendo la varianza - y la desviación estándar -, es decir, asumiendo menos riesgos en nuestras inversiones porque tendremos menor volatilidad. 

\subsection{Retorno esperado y varianza de una cartera (\emph{portfolio})}

En este apartado se tratará de describir una cartera de valores de forma matemática, con el objetivo de buscar una manera de optimizarla. Para ello, empezaré con una serie de definiciones estadísticas.

Los valores de una cartera tienen unos pesos en la misma. A esos pesos se les puede caracterizar como un vector $w$:

\begin{equation*}
	w = vector\, de\, longitud\, D
\end{equation*}

Donde $D$ es la cantidad de valores que tenemos en cartera\footnote{En mi código, en lugar de $D$ utilizo \textit{num\_valores} para hacerlo más intuitivo y para cumplir con las reglas de estilo de \textit{Python}}. Así, la ponderación de un único valor en cartera vendrá representada por $w_{i}$, donde $i = 1,...,D$.

Los pesos tendrán algunas restricciones relevantes, como que la suma de todos ellos debe ser 1:

\begin{equation}
	\sum_{i=1}^{D}w_{i} = 1
\end{equation}

Además, podríamos tener otras restricciones como que los pesos deben ser positivos, lo que limitaría el uso de posiciones cortas \citep{wiki:posicion_corta} en cartera. En estos casos, se puede limitar indicando la condición de que $w_{i} \geq 0$. 

Por otro lado, hay que tener en cuenta algunas definiciones que se utilizan de forma habitual, como:

\begin{equation*}
	R_{i} = retorno\, del\, valor\, i
\end{equation*}

El retorno medio es el valor esperado de $R_{i}$:

\begin{equation}
	E(R_{i}) = \mu_{i}\;  (forma\, vectorial\, de\, todos\, los\, \mu_{i}\, :\, \mu)
\end{equation}

Hay que considerar que es posible que exista una correlación entre los retornos de los valores y, por tanto, necesitaremos hacer uso de la matriz de covarianza:

\begin{equation}
	E\{(R_{i}-\mu_{i}) (R_{j}-\mu_{j})\} = \sum_{ij}\;  (forma\, matricial\, :\, \sum_{DxD})
\end{equation}

Si tenemos en cuenta lo anterior, ya es posible definir dos conceptos fundamentales que son el retorno medio esperado y la varianza del retorno de una cartera (\emph{portfolio}):

\begin{align} \label{eq:2}
	\mu_{p} &= E(R_{p})\\
	\sigma_{p}^{2} &= var(R_{p})
\end{align}

Para entender cómo se calculan, voy a empezar por el caso más  sencillo, en el supuesto de tener sólo dos valores en cartera:

\begin{equation} \label{eq:3}
	R_{p} = wR_{1} + (1 - w)R_{2}
\end{equation}

$R_{p}$ es una función de variables aleatorias y, por tanto, tendrá una distribución en términos de media y varianza. La media de $R_{p}$ es su valor esperado y recordando que $E$ es un operador lineal podemos operar. Además, podemos sustituir por \ref{eq:2}:

\begin{align} 
    E(R_{p}) &= E(wR_{1} + (1-w)R_{2}) = wE(R_{1}) + (1-w)E(R_{2}) \\
    \mu_{p}  &= w\mu_{1} + (1-w)\mu_{2}
    \label{eq:4}
\end{align}

La varianza de $R_{p}$ requiere de más cálculos; no podemos hacer la suma directa de las dos varianzas porque puede existir correlación entre $R_{1}$ y $R_{2}$. Entonces, los más sencillo es sustituir en la definición de varianza por \ref{eq:3} y \ref{eq:4}:

\begin{equation} \label{eq:5}
\begin{aligned}
    \text{var}(R_{p}) &= E \{(R_{p} - \mu_{p})^{2}\} \\
    &= E\{[wR_{1} + (1-w)R_{2} - w\mu_{1} - (1-w)\mu_{2}]^{2}\} \\
    &= E\{[w(R_{1} - \mu_{1}) + (1-w)(R_{2} - \mu_{2})]^{2}\} \\
    &= E\{w^{2}(R_{1}-\mu_{1})^{2}\} + E\{(1-w)^{2}(R_{2}-\mu_{2})^{2}\} \\
    &\quad + 2E\{w(1-w)(R_{1}-\mu_{1})(R_{2}-\mu_{2})\} \\
    &= w^{2}\text{var}(R_{1}) + (1-w)^{2}\text{var}(R_{2})+2w(1-w)\text{cov}(R_{1}, R_{2})
\end{aligned}
\end{equation}

Lo visto es \ref{eq:5} se puede escribir en términos de la correlación en lugar de la covarianza, $corr_{12} = \rho_{12} = \sigma_{12}/(\sigma_{1}\sigma_{2})$ \citep{wiki:covarianza_correlacion}:

\begin{equation} \label{eq:6}
	\sigma_{p}^{2} = w_{2}\sigma_{1}^{2} + (1-w)^{2}\sigma_{2}^2 + 2w(1-w)\sigma_{12}
\end{equation}

\begin{equation} \label{eq:7}
	\sigma_{p}^{2} = w_{2}\sigma_{1}^{2} + (1-w)^{2}\sigma_{2}^2 + 2w(1-w)\rho_{12} \sigma_{1}\sigma_{2}
\end{equation}
	

Cualquiera de estas dos fórmulas puede ser utilizada para calcular la varianza de una cartera de valores. 

\subsection{Correlación entre valores}

Inicialmente asumía que los retornos de los valores eran totalmente independientes para demostrar que la diversificación disminuye la varianza y, por tanto, el riesgo. Sin embargo, aquí vemos que los retornos no tienen por qué ser independientes. 

Entonces, analizando detenidamente \ref{eq:6} y \ref{eq:7} vemos que, si $0 < w < 1$ y $R_{1}$ y $R_{2}$ están positivamente correlados, la varianza de la cartera aumenta. Y si $R_{1}$ y $R_{2}$ están negativamente correlados disminuimos la varianza del portfolio y, por tanto, tendremos menor riesgo. 

Ahora, para poder adecuar a código de \emph{Python} estas fórmulas, voy a pasarlas a la notación de producto escalar y despejar la matriz de covarianza de $R$, $w^{T}\Sigma w$:

\begin{equation} \label{eq:8}
\begin{aligned}
	var(R_{p}) &= E\{(R_{p}-\mu_{p}^{2})\\
	&= E\{(R^{T}w - \mu^{T}w)^{2}\} \\
	&= E{(R^{T}w - \mu^{T}w)^{T}(R^{T}w - \mu^{T}w)} \\
	&= E\{w^{T}(R^{T}-\mu^{T})^{T}(R^{T}-\mu^{T})w\} \\
	&= w^{T}E\{(R-\mu)(R-\mu)^{T}\}w \\
	&= w^{T}\Sigma w
\end{aligned}
\end{equation}

\subsection{Simulación de Montecarlo}

Es habitual representar las carteras de valores en términos de la relación rentabilidad/riesgo. Para mantener una idea matemática de los conceptos de este trabajo, al riesgo lo voy a denominar volatilidad:

\begin{equation}
	volatilidad\, cartera = volatilidad_{p} = \sqrt(var(R_{p}))
\end{equation}

Una vez tenemos las fórmulas definidas se puede realizar una simulación de Montecarlo \citep{simulacion_montecarlo} con múltiples posibles carteras de inversión y ver la relación rentabilidad/riesgo (retorno esperado/varianza):

\imagen{img_03_simulación_montecarlo.png}{Simulación de 10.000 posibles portfolios con los valores \emph{RED.MC}, \emph{EOAN.DE} y \emph{CSCO}, permitiendo posiciones cortas (covarianza calculada con datos de mayo-2023 a mayo-2024). Fuente: elaboración propia}{1}

De esta gráfica podemos deducir que es posible obtener una mejor rentabilidad sin aumentar el riesgo, i.e., puede mejorarse el rendimiento esperado de nuestra cartera manteniendo la misma volatilidad. Por ejemplo, el punto naranja indica una cartera eficiente, para la que dada una volatilidad, obtenemos el máximo rendimiento esperado posible. Mientras que la cartera representada con un punto rojo obtienen un rendimiento menor para la misma volatilidad. 

Además, podemos intuir que según aumentamos el riesgo que estamos dispuestos a asumir podemos esperar mayores retornos. 

\subsection{Retornos máximo y mínimo posibles}

Para calcular el retorno de una cartera podemos hacer también la representación de producto escalar:

\begin{equation}
	\mu_{p} = \mu^{T}w
\end{equation}

Si utilizo la intuición rápida de maximizar el retorno por sí sólo, estaré cometiendo un error en los cálculos, ya que como es de esperar el retorno no puede crecer indefinidamente (el máximo de la ecuación previa es $\infty$). Entonces, hay que añadir al menos dos restricciones, que son la de que los pesos de los valores en cartera deben sumar 1 y que los pesos deben ser positivos. Por tanto, una representación más adecuada sería:

\begin{equation}
\begin{aligned}
	\max_{w} \mu^{T}w \\
	sujeto\, a: 1_{D}^{T}w = 1 \\
	w_{i} \geq 0
\end{aligned}
\end{equation}

Es decir, estamos ante un problema de optimización con restricciones\footnote{Es habitual añadir otras restricciones como que, por ejemplo, ningún valor tenga un peso mayor al 50\%, i.e., $w_{i} \leq 0.5$ pero en mi caso no utilizaré esta limitación.} Y, más concretamente, se trata de un problema de programación lineal (LP)\citep{programacion_lineal}. 

De manera similar se puede calcular el retorno mínimo:

\begin{equation}
\begin{aligned}
	\min_{w} \mu^{T}w \\
	sujeto\, a: 1_{D}^{T}w = 1 \\
	w_{i} \geq 0
\end{aligned}
\end{equation}

En \emph{Python} estas funciones pueden representarse a través de la librería \emph{Scipy}, concretamente, con \texttt{scipy.optimize.linprog} \footnote{En este proyecto se puede ver cómo se aplica \texttt{scipy.optmize.linprog} en \texttt{DashBoard.views.py}, en el método \texttt{\_rendimientos\_min\_y\_max(retornos\_df)}.}

\subsection{Optimización en términos de retorno y riesgo simultáneamente}

En el apartado anterior se ha visto cómo optimizar los retornos, pero sin tener en cuenta el riesgo (o volatilidad). En este apartado se añade el concepto de riesgo para realizar una optimización simultánea. 

Ya hemos visto, de forma intuitiva, que según aumenta el retorno esperado también aumenta el riesgo. La idea de la optimización de carteras es que no tomemos más riesgos de los necesarios. 

El riesgo lo podemos medir con la desviación estándar. Como la minimización de la varianza también implica la minimización de la desviación estándar (la raíz cuadrada es una función monótona creciente), usaré la varianza calculada en \ref{eq:8} por comodidad en los cálculos.

Si suponemos que queremos un determinado retorno $r$, podríamos representar la minimización del riesgo de la siguiente manera:

\begin{equation} \label{eq:9}
\begin{aligned}
	min_{w} w^{T}\Sigma w \\
	sujeto a: \mu^{T}w = r \\
	1_{D}^{T}w = 1 \\
	w_{i} \geq 0
\end{aligned}
\end{equation}

Como vemos, estamos ante un problema de programación cuadrática (QP), porque la función objetivo es cuadrática en lugar de lineal, aunque las restricciones sí siguen siendo lineales. 

Para simular la optimización de una función cuadrática en \emph{Scipy} hay que utilizar una función genérica llamada \texttt{minimize()} \footnote{En este proyecto se puede ver cómo se aplica \texttt{minimize()} en \texttt{DashBoard.views.\_frontera\_eficiente\_por\_optimizacion()} como un problema de QP y en \texttt{DashBoard.views.\_mejores\_pesos\_por\_optmizacion()} como un problema que no es QP ni LP}. Hay otras librerías más específicas, pero podemos adecuar \emph{Scipy} para problemas QP.


\subsection{Frontera eficiente}

Una vez hemos conocidos los retornos mínimo y máximo posibles, con la función que queremos optimizar, \ref{eq:9}, podemos hacer que el retorno, $r$, sea una sucesión de puntos entre el mínimo y el máximo e ir calculando la varianza mínima con esos retornos objetivos. Esto nos dará el mejor nivel de riesgo posible para cada retorno entre el mínimo y el máximo posibles. 

Esto nos dará una representación en forma de hipérbola que se conoce como frontera eficiente \citep{wiki:frontera_eficiente} y que Harry Markowitz representó en \citep{book:Portfolio_selection} con su forma parabólica de la siguiente manera:

\imagen{img_04_frontera_eficiente}{Frontera eficiente. Fuente: \citep{book:Portfolio_selection}}{0.45}

Si aplicamos estos conceptos a la simulación de Montecarlo que se realizaba previamente, se obtiene lo siguiente:

\imagen{img_05_montecarlo_frontera_eficiente}{Simulación de Montecarlo con frontera eficiente y rentabilidad de cartera con valores \emph{RED.MC}, \emph{EOAN.DE} y \emph{CSCO}, permitiendo posiciones cortas (covarianza calculada con datos de mayo-2023 a mayo-2024) Fuente: Elaboración propia}{0.95}

La parte superior \footnote{Aunque se obtiene toda la curva, por el propio proceso de optimización, la parte inferior no se puede considerar eficiente porque sólo tenemos que proyectar hacia la parte superior de la misma para ver retornos mejores.} de la línea curva negra que rodea la nube de posibles carteras, obtenidas con la simulación de Montecarlo, es lo que se conoce como frontera eficiente. Lo interesante de esta curva es que cualquier punto que seleccionemos de ella nos indica que no hay otra posible cartera con menor riesgo para el mismo retorno - o que no podemos encontrar un retorno esperado mejor para un determinado nivel de riesgo -.


\subsection{Sharpe ratio}

Hasta ahora se ha visto que podemos encontrar diferentes carteras óptimas - distintas distribuciones de pesos de los mismos valores cotizados - a lo largo de la frontera eficiente, pero cabe preguntarse cómo podemos comparar dos carteras, i.e., cuál es mejor si las dos están en la frontera eficiente. 

En principio, podemos asumir que el perfil del inversor influirá en una mayor o menor aversión al riesgo, pero lo ideal es hacer un ratio entre el retorno esperado y la volatilidad para tener una medida objetiva. Ese ratio se conoce como \emph{Sharpe ratio} \citep{wiki:sharpe_ratio}:

\begin{equation}
	SR = \frac{E(R_{p}) - r_{f}}{\sigma_{p}}
\end{equation}

Donde $r_{f}$ representa la tasa libre de riesgo \footnote{En este proyecto se considera una tasa libre de riesgo de 0, porque sólo se utilizan acciones y, aunque tienen rentabilidades por dividendos, éstos no están garantizados. Otra perspectiva podría ser tomar las rentabilidades de los bonos del estado como referencia para la tasa libre de riesgo, pero he preferido limitar los cálculos al mercado de acciones cotizadas.}, que es un retorno garantizado que pueden tener determinados activos como depósitos, bonos, letras o similares. 

La obtención del \emph{Sharpe ratio} se puede realizar de dos formas diferentes. Por un lado, podemos optimizar la función del \emph{Sharpe ratio} \footnote{En este trabajo se puede ver cómo se optimiza en \texttt{DashBoard.views.\_mejores\_pesos\_por\_optimizacion()}} pero también podemos aprovechar las múltiples carteras de la simulación de Montecarlo y buscar la de mejor ratio entre todas ellas. En cualquier caso, si se han simulado suficientes carteras, los resultados deben de ser similares - no iguales porque en la simulación puede que no se haya creado la cartera óptima global -. 

Para facilitar la comprensión al usuario se puede realizar una gráfica con toda la información necesaria y acompañarlo de información adicional sobre la distribución de pesos de cada caso:

\imagen{img_06_sharpe_ratio}{Simulación de Montecarlo con frontera eficiente y Sharpe ratio con los valores \emph{RED.MC}, \emph{EOAN.DE} y \emph{CSCO}, permitiendo posiciones cortas (covarianza calculada con datos de mayo-2023 a mayo-2024) Fuente: Elaboración propia}{0.95}





-----------------------------------
Y también podemos estudiar las correlaciones entre los retornos de diferentes valores. 

De hecho, podemos explotar la correlación entre valores para distribuir los pesos de nuestros valores de forma óptima en una cartera. 
-----------------------------------

\subsection{Volatilidad y rendimiento de un único valor}

\subsection{Subsecciones}

Volatilidad y rendimiento de una cartera con varios valores

\subsection{Subsecciones}

Modelo de Markowitz

\subsection{Subsecciones}

Simulación de Monte Carlo y sharpe ratio


\subsection{Subsecciones}

Estudio de correlación con NetworkX



\subsubsection{Subsubsecciones}

Y subsecciones. 


\section{Referencias}

Las referencias se incluyen en el texto usando cite~\cite{wiki:latex}. Para citar webs, artículos o libros~\cite{koza92}, si se desean citar más de uno en el mismo lugar~\cite{bortolot2005, koza92}.


\section{Imágenes}

Se pueden incluir imágenes con los comandos standard de \LaTeX, pero esta plantilla dispone de comandos propios como por ejemplo el siguiente:

\imagen{escudoInfor.pdf}{Autómata para una expresión vacía}{.5}



\section{Listas de items}

Existen tres posibilidades:

\begin{itemize}
	\item primer item.
	\item segundo item.
\end{itemize}

\begin{enumerate}
	\item primer item.
	\item segundo item.
\end{enumerate}

\begin{description}
	\item[Primer item] más información sobre el primer item.
	\item[Segundo item] más información sobre el segundo item.
\end{description}
	
\begin{itemize}
\item 
\end{itemize}

\section{Tablas}

Igualmente se pueden usar los comandos específicos de \LaTeX o bien usar alguno de los comandos de la plantilla.

\tablaSmall{Herramientas y tecnologías utilizadas en cada parte del proyecto}{l c c c c}{herramientasportipodeuso}
{ \multicolumn{1}{l}{Herramientas} & App AngularJS & API REST & BD & Memoria \\}{ 
HTML5 & X & & &\\
CSS3 & X & & &\\
BOOTSTRAP & X & & &\\
JavaScript & X & & &\\
AngularJS & X & & &\\
Bower & X & & &\\
PHP & & X & &\\
Karma + Jasmine & X & & &\\
Slim framework & & X & &\\
Idiorm & & X & &\\
Composer & & X & &\\
JSON & X & X & &\\
PhpStorm & X & X & &\\
MySQL & & & X &\\
PhpMyAdmin & & & X &\\
Git + BitBucket & X & X & X & X\\
Mik\TeX{} & & & & X\\
\TeX{}Maker & & & & X\\
Astah & & & & X\\
Balsamiq Mockups & X & & &\\
VersionOne & X & X & X & X\\
} 

\capitulo{4}{Técnicas y herramientas}

\section{Técnicas metodológicas}\label{metodologias}

\subsection{Scrum}\label{scrum}

Scrum \citep{wiki:Scrum} es un marco de trabajo relativamente estructurado y con roles específicos 
dentro de la metodología Agile (roles principales: Product Owner, Scrum Master y desarrollador) . Se 
puede utilizar tanto para la gestión de proyectos como para el desarrollo de productos, especialmente 
en el despliegue de \emph{software}. 

Con Scrum los proyectos se dividen en iteraciones cortas llamadas \emph{sprints}. Al final de cada 
\emph{sprint} se debe presentar un producto mínimo viable y evaluar lo que se ha hecho bien y lo 
que se puede mejorar. 

Se ha optado por esta metodología, frente a otras como \emph{Waterfall}, porque ofrece una alta 
adaptabilidad y genera entrega temprana de valor, con productos viables y valorables por el usuario 
final desde las primeras fases. 

\subsection{\emph{Test-Driven Development} (TDD)}\label{test_driven_development_tdd}

TDD \citep{wiki:TDD} es una metodología de desarrollo de software que se enfoca en escribir una batería 
de tests automatizados antes de iniciar la implementación del código fuente del propio software. Posteriormente, 
se hace un proceso de refactorización para mejorar o solucionar los defectos encontrados. 

Mis conocimientos previos de \emph{Django} y \emph{SQLite} no me han permitido utilizar de forma integral 
esta metodología, pero sí que se ha seguido en diferentes etapas del desarrollo, mejorando notablemente la 
calidad del código final. 

\subsection{\emph{Behavior-Driven Development} (BDD)}\label{behavior_driven_development_bdd}

BDD \citep{wiki:BDD} es una metodología que se basa en el comportamiento del software y me ha resultado útil 
en aquellas fases del proyecto en las que no tenía una idea preconcebida del cómo trabajar con \emph{Django} 
pero sí que conocía el resultado final esperado. 

La ventaja de este enfoque es que las pruebas se escriben en un lenguaje natural y es sencillo extrapolarlas a 
un gestor de tareas con un sistema Kanban. 

\subsection{\emph{Kanban}}\label{kanban}

Kanban \citep{wiki:Kanban} es un método visual de gestión de proyectos a través de la utilización de un tablero, 
en el que se disponen una serie de tarjetas con las tareas pendientes, en curso o finalizadas. Esto permite crear 
un flujo de trabajo que prioriza aquellas tareas más urgentes o que aportan antes valor a un producto.  



\section{Patrones de diseño}\label{patrones_diseno}

\subsection{\emph{Model-View-Template} (MVT)}\label{model_view_template}

Es el patrón de diseño de \emph{Django}, Modelo-Vista-Plantilla \citep{online:django_MVT_1,online:django_MVT_2}, que 
es similar al Modelo-Vista-Controlador (MVC) \citep{wiki:modelo_MVC}. En \emph{Django}, el Modelo representa la 
estructura de los datos, la Vista maneja la lógica de la aplicación (el controlador en MVC) y la Plantilla se encarga 
de la presentación de los datos (la vista en MVC). 

Una de las ventajas de \emph{Django} es que este modelo está plenamente integrado y promueve un acoplamiento 
débil, lo que facilita el mantenimiento y la escalabilidad de una aplicación. 

\imagen{img_01_django_MVT.png}{Patrón MVT. Fuente: realización propia}{1}



\section{Control de versiones}\label{control_versiones}

\begin{itemize}
\tightlist
\item  
Herramientas consideradas: Git \citep{online:git}, Apache Subversion \citep{online:apache_subversion} 
y Mercurial \citep{online:mercurial}.
\item
  Herramienta elegida: Git.
\end{itemize}

Git y Mercurial son sistemas de control de versiones distribuidos (DVCS),
mientras que Subversion - o SVN - es centralizado (VCS). 

Una de las ventajas de Git es que permite a cada desarrollador tener una copia en local del repositorio 
completo y, aunque es menos eficiente para proyectos muy grandes, es más sencillo de utilizar para proyectos 
pequeños. Además, el sistema de ramificación de Git es más intuitivo y facilita la tarea de los desarrolladores. 

\section{Alojamiento del repositorio}\label{alojamiento_repositorio}

\begin{itemize}
\tightlist
\item
  Herramientas consideradas: GitHub \citep{online:github}, GitLab \citep{online:gitlab} y Gitea \citep{online:gitea}.
\item
  Herramienta elegida: GitHub. 
\end{itemize}

Me he decantado por GitHub porque ya lo conocía, porque se utiliza en 
algunas asignaturas del Grado de Ingeniería Informática y porque es muy 
popular, lo que facilita la resolución de problemas gracias a su mayor 
comunidad. 

GitHub puede ofrecer menor control sobre proyectos grandes - Gitea y GitLab 
permiten auto hospedaje con la configuración que más nos interese -, pero en 
proyectos medios o pequeños es una herramienta práctica y sencilla de 
utilizar, con diferentes integraciones y que facilita el uso de flujos 
de trabajo CI/CD. 

\section{Gestión del proyecto}\label{gestion-del-proyecto}

\begin{itemize}
\tightlist
\item
  Herramientas consideradas: Zube, ZenHub, Trello y Jira. 
\item
  Herramienta elegida: Zube.
\end{itemize}

Zube es una plataforma de gestión de proyectos que se integra muy bien con GitHub. Además,
permite la sincronización en tiempo real con el repositorio de referencia que se esté 
utilizando y ofrece una interfaz fácil de utilizar con posibilidad de seguimiento
a través de \emph{burndonws} , \emph{burnups} y \emph{throughput} del equipo de desarrollo 
o de los desarrolladores de forma individual. 

Frente a las alternativas valoradas, Zube ha sido la más intuitiva,
permitiendo hacer seguimiento y planificación del proyecto en pocos pasos. 


\section{Comunicación}\label{comunicacion}

\begin{itemize}
\tightlist
\item
  Herramientas consideradas: email, GitHub y Microsoft Teams \citep{online:ms_teams}.
\item
  Herramientas elegidas: todas las anteriores. 
\end{itemize}

La comunicación en tiempo real, con llamadas o vídeo llamadas a través de Teams, 
 aporta soluciones rápidas por el continuo flujo de preguntas-respuestas. Pero no 
 siempre se pueden utilizar estos medios y es preferible hacer uso de email 
 o de \emph{requests} de \emph{GitHub}. Además, recientemente, existe la posibilidad
 de integrar MS Teams con GitHub \citep{online:integrar_teams_github} para enviar notificaciones 
 a un grupo de trabajo. 

\capitulo{5}{Aspectos relevantes del desarrollo del proyecto}

\section{Inicio del proyecto}\label{inicio-del-proyecto}

Este proyecto surge con el afán de aprender más sobre la obtención, el almacenamiento, el tratamiento y el análisis de datos financieros. Inicialmente me planteé otras posibilidades que podían cubrir algunas de estas ideas de manejo de información, como el análisis de datos climatológicos, pero la obtención de la información no era sencilla y, casi al mismo tiempo, descubrí que era posible conseguir los datos de valores cotizados a través de diferentes \emph{APIs}\citep{wiki:api} y que podría trabajar con ellos en mi entorno de trabajo local.   

Por otro lado, había leído sobre la aplicación de modelos estadísticos para estimar posibles tendencias y quería comprobar el recorrido que tendrían esas estrategias. Tal vez, de fondo estuviera esa idea ingenua de que algún día nos haremos ricos con algún tipo de algoritmo mágico; pero lo cierto es que algunas de las técnicas empleadas en este trabajo han resultado interesantes y pueden ayudar a formar una cartera bien diversificada, así como a propiciar inversiones con menor riesgo del necesario. Incluso el enfoque de trading algorítmico planteado en este trabajo puede ser útil para realizar pronósticos a muy corto plazo - se realizan estimaciones de un día -.

Finalmente, al plantear mis ideas al Dr. José Ignacio Santos Martín y a la Dra. Virginia Ahedo García, ya me avisaron de que había alguna de mis ideas, como la aplicación de modelos ARIMA, que no iban a funcionar y así ha sido. Pero quería aprovechar la oportunidad para ofrecer herramientas que ayudaran a los usuarios a entender por qué este modelo no funciona, sobre todo basándome en el manejo de los datos, y qué podemos hacer los inversores para utilizar la información a nuestro alcance con otros modelos o métodos de inversión. 

\section{Metodologías}\label{inicio-del-proyecto}

De forma global se ha intentado seguir una metodología ágil a lo largo de todo el proyecto, concretamente, \emph{Scrum}. El inconveniente más evidente de seguir esta filosofía en un proyecto educativo es la falta de un equipo personas que cubran los diferentes roles necesarios. Sin embargo, he intentado ponerme en el papel de cada componente del equipo, haciéndome preguntas constantemente sobre el tiempo disponible, el producto final requerido en cada sprint, qué esperaría un potencial cliente y cómo se comprobaría el código en un entorno laboral. 

Algunos de los procesos más relevantes llevados a cabo han sido:

\begin{itemize}
\tightlist
\item  
Realización de iteraciones, \emph{sprints}, de forma constante y con diferentes temporalidades dependiendo del producto esperado al final de cada periodo. Aquí he intentado realizar un esfuerzo extra en cuanto a lo esperable en las reuniones diarias y, aunque han sido discusiones conmigo mismo, he de reconocer que el resultado en general es satisfactorio, porque me ha ayudado a plantearme diferentes cuestiones que han contribuido a mejorar mi manera de trabajar y que, desde mi punto de vista, han mejorado los incrementos al final de cada \emph{sprint}. 
\item
Disposición de tareas, conocidas como \emph{Issues}\citep{wiki:issue}, asignadas como si se tratara de un proyecto con un equipo multidisciplinar. Estas tareas han sido creadas con un gestor de proyectos llamado \emph{Zube}, que ha facilitado la consecución de objetivos y que ofrece una elevada integración con repositorios de \emph{GitHub}, lo que cual evita tener que desplegar \emph{issues} en diferentes entornos. 
\item
Utilización de un panel \emph{Kanban} integrado en \emph{Zube}. Este tablero de tarjetas ha facilitado el seguimiento de las tareas pendientes. Además, es una forma visual rápida de detectar todo el trabajo que queda por hacer, el tiempo disponible para ello y ayuda a centrar los esfuerzos en las \emph{issues} más urgentes. 
\item
Implementación de un flujo de trabajo ágil, dirigido por las tareas dispuestas en el panel \emph{Kanban}, con diferentes estados para las mismas: \emph{inbox, backlog, ready, in progress, in review} y \emph{done}. Los estados más utilizados han sido los \emph{inbox} para nuevas tareas que se me iban ocurriendo según avanzaba el proyecto, la de \emph{ready} con todas las tareas preparadas para ser realizadas y la de \emph{in progress} para tener claro lo que se estaba realizando en cada momento. El resto también han tenido relevancia, pero en menor grado. 
\end{itemize}

Adicionalmente, se han medido la cantidad de trabajo realizado y las tareas pendientes con gráficos \emph{burndown}\citep{wiki:burndown}. Ha sido frecuente no cumplir a la perfección con los plazos esperados durante los \emph{sprint} y esto se ha debido a la incorporación de pequeñas mejoras en los períodos ya definidos - algo que no se debería hacer -. Sin embargo, la visualización de los \emph{burndown} me ha ayudado a dedicar esfuerzos adicionales para llegar a cumplir con prácticamente todos los objetivos al final de cada iteración. 

\subsection{Ensayo y error en fases tempranas del proyecto}

Antes de la realización de este trabajo había utilizado \emph{Python} de forma extensiva, en diferentes asignaturas y en el ámbito personal, sin embargo, nunca había usado \emph{Django} y esto me obligó a realizar múltiples pruebas inicialmente. 

Buena parte del código que implementé inicialmente se basó en ensayo y error, buscando guías de ayuda y utilizando la documentación de \emph{Django}\citep{online:django_doc}. Este proceso de pruebas iniciales me ha favorecido en la fase final del trabajo, porque me ha permitido realizar los últimos \emph{sprints} de forma más eficiente. 

\subsection{Diseño dirigido por pruebas. TDD}

En la medida de lo posible se han intentado desarrollar tests previos a la implementación de código pero, en múltiples ocasiones, no ha sido posible. Esto se ha debido al desconocimiento previo del autor sobre el funcionamiento de \emph{Django} y a que muchas de las pruebas iban dirigidas hacia el entorno \emph{web} y la comprobación de los resultados esperados en los métodos. 

Sin embargo, el mero hecho de haber intentado implantar este tipo de desarrollo me ayudó en fases iniciales a detectar diferentes fallos y a fortalecer la estructura de pruebas que tenía implementada.



\section{Formación}\label{formación}

Este trabajo ha requerido de algunos conocimientos que previamente no se tenían, especialmente en lo referente a \emph{Django}, a la formación de carteras diversificadas y al \emph{trading} algorítmico. Pero ya se tenían conocimientos de modelos estadísticos, de bases de datos y de desarrollo web que han facilitado algunas de las fases del proyecto. 

Las fuentes fundamentales en las que se ha adquirido el conocimiento necesario sobre \emph{Django} han sido: 

\begin{itemize}
\tightlist
\item  
La documentación de \emph{Django}\citep{online:django_doc}. 
\item
Vídeos de \emph{YouTube} de canales especializados: 
\begin{itemize}
\item
Curso de \emph{Django} para principiantes\citep{online:django_fatz}.
\item
\emph{Django full course}, del que se ha obtenido buena información, especialmente para la implemetación de \emph{routers} para el uso de diferentes bases de datos\citep{online:django_full_course}.
\end{itemize}
\item
La documentación de \emph{Jinja}\citep{online:jinja_doc}.
\end{itemize}

En cuanto a la parte más financiera del trabajo, además de las lecturas realizadas durante años previos sobre diferentes métodos de inversión, debo destacar los siguientes - sobre todo por la nueva visión que me han aportado en cuanto a la creación de una cartera bien diversificada y a la implementación de técnicas de \emph{trading} algorítmico -:

\begin{itemize}
\tightlist
\item  
Parte de la teoría de Harry M. Markowitz, especialmente, con su libro \emph{Portfolio selection. Efficient diversification of investments}\citep{book:Portfolio_selection}.
\item 
Curso de \emph{Udemy} de \emph{Financial Engineering and Artificial Intelligence in Python}. Un curso muy completo que no he terminado todavía, pero que tiene información 
especializada y formal que ha contribuido al desarrollo teórico de este proyecto y a la comprensión de varios conceptos. Lo imparte \emph{The Lazy Programmer Team} y es altamente recomendable para cualquier interesado en la materia\citep{online:financial_engineering}. 
\item
Vídeos de \emph{YouTube} de fuentes especializadas: 
\begin{itemize}
\item
Serie de vídeos sobre cómo implementar la frontera eficiente en \emph{Python}\citep{online:efficient_frontier}\footnote{Aunque no se ha seguido esta técnica exactamente, por el propio proceso de obtención de datos, sí que se han conseguido resultados similares.}. 
\item
Clase del Dr. Peter Kempthorne, del MIT, sobre \emph{Portfolio Theory}\citep{online:portfolio_theory_mit}.
\end{itemize}
\item
Artículo \emph{Modern Portfolio Theory with Python} con algunas ideas interesantes\citep{online:mpt_python_medium}.
\end{itemize}

Han habido otra muchas fuentes y recursos consultadas pero, sin duda, las más relevantes son las expuestas en esta sección. 


\section{Obtención y procesamiento de datos}\label{procesado_de_datos}

La obtención de la información se planteó como algo fundamental en las primeras fases del proyecto y, hasta las fases finales se ha mantenido la estrategia que se decidió adoptar del uso de una API. Para conseguir información de mercados financieros hay diferentes APIs, (\emph{Alpha Vantage}, \emph{Finage} e \emph{IEX CLoud}, entre otras) pero muchas de ellas son de pago o permiten un acceso restringido a pocos datos. Sin embargo, hay una API que está bastante extendida que se llama \emph{yfinance}\citep{online:yfinance}. Esta API, probablemente, sea la mejor ahora mismo para obtener datos de valores cotizados de los principales mercados financieros de forma gratuita. Hace uso de la información de \emph{Yahoo Finance} y es muy utilizada en entornos en los que no es necesario un uso intensivo de datos en tiempo real, como es el caso de este trabajo. Su popularidad en \emph{GitHub} es merecida\citep{online:yfinance_github}.

Teniendo claro cómo obtener los datos y que se iban a almacenar en una base de datos SQLite3, decidí crear unos modelos en \emph{Django} (cada valor cotizado tenía su propio modelo y su correspondiente tabla, creados de forma manual) que permitieran almacenar la información necesaria. Inicialmente sólo tenía una base de datos para el índice DJ30 y en ella se mezclaba información sobre otros índices. Pero rápido se detectó que la estructura no era adecuada. Entonces, la división en diferentes bases de datos, con múltiples modelos creados de forma semiautomática, han resultado ser de gran utilidad y han permitido extender a posteriori la información disponible sin demasiadas complicaciones. 

\subsection{Paso 1. Creación de modelos}

Tras tener claro de dónde obtener los datos empecé a trabajar con algunos modelos muy básicos, que se creaban a mano y se migraban a una única base de datos. Esta forma de trabajar no era práctica e invertí una buena cantidad de tiempo en descubrir cómo generar modelos de forma automática a partir de una lista de \emph{tickers}. La idea fue generar listas con los \emph{tickers} disponibles de cada índice, lo que cual es sencillo y, a partir de esas listas ir creando modelos automáticos para diferentes bases de datos. 

La creación de modelos fue uno de los mayores problemas encontrados al inicio del trabajo, sobre todo porque quería que  fuera escalable a más índices y valores cotizados (hasta el punto de que ahora mismo es viable añadir una nueva base de datos en relativamente pocos pasos). La estrategia seguida fue la de crear un modelo base común e ir modificando el nombre de la tabla de manera dinámica. Así, cada valor cotizado tiene su propia tabla en la base de datos de su índice. La forma de conseguirlo fue la siguiente\footnote{Consultar \texttt{Analysis.models.py} para más información.}: 

\begin{verbatim}
Crear una clase base con una serie de atributos (id, precios, moneda, etc.)

Crear un diccionario de clases vacío

Recorrer una lista con todos los tickers de todos los índices:
    Por cada ticker generar una clase dinámica del tipo de la clase base
	Asignar a la clase dinámica el nombre de la tabla como el del ticker    
	Crear una nueva entrada en el diccionario con la nueva clase dinámica

Hacer el diccionario accesible a los scripts de creación de bases de datos
\end{verbatim}

\subsection{Paso 2. Enrutamiento a la base de datos adecuada}

Como se iban a utilizar diferentes bases de datos y tenía que redirigir la información de manera adecuada configuré un \emph{router} siguiendo los pasos indicados en la documentación de \emph{Django}\citep{online:django_routers}\footnote{Se puede consultar más información sobre el multi routing en el código del proyecto, en \texttt{FAT.routers.router\_bases\_datos.py}}

\subsection{Paso 3. Realizar las migraciones}

Para poder migrar la información a las bases de datos es necesario configurar los motores en \emph{Django}:

\begin{verbatim}
DATABASES = {
    'default': {
        'ENGINE': 'django.db.backends.sqlite3',
        'NAME': BASE_DIR / 'databases/db.sqlite3',
    },

    'dj30': {
        'ENGINE': 'django.db.backends.sqlite3',
        'NAME': BASE_DIR / 'databases/dj30.sqlite3',
    },

    'ibex35': {
        'ENGINE': 'django.db.backends.sqlite3',
        'NAME': BASE_DIR / 'databases/ibex35.sqlite3',
    },

    'ftse100': {
        'ENGINE': 'django.db.backends.sqlite3',
        'NAME': BASE_DIR / 'databases/ftse100.sqlite3',
    },

    'dax40': {
        'ENGINE': 'django.db.backends.sqlite3',
        'NAME': BASE_DIR / 'databases/dax40.sqlite3',
    },
    
}
\end{verbatim}

Tras haber configurado de manera adecuada las bases de datos ya se pueden hacer las migraciones con los comandos \texttt{makemigrations} y \texttt{migrate} que proporciona el \emph{framework}. Esto es especialmente interesante porque nos evita el tener que realizar las migraciones a mano o tener que generar las consultas SQL necesarias. Además, gestiona el guardado de la información. 

\subsection{Paso 4. Almacenamiento de los datos}

Cada índice tiene su propia base de datos y, adicionalmente, hay otra base de datos común a todas ellas, que tiene información sobre los usuarios, sobre las carteras de los usuarios, sobre divisas (para realizar cálculos adecuados en las carteras de los usuarios) y sobre los sectores de los valores (para agilizar las búsquedas y poder realizar comparaciones)\footnote{La estructura de las tablas de las bases de datos puede verse en los anexos de este trabajo}. 

El pseudocódigo para la obtención de la información y el volcado inicial a las bases de datos es el siguiente\footnote{Para ver cómo se rellenan las bases de datos se puede consultar el script \texttt{util.CrearBDs.py}}

\begin{verbatim}

Inicio de la función crear_bds(índice, bd, logger):
    Registrar información en el logger
    Establecer conexión a la base de datos
    Intentar:
        Iniciar transacción con la conexión (asegurando atomicidad)
        Para cada ticker en el índice:
            Obtener datos históricos del stock (stock = yfinance.Ticker(ticker))
            Ajustar formato de fechas según la base de datos
            Agregar columnas relevantes a los datos históricos
            Calcular medias móviles y añadir información de la compañía
            Verificar existencia de la tabla en la base de datos
            Si la tabla existe:
                Guardar los datos en la base de datos
                Registrar éxito en el logger
            Si no:
                Registrar error en el logger
    Capturar excepciones de SQLite y otras
    Finalmente:
        Cerrar conexión a la base de datos
        Registrar finalización del proceso en el logger
    Retornar
                
\end{verbatim}

\section{División por aplicaciones}

Cuando se trabaja con \emph{Django} es habitual implementar distintas aplicaciones que aporten una solución, a cada problema planteado, de forma individual. Esta estructura y la filosofía subyacente en ella me han permitido ir ampliando los recursos disponibles en la web. 

En la primera fase sólo tenía como objetivo almacenar la información en bases de datos, recuperar esa información y mostrarla en una web. Pero poco a poco fui descubriendo el potencial de \emph{Django} y vi que podía añadir nuevas funcionalidades que resultaran útiles. 

Todas esas funcionalidades, que en cierta medida esperaba poder desarrollar desde el principio, tenían que seguir un orden lógico, así que decidí crear mi propia estructura de aplicaciones internas:

\begin{table}[H]
\centering
\begin{tabular}{p{3cm} p{10cm}}
\toprule
\textbf{Aplicación} & \textbf{Funcionalidad} \\
\midrule
	& Aplicación raíz del proyecto \\
FAT & Gestión y configuración de otras aplicaciones \\
    & Control de migraciones de bases de datos \\
    & Control de enrutamiento a bases de datos \\
\midrule
		 & Registro y login de usuarios \\
         & Mostrar datos de componentes de índices bursátiles \\
         & Gráficas interactivas de valores cotizados \\
Analysis & Comparación de un valor con su sector de referencia \\
         & Comparación entre diferentes valores \\
         & Grafo de correlaciones entre valores \\
         & Noticias RSS \\
\midrule
		  & Control de valores en cartera \\
          & Información sobre evolución de una cartera \\
DashBoard & Gráfica de Markowitz y ratio de Sharpe \\
          & Control de valores en seguimiento \\
          & Diagramas sobre distribución en divisas y sectores \\
\midrule
    & Forecasting de series temporales con ARIMA \\
Lab & Trading algorítmico con cruce de medias \\
    & Predicción con estrategias basadas en ML \\
\midrule
News & Noticias de portada \\
     & Control de gráficos de portada \\
\bottomrule
\end{tabular}
\caption{Estructura de aplicaciones}
\label{apps}
\end{table}

Además, existen otras rutas relevantes dentro del proyecto, en las que se encuentran archivos estáticos, \emph{scripts} de utilidad (con fuentes RSS y métodos para manejar los \emph{tickers}), tests y un log que permite controlar, de manera interna, cómo han funcionado los tests\footnote{Ver los directorios \texttt{static}, \texttt{util}, \texttt{tests} y \texttt{log} respectivamente.}. 

Por otro lado se pueden encontrar las carpetas de documentación, \emph{docs}, del trabajo y la ruta con el índice para revisar la cobertura del código, \emph{htmlcov}, generada con la herramienta \texttt{coverage}\footnote{Ver la estructura completa de los directorios en los anexos}.

 

\section{Fuentes de noticias}\label{noticias}

Además de datos de cotizaciones, en la web se pueden consultar diversas noticias tanto en la página principal como en la página general de cada índice. De forma casi experimental, en la página principal de la aplicación se utiliza una API adicional que es \emph{NewsAPI}. Esta fuente de noticias tiene la ventaja de que provee imágenes asociadas a la información y que se puede hacer un proceso de filtrado sobre diferentes campos y en distintos idiomas. Sin embargo, tras múltiples pruebas, he detectado que su información suele centrarse en noticias de la India y EEUU, y que es muy complicado obtener noticias de España. Pero, en cualquier caso, he decidido dejar esta herramienta habilitada para que la página principal resulte más atractiva.

Por otro lado, se hace uso de fuentes RSS con noticias relacionadas a cada índice bursátil. Las fuentes son públicas y los \emph{feeds} se parsean gracias al módulo \texttt{feedparser}\citep{online:feedparser}.


\section{Securizar claves de APIs}\label{noticias}

Cuando se utiliza una API suele necesitarse una clave para conectarse a ella. El problema de estas claves es que no se deben de hacer públicas y, por tanto, hay que protegerlas. Con la API de \texttt{yFinance} no es necesaria una clave, pero con \texttt{NewsAPI} sí. Por su parte \emph{Django}, además, genera su propia clave (denominada \emph{SECRET\_KEY}). 

Sin estas claves no funciona el proyecto, así que tuve que buscar una manera de favorecer el funcionamiento pero sin comprometer la seguridad. Por supuesto, ya había algo desarrollado que me facilitaría el proceso y ese algo era \texttt{dotenv} y las variables de entorno. 

Tanto en el repositorio de \emph{GitHub} como en la web publicada se han eliminado las claves y se han sustituido por variables de entorno que permiten la ejecución del código sin problema. En el repositorio se crean variables de entorno para poder utilizar las \emph{GitHub actions}, concretamente, la que permite comprobar la cobertura del código con \texttt{coverage}\footnote{Ver el código de \texttt{/.github/worflows/coverage.yml} para ampliar información}. Y en la web se han creado variables de entorno con una consola en el servidor. 

Dentro de este proyecto se ha dejado un archivo de ejemplo, \texttt{FAT/.env.example}, para la configuración de las variables de entorno. Cada usuario que descargue el repositorio deberá utilizar sus propias claves, que son fáciles de conseguir en NewsAPI y al al crear nuestro propio proyecto de \emph{Django}\footnote{Ver la guía de usuario de los anexos para mayor información}. Una vez sustituidas las claves de \texttt{FAT/.env.example} sólo es necesario cambiar el nombre de ese mismo archivo a \texttt{FAT/.env} y todo debería de funcionar con normalidad. 



\section{Desarrollo backend y frontend de \emph{DashBoard} de usuario}\label{desarrollo_dashboard}

Una de las fases del trabajo en las que más tiempo se ha invertido ha sido en el desarrollo de una interfaz que permitiera al usuario ver toda la información relativa a su cartera y que pudiera entender, con facilidad, qué posibilidades existían de mejora, haciendo uso de información sobre una gráfica de Markowitz y unas tablas con información sobre el reparto de pesos de los distintos valores cotizados dentro de la propia cartera. 

Inicialmente, la información se mostraba con tablas muy básicas que no permitían entender los resultados obtenidos con claridad. Tras realizar algunas mejoras, como la inclusión de funciones de \emph{JavaScript}, la compresión de los resultados fue mejorando. 

En este apartado he tenido que realizar modificaciones continuamente, porque iba detectando \emph{bugs} o posibles mejoras que resultaran útiles. En una primera aproximación sólo tenía tablas de valores en cartera, posteriormente añadí información sobre valores en seguimiento, luego detecté que se debía mejorar la contabilidad utilizando una única divisa (me decanté por el euro, por motivos evidentes) y así sucesivamente. 

Entre las mejoras más interesantes realizadas a lo largo del tiempo, destacaría los siguientes:

\begin{itemize}
\item
Creación de un \emph{donut} para obtener información agregada sobre los valores en cartera (con cambio a EUR automático para que el usuario tenga una referencia clara). 
\item
Inclusión de un diagrama de barras que permite ver el reparto de pesos de los valores según el sector al que pertenecen. 
\item
Añadir información de valores en el mismo sector que aquellos que el usuario tenga en seguimiento. 
\end{itemize}

Todas estas modificaciones iban en el mismo sentido: tratar de que los usuarios tuvieran una herramienta para buscar los mejores valores con los que diversificar su cartera. 

Finalmente, en las últimas modificaciones conseguí añadir una gráfica de Markowitz junto con el ratio de Sharpe, así como una tabla que permite ver el reparto de pesos actual y lo que sería idóneo según la \emph{Modern Portfolio Theory}. En el backend\footnote{Es recomendable ver el método \texttt{DashBoard.views.mostrar\_markowitz\_frontera\_y\_mejores()} y todos sus métodos auxiliares} de esta gráfica hay una serie de métodos de minimización que son muy interesantes, pero que me costó implementar, especialmente por la inclusión de restricciones adicionales a los problemas (problemas LP y QP, como se ha explicado en la sección de teoría de esta memoria). 

Todos estos cambios implicaron la creación de modelos adicionales para guardar información sobre los valores en cartera y en seguimiento de un usuario - \texttt{StockComprado} y \texttt{StockSeguimiento}\footnote{Ver \texttt{DashBoard.models.py}} - y sobre el cambio de divisas - \texttt{CambioMoneda}\footnote{Ver \texttt{Analysis.models.py}} -. Según iba creando nuevos modelos iba realizando mejoras incrementales sobre la creación de las bases de datos. 



\section{Desarrollo backend y frontend del \emph{Lab}}\label{desarrollo_lab}

\subsection{Implementación de herramientas para modelo ARIMA}

Los primeros pasos para interactuar con modelos ARIMA fueron los más sencillos posibles. Inicialmente sólo intentaba mostrar un informe ARIMA/SARIMAX con \texttt{auto\_arima} de \texttt{pmdarima} he iba realizando comprobaciones de las métricas obtenidas. 

Posteriormente fui añadiendo formularios para que el usuario pudiera interactuar con el modelo en la búsqueda de los mejores parámetros (p, d, q) posibles. En esta fase me tuve que plantear muchas preguntas:

\begin{itemize}
\item
Un usuario experimentado sabe que este modelo no va a funcionar ¿qué puedo ofrecer que le resulte interesante?
\item
¿Una persona que no haya utilizado ARIMA con anterioridad puede manejarlo de alguna manera muy sencilla?
\item
¿Qué le gustaría ver a una persona que sólo va a analizar los datos?
\end{itemize}

Tratando de contestar a todas estas preguntas fui formando conjuntos de datos fácilmente entendibles y gráficas que resumían las estimaciones. Además, decidí añadir unas tablas que complementan la información de las gráficas, mostrando los errores cometidos, para prevenir al usuario final de utilizar este modelo a través de la comparación con una estrategia naíf - contrastar con un modelo que estima que el día siguiente el precio será el mismo que el día actual -.

Por otro lado, intenté desarrollar un método de validación \emph{walk forward anchored} que permitiera hacer comprobaciones diarias. Mi idea se basó en los códigos del PhD Jason Brownlee\citep{online:walk_forward_code}, pero realicé pequeñas modificaciones que me permitían comprobar el acierto en la tendencia, ya que el precio no me interesaba. El pseudocódigo es el siguiente:

\begin{verbatim}
función validación_walk_forward():
    Inicializar modelo, predicciones y aciertos_tendencia

    Dividir datos en conjuntos de entrenamiento y test

    Inicializar conjunto_total

    Para cada paso t en el conjunto de test:
        Ajustar modelo ARIMA con orden y conjunto_total
        Predicción del valor siguiente
        Agregar predicción a predicciones
        Agregar valor real a conjunto_total

        Si t > 0, obtener valor anterior del conjunto de test; 
        si no, del conjunto de entrenamiento
        
        Determinar si la predicción fue correcta según valores
        anterior y real
        
        Registrar la precisión en aciertos_tendencia

    Devolver modelo, aciertos_tendencia y predicciones
\end{verbatim}

Por último, integré las gráficas de las funciones ACF y PACF, que permiten a los usuarios más avanzados extraer los mejores parámetros (p, d, q) por sí mismos. 

\subsection{Implementación de algoritmos de trading}

En este apartado trabajé de forma casi exploratoria, porque no tenia conocimientos previos sobre cómo aplicar de estas técnicas, aunque conocía que se podían implementar.

\subsubsection{Estrategia de cruce de medias}

Inicialmente, generé los datos necesarios para la estrategia de cruce de medias móviles, algo que con \texttt{pandas} es casi trivial:

\begin{verbatim}
# Ejemplo para una MM50
datos['MM50'] = datos["precio_cierre"].rolling(50)
\end{verbatim} 

Posteriormente, generé el algoritmo como se explica en el apartado teórico de esta memoria. Cabe destacar que lo más complicado es entender el porqué de las señales de este algoritmo y cómo se retrasan los valores que marcan si se está invertido, o no, para obtener los resultados deseables. 

El objetivo con la estrategia de cruce de medias era claro: buscar las medias móviles simples que ofrecieran la mayor rentabilidad en un período concreto, para que el usuario dedujera si merecía la pena seguir ese mismo enfoque en sesiones futuras. Es decir, no se trata de predecir un valor futuro, sino de descubrir si una estrategia ha funcionado bien en el pasado y cómo de bueno sería aplicarlo a posteriori. 

De nuevo llegué a un punto en el que necesitaba comparar los resultados para saber cómo de bien se comportaba este método y la idea fue la de comparar contra una estrategia \emph{buy and hold} en la misma ventana temporal en la que se utiliza el cruce de medias. Tras múltiples pruebas, puedo intuir que la estrategia \emph{buy and hold} suele obtener mejores rentabilidades, pero hay cierto nivel de dependencia con las temporalidades y el valor seleccionado\footnote{Los resultados son tan variables que es recomendable que el usuario saque sus propias conclusiones con los resultados mostrados en la web.}.   

\subsubsection{Estrategia basada en \emph{machine learning}}

El enfoque para utilizar esta estrategia es distinto al utilizado para el método de cruce de medias en el sentido de que aquí sí se busca una predicción para la próxima sesión. Es decir, lo que se quería saber desde el principio era si un modelo de regresión o de clasificación podía estimar si la sesión futura sería alcista o bajista, sin importar el precio final. 

Para poder utilizar estos modelos era esencial disponer de la información bien estructurada y que tanto los valores de entrenamiento como los de test tuvieran cierto sentido. Empecé por probar los modelos sobre un sólo valor, pero los resultados no tenían demasiado sentido, así que decidí seguir un enfoque típico de minería de datos, formando un conjunto de valores cotizados como si fueran atributos y una clase a predecir que sería el índice bursátil de referencia de esos mismos valores cotizados. La selección de los valores se podía haber realizado de múltiples formas (por ejemplo, con un \texttt{Pipeline} con elección de atributos a través de un modelo \texttt{RandomForestClassifier()}) pero decidí que lo más adecuado era darle especial relevancia a los valores que tienen mayor peso en los propios índices. Esta decisión se basa en la capitalización de mercado de cada valor y en la capacidad que tienen algunos valores para \emph{mover} por sí solos la cotización de un índice, precisamente, por tener mayor peso en ellos.

En la fase preliminar de este apartado generé pruebas externas a la plataforma web. Extraje información de mis bases de datos y realicé adaptaciones para ver si era viable utilizar estos modelos en términos de tiempos de ejecución. Los resultados no pudieron ser más satisfactorios, porque se tardaba relativamente poco tiempo en generar una salida y mostrar la información deseada. 

Una vez comprobado que los modelos funcionaban de manera adecuada, empecé a integrar estas soluciones en el proyecto. De nuevo, la técnica para la generación del algoritmo se detalla en en la sección teórica de esta memoria, porque resultará más práctico para el lector. Pero se puede destacar - de manera similar al método del cruce de medias - que lo más relevante es entender que se tienen que retrasar los precios de cierre de los índices un día, para que la estimación que hagan los modelos tenga sentido. 

Tras tomar la decisión de qué valores debían representar a cada índice y con los datos ordenados, sólo quedaba aplicar los modelos y mostrar los resultados. En esta ocasión se decide no disponer los resultados con una gráfica, porque no aportan valor añadido a la interpretación - de hecho, podrían llegar a confundir al usuario final -. Así que la presentación de la salida se realiza en una tabla que compara los aciertos de tendencia tanto en entrenamiento como en test y coteja los retornos logarítmicos contra los que se podrían haber obtenido con una estrategia \emph{buy and hold}. 




\section{Desarrollo de formularios para interactuar con el usuario}\label{desarrollo_formularios}

En prácticamente todos los \emph{sprints} he tenido que trabajar con HTML, CSS, Bootstrap y JavaScript. En algunos momentos de forma intensiva para conseguir un resultado visualmente aceptable y entendible. 

En la parte más dinámica de interactuación con el usuario, los formularios, he realizado un esfuerzo importante en cuanto a manejo de datos y control de errores. Todos los formularios son formularios de \emph{Django}, es decir, están en el lado del servidor y no del cliente. Esto permite, entre otras ventajas, realizar validación automática de usuarios y de modelos asociados a ellos. Además, al utilizar formularios de \emph{Django}, se ha mejorado la gestión de posibles fallos a la hora de introducir los datos necesarios. 

La filosofía seguida para el desarrollo de los formularios ha mantenido dos ideas básicas:

\begin{itemize}
\item
El usuario tiene que saber la información que debe introducir y, a ser posible, con una interfaz amigable. 
\item
El usuario no tiene que poder introducir valores erróneos (intencionada o no intencionadamente).
\end{itemize}

En lo relativo al \emph{DashBoard} hay formularios disponibles para agregar nuevos valores en cartera y para añadir nuevos valores en seguimiento. Y en cuanto al \emph{Lab} hay disponibles formularios para interactuar con los modelos ARIMA y para los algoritmos basados en ML. En todos los casos los métodos de control de errores se han diseñado para que sólo se permitan aquellos datos estrictamente correctos y se rechacen, informando al usuario, aquellos datos que no cumplan con las restricciones necesarias\footnote{Ver los métodos \texttt{DashBoard.views.\_hay\_errores()} y \texttt{Lab.views.\_comprobar\_formularios()}}.


\section{Testing, log de tests y algunos estadísticos relevantes}

Desde el inicio del trabajo tenía claro que una de las facetas a cubrir era el testeo de las aplicaciones de forma ordenada. Cada aplicación, así como las utilidades, tienen sus propios tests por separado. Todas las pruebas se han estructurado de la misma manera en la que están distribuidos los directorios de las aplicaciones. Esto permite llevar un mejor control e ir incrementando las pruebas de manera casi natural. 

Hay 250 tests, que dan una cobertura de código casi completa:

\imagen{img_18_tests.png}{Número de tests realizados. Fuente: elaboración propia}{1}

Las comprobaciones se han realizado con \texttt{coverage} y es posible encontrar una serie de plantillas HTML con información adicional. Dentro del directorio \textbf{htmlcov} se puede encontrar un archivo \textbf{index.html} que al ejecutarlo nos abrirá un informe con datos relevante sobre la cobertura de todo el proyecto. Aquí se muestra el resultado final de dicho informe: 

\tablaSmall{Cobertura de código}
{c|c|c|c|c}
{tabla_coverage}
{
\rowcolor{gray!35}
\textbf{Module} & \textbf{statements} & \textbf{missing} & \textbf{excluded} & \textbf{coverage} \\
}
{
Total & 4094 & 7 & 0 & 100\% \\
}

Además, se puede comprobar un \emph{log}, que fui realizando para un mejor control y que guarda un comentario de todas aquellas pruebas que se van pasando. La estructura del \emph{log} se controla con un \emph{logger} y se pueden comprobar estos datos en el directorio \textbf{log}.

Otras métricas relevantes de este trabajo son\footnote{Información extraída con \texttt{pygount} sin contar los archivos estáticos de \emph{Django} ni el directorio del entorno virtual}:

\tablaSmallAjustadaConPosicion{Número de líneas de código \emph{Python}}
{c|c|c|c}
{tabla_lineas_codigo}
{
\textbf{App/Directorio} & \textbf{Archivos} & \textbf{Líneas código} & \textbf{Líneas comentarios} \\
}
{
FAT & 5 & 86 & 135\\
News & 3 & 139 & 104 \\
Analysis & 6 & 5013 & 523 \\
DashBoard & 7 & 521 & 686 \\
Lab & 6 & 906 & 743 \\
util & 6 & 428 & 812 \\
tests & 16 & 2309 & 307 \\
}

\tablaSmallAjustadaConPosicion{Número de líneas de lenguajes de marcas y \emph{JavaScript}}
{c|c|c|c}
{tabla_lineas_html_css_js_jinja}
{
\textbf{App/Directorio} & \textbf{Archivos} & \textbf{Líneas código} & \textbf{Líneas comentarios} \\
}
{
News & 2 & 258 & 23 \\
Analysis & 16 & 449 & 76 \\
DashBoard & 14 & 698 & 73 \\
Lab & 10 & 830 & 77 \\
static/css & 273 & 5 \\
}



\capitulo{6}{Trabajos relacionados}

Publicaciones sobre inversión y control de la diversificación de las carteras hay muchas, pero por el conocimiento que me han aportado tengo que destacar \emph{El inversor inteligente} \citep{book:Inversor_inteligente} y uno de los libros de Andrea Redondo, \emph{Inversión: Claves para alcanzar la libertad financiera} \citep{book:libertad_financiera}.

Por otro lado, intentar predecir la evolución de los precios de valores cotizados no parece algo nuevo y, de hecho, en los últimos años se han ido publicando múltiples \emph{papers}, artículos y libros relacionados con algunos de los conceptos tratados en este trabajo. No sólo podemos encontrar artículos sencillos con porciones básicas de código, sino que existe una cantidad importante de estudios serios especialmente relacionados con el \emph{forecasting} de series temporales y con la aplicación de técnicas de \emph{machine learning}. Algunos de los más relevantes se listan a continuación:

\begin{itemize}
\tightlist
\item \emph{Machine Learning for Asset Managers} \citep{paper:ML_Asset_Managers}
\item \emph{Financial Time Series Forecasting with Deep Learning: A Systematic Literature Review: 2005-2019} \citep{paper:Deep_Learning_Review}. Aquí se pueden encontrar multitud de referencias a trabajos especializados en este campo. 
\item \emph{Deep Learning for Finance: Deep Portfolios} \citep{paper:Deep_Portfolios}
\item \emph{A systematic review of fundamental and technical analysis of stock market predictions} \citep{paper:fund_and_tech_analysis}
\item \emph{Technical Analysis and Machine Learning: A Systematic Review} \citep{paper:Technical_Analysis_ML}
\end{itemize}

Además de la aplicación de técnicas de \emph{trading} algorítmico, también hay una parte importante de la comunidad analista-inversora que ha dedicado históricamente esfuerzos a aplicar redes neuronales al mundo financiero:

\begin{itemize}
\tightlist
\item \emph{Forecasting Stock Prices with Neural Networks} \citep{paper:Forecasting_NN}
\item \emph{A Survey on LSTM Neural Networks for Time Series Forecasting} \citep{paper:LSTM_Survey}
\end{itemize}

Y otros investigadores han continuado en esa misma línea:

\begin{itemize}
\tightlist
\item \emph{Neural Networks for Algorithmic Trading} \citep{paper:NN_Algorithmic_Trading}
\item \emph{Deep learning for financial applications: A survey} \citep{paper:Deep_Learning_Survey}
\item \emph{Deep Reinforcement Learning for Trading} \citep{paper:Deep_Reinforcement_Trading}
\item \emph{Stock Price Prediction Using Convolutional Neural Networks on a Gridded Time Representation} \citep{paper:CNN_Stock_Prediction}
\end{itemize}


También es casi obligatorio mencionar que es frecuente ver publicaciones sobre la programación de \emph{bots} para realizar \emph{trading} de forma automática. Si bien es cierto que no se suele utilizar \emph{Python} para la programación de estos \emph{bots}, sino lenguajes especializados como MQL4 o MQL5, podemos asumir que hay conceptos extrapolables que resultan de interés:

\begin{itemize}
\tightlist
\item \emph{Automated Trading with Python: Designing and Developing Automated Trading Systems in Python} \citep{paper:Automated_Trading_Python}
\item \emph{Building Machine Learning Powered Trading Bots} \citep{paper:ML_Trading_Bots}
\end{itemize}

Por último, si utilizamos las webs más populares sobre bolsa e inversión, veremos que empieza a ser habitual que integren algún tipo de sección - normalmente de pago - con predicción automática de precios o análisis de señales de compra-venta. Incluso hay entidades que están ofertando a sus clientes herramientas de formación automática de carteras, lo cual lleva a pensar que este tipo de servicios se están popularizando\footnote{No parece adecuado convertir este trabajo en un mecanismo publicitario para webs o entidades y por ello no indico referencias; confío en que cualquiera que esté interesado en este mundo conocerá las opciones \emph{pro} de múltiples webs o las herramientas de tipo \emph{roboadvisor}.}. 
\capitulo{7}{Conclusiones y Líneas de trabajo futuras}

\section{Conclusiones}

El análisis técnico de datos de valores cotizados es un asunto especialmente complejo, como ha quedado patente en este trabajo con el uso (y fracaso) de modelos ARIMA para realizar \emph{forecasting} de series temporales. Sin embargo, cada vez se van viendo mayor número de artículos - más o menos acertados - que tratan de avanzar en el campo del \emph{trading} automatizado o que buscan incorporar nuevas técnicas de análisis para favorecer a sus inversiones. Hoy día no es raro encontrar \emph{bots} que hacen operaciones de compra y venta semiautomatizadas o algoritmos que tratan de competir por obtener las mejores rentabilidades. Pero el objetivo de este trabajo no era obtener un muy buen resultado una única vez - que es lo que suele ocurrir en los casos mencionados previamente -, sino tratar de crear herramientas poco comunes que permitan hacer inversiones rentables a largo plazo y que tengan detrás una base de conocimiento justificable, como lo propuesto en el algoritmo del cruce de medias o en las técnicas de \emph{trading} basado en \emph{machine learning}. 

De entre todas las webs más famosas sobre bolsa e inversión, en ninguna de ellas he encontrado una herramienta que permita hacer un análisis de rentabilidad-riesgo como el que se propone en este trabajo. Y tampoco está disponible el uso de técnicas basadas e \emph{machine learning} para hacer predicciones de tendencias. Seguramente se deba a que los esfuerzos de las grandes plataformas están concentrados en la presentación agradable de datos en tiempo real y en mantener al inversor entretenido navegando por las noticias de esos portales; así que en este sentido mi propuesta podría llegar a tener una muy pequeña ventaja competitiva. 

En cuanto al desarrollo técnico, cabe destacar que el proceso de aprendizaje de \emph{Django} puede ser complicado, pero a cambio se obtiene un elevado control sobre la información y se dispone de una serie de herramientas ya integradas que facilitan el despliegue en pocos pasos, sobre todo, en lo que concierne a la gestión de las bases de datos.

He de reconocer que este trabajo ha supuesto todo un reto, especialmente por las técnicas utilizadas y por el elevado número de nuevas utilidades que he añadido a mi \emph{caja de herramientas} personal. Hace sólo unos meses no había trabajado con \emph{Django} ni con \LaTeX , tampoco había utilizado \emph{Github actions} ni había generado documentación automática con \emph{Read the docs} y \emph{Sphinx}. Sin duda guardaré buen recuerdo de todas las herramientas utilizadas, porque me han servido para salir de mi zona de confort y fortalecer algunas ideas. 

\section{Líneas de trabajo futuras} 

Además de los contenidos comentados a lo largo de esta memoria, queda pendiente una mención al uso de redes LSTM\citep{wiki:lstm} para \emph{forecasting} de series temporales. Uno de mis propósitos iniciales era utilizar este tipo de redes para realizar predicciones, pero tras varias implementaciones detecté que los resultados no eran del todo satisfactorios. 

En algunas fases del trabajo se desarrolló toda la estructura de formularios y dependencias necesarias para poder trabajar con redes LSTM. Esta parte del código está prácticamente completa - con tests incluidos - pero mi falta de confianza en los resultados obtenidos me llevó a tomar la decisión de no incorporarlo al proyecto final. En la página web no está habilitado este apartado, pero en el repositorio de \emph{GitHub} se ha dejado disponible para que se vea cómo se podría utilizar, como una herramienta adicional dentro del \emph{Lab}. Es más, aunque la presentación de resultados no está depurada, es posible realizar un \emph{forecasting} con una red que disponga de una capa oculta de cinco neuronas. La dinámica de interacción con el usuario es la misma que en otras aplicaciones de la web y no resultará extraña la interpretación de los datos de salida. 

Una de las ventajas de cómo se planteó inicialmente la aplicación del \emph{Lab} es que permitiría ir acoplando nuevas utilidades según evolucionaran los requerimientos de los usuarios o clientes. Es decir, se pueden añadir nuevas funcionalidades como la de las redes LSTM, pero también, por ejemplo, se podría utilizar \emph{fbprophet}\citep{online:fbprophet} u otras técnicas de \emph{forecasting} que pudieran resultar interesantes. 

En el laboratorio virtual, de forma adicional, se podría haber realizado algún tipo de técnica de validación cruzada para series temporales \citep{wiki:cross_val_series_temporales} al usar las estrategias de \emph{trading} basadas en \emph{machine learning}. El rendimiento podría disminuir y, por ello, no se utilizó este enfoque inicialmente, pero sería recomendable seguir esta idea en futuras implementaciones.

Por otro lado, hay un conjunto importante de datos que me habría gustado recabar, que son todos los relativos a análisis fundamental\citep{wiki:analisis_fundamental}. Para la inclusión de este tipo de información también se puede utilizar la API de \texttt{yfinance}, aunque requeriría de nuevas bases de datos o, en el mejor de los casos, de una severa adaptación de las existentes. La inclusión de dichos datos podría ofrecer una visión mucho más completa a un posible inversor y facilitaría la toma de decisiones.

Otra característica relevante que hay que señalar es que a lo largo de este trabajo se han utilizado bases de datos \emph{SQLite} por la fácil integración que hay con \emph{Django} pero, seguramente, en un entorno más profesional sería conveniente hacer una migración hacia un sistema gestor de bases de datos como \emph{PostgreSQL}. 

Finalmente, no puedo dejar pasar la ocasión de comentar que la página web está alojada en un servidor de \emph{pythonanywhere} y, tal vez, éste no sea el mejor \emph{hosting}  posible. Si un día se empezara a tener un elevado número de usuarios es más que probable que el rendimiento cayera considerablemente. Por tanto, en caso de realizar mejoras, sería altamente recomendable ampliar los servicios de alojamiento y habría que comprar un nombre de dominio adecuado. 


\bibliographystyle{plain}
\bibliography{bibliografia}

\newpage\null\thispagestyle{empty}

% Imagen y texto de licencia al final de la memoria
\newenvironment{bottompar}{\par\vspace*{\fill}}{\clearpage}

\begin{bottompar}
\begin{figure}[H]
	\centering
	\includegraphics[width=4cm]{licencia}
\end{figure}

\begin{center}
Este obra está bajo una licencia Creative Commons Reconocimiento-NoComercial-CompartirIgual 4.0 Internacional\\
(\href{https://creativecommons.org/licenses/by-nc-sa/4.0/?ref=chooser-v1}{CC BY-NC-SA 4.0 DEED}).
\end{center}
\end{bottompar}

\end{document}