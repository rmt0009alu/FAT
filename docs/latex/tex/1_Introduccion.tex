\capitulo{1}{Introducción}

Vivimos en una sociedad en la que diferentes conflictos de interés generan desinformación en los medios de comunicación, lo cual se hace notable a través de la difusión de noticias falsas o deliberadamente incompletas. En los mercados financieros este problema cobra especial relevancia, con la difícilmente demostrable manipulación de precios, la falta de transparencia en determinados productos cotizados y la continua propagación de rumores. 

A lo anterior debemos sumar que, según el Plan de Educación Financiera 2022-2025\citep{cnmv-informe} de la CNMV\citep{cnmv-portal} y del Banco de España\citep{bde}, existe un consenso generalizado sobre la necesidad de mejorar el nivel de cultura financiera, independientemente del país y las circunstancias de los ciudadanos. 

Además, no debemos de olvidar que cada vez vivimos más conectados y que, en los últimos años, hemos cambiado nuestra forma de acceder y gestionar los activos financieros. La constante digitalización ha traído consigo problemas estructurales como la falta de acceso a servicios para personas mayores; pero también ha facilitado la accesibilidad a los datos. Y el análisis de estos datos puede ofrecer una perspectiva razonablemente buena de lo que está ocurriendo en realidad. 

Por lo tanto, son necesarias herramientas digitales que permitan realizar una toma de decisiones informada, que sean transparentes y que generen confianza en el usuario final. Además, necesitamos poder planificar nuestro ahorro e inversión de forma coherente con nuestro nivel de tolerancia al riesgo, tratando de diversificar nuestro capital en base a una información confiable. 

La herramienta propuesta en este trabajo tiene como objetivo recopilar los aspectos técnicos y fundamentales más relevantes de algunas empresas y sus productos cotizados. La información final, en gráficos y tablas, están explicados para que se pueda entender, de forma sencilla, el análisis que se hace sobre un determinado conjunto de datos. Esta herramienta proporciona acceso a la información de una manera diferente a lo que suelen ofrecer sitios web y aplicaciones de bolsa y finanzas, porque aquí podremos entender qué está ocurriendo con la evolución de las cuentas de una empresa y compararlo con sus precios cotizados y rendimientos. Además, se ofrece al usuario la posibilidad de ver los resultados de aplicar algunos modelos y algoritmos de trading, pero siempre desde la perspectiva de inversiones a medio o largo plazo y utilizando datos al cierre de los mercados. 


\section{Estructura de la memoria}\label{estructura-de-la-memoria}

La memoria sigue la siguiente estructura:

\begin{itemize}
\tightlist
\item
  \textbf{Introducción:} breve descripción del problema a resolver y la
  solución propuesta. Estructura de la memoria y listado de materiales
  adjuntos.
\item
  \textbf{Objetivos del proyecto:} exposición de los objetivos que
  persigue el proyecto.
\item
  \textbf{Conceptos teóricos:} breve explicación de los conceptos
  teóricos clave para la comprensión de la solución propuesta.
\item
  \textbf{Técnicas y herramientas:} listado de técnicas metodológicas y
  herramientas utilizadas para gestión y desarrollo del proyecto.
\item
  \textbf{Aspectos relevantes del desarrollo:} exposición de aspectos
  destacables que tuvieron lugar durante la realización del proyecto.
\item
  \textbf{Trabajos relacionados:} estado del arte en las aplicaciones y sitios web de bolsa y finanzas.
\item
  \textbf{Conclusiones y líneas de trabajo futuras:} conclusiones
  obtenidas tras la realización del proyecto y posibilidades de mejora o
  expansión de la solución aportada.
\end{itemize}

Junto a la memoria se proporcionan los siguientes anexos:

\begin{itemize}
\tightlist
\item
  \textbf{Plan del proyecto software:} planificación temporal y estudio
  de viabilidad del proyecto.
\item
  \textbf{Especificación de requisitos del software:} se describe la
  fase de análisis; los objetivos generales, el catálogo de requisitos
  del sistema y la especificación de requisitos funcionales y no
  funcionales.
\item
  \textbf{Especificación de diseño:} se describe la fase de diseño; el
  ámbito del software, el diseño de datos, el diseño procedimental y el
  diseño arquitectónico.
\item
  \textbf{Manual del programador:} recoge los aspectos más relevantes
  relacionados con el código fuente (estructura, compilación,
  instalación, ejecución, pruebas, etc.).
\item
  \textbf{Manual de usuario:} guía de usuario para el correcto manejo de
  la aplicación.
\end{itemize}

\section{Materiales adjuntos}\label{materiales-adjuntos}

Los materiales que se adjuntan con la memoria son: 

\begin{itemize}
\tightlist
\item
	\textbf{Aplicación FAT}: Financial Analysis Tool.
\item	
	\emph{Dataset} de \textbf{vídeos de prueba}.
\end{itemize}

Además, los siguientes recursos están accesibles a través de internet:

\begin{itemize}
\tightlist
\item
  \textbf{Página web} del proyecto \citep{FAT:web}.
\item
  \textbf{Repositorio} del proyecto \citep{FAT:repo}.
\end{itemize}


