\capitulo{1}{Introducción}

Los mercados financieros juegan un papel fundamental en la economía global. 
Las empresas y Administraciones Públicas que necesitan financiarse acuden a 
estos mercados en busca de capital proveniente de ahorradores, que esperan 
obtener un rendimiento sobre el dinero aportado. 

Los ahorradores prestan su dinero, depositando su confianza en una empresa,
 a través de la compra de acciones, bonos, pagarés y obligaciones - o productos
 derivados, así como materias primas -. Y para estos inversores, la toma de 
 decisiones informada debe de ser la base principal de su estrategia de negocio. 

Siguiendo las valiosas recomendaciones de Benjamin Graham \citep{book:Inversor_inteligente}
 podremos invertir de forma sensata, realizando un análisis minucioso, basado 
 en unos principios subyacentes que no se van a modificar sustancialmente con 
 el paso del tiempo, pero que sí requieren de una constante actualización de 
 la información sobre el entorno de las empresas y los mercados en los que 
 cotizan. Por ello, en este trabajo se hace uso de la información disponible 
 para ayudar a los inversores a formar una cartera bien diversificada, teniendo 
 en cuenta diferentes divisas y mercados; y se aportan algunas herramientas de 
 análisis poco frecuentes en otras plataformas web \citep{online:MarketScreener,online:Investing,online:Bloomberg}, 
 como puede ser el análisis visual de correlaciones entre valores y la 
 comparación gráfica con sectores de referencia, entre otros. 
 
Por otro lado, según el Plan de Educación Financiera 2022-2025 \citep{report:cnmv_informe} 
 de la CNMV \citep{online:cnmv_portal} y del Banco de España \citep{online:bde}, existe un consenso 
 generalizado sobre la necesidad de mejorar el nivel de cultura financiera, 
 independientemente del país y las circunstancias de los ciudadanos, por lo que, de 
 manera experimental, se introduce al usuario en la utilización de modelos para análisis 
 de series temporales, con la intención de aportar herramientas adicionales a su \textit{backup} 
 financiero. Concretamente, se da acceso al uso de modelos ARIMA \citep{wiki:ARIMA} 
 y de estrategias basadas en \emph{machine learning}. Además, se ofrece una herramienta de control de carteras para que el usuario pueda hacer su propio análisis de rentabilidad-riesgo y se aportan soluciones que favorecen la diversificación. 


\section{Estructura de la memoria}\label{estructura-de-la-memoria}

La memoria sigue la siguiente estructura:

\begin{itemize}
\tightlist
\item
  \textbf{Introducción:} breve descripción del problema a resolver y la
  solución propuesta. Estructura de la memoria y listado de materiales
  adjuntos.
\item
  \textbf{Objetivos del proyecto:} exposición de los objetivos que
  persigue el proyecto.
\item
  \textbf{Conceptos teóricos:} breve explicación de los conceptos
  teóricos clave para la comprensión de la solución propuesta.
\item
  \textbf{Técnicas y herramientas:} listado de técnicas metodológicas y
  herramientas utilizadas para gestión y desarrollo del proyecto.
\item
  \textbf{Aspectos relevantes del desarrollo:} exposición de aspectos
  destacables que tuvieron lugar durante la realización del proyecto.
\item
  \textbf{Trabajos relacionados:} estado del arte en las aplicaciones y sitios web de bolsa y finanzas.
\item
  \textbf{Conclusiones y líneas de trabajo futuras:} conclusiones
  obtenidas tras la realización del proyecto y posibilidades de mejora o
  expansión de la solución aportada.
\end{itemize}

Junto a la memoria se proporcionan los siguientes anexos:

\begin{itemize}
\tightlist
\item
  \textbf{Plan del proyecto software:} planificación temporal y estudio
  de viabilidad del proyecto.
\item
  \textbf{Especificación de requisitos del software:} se describe la
  fase de análisis; los objetivos generales, el catálogo de requisitos
  del sistema y la especificación de requisitos funcionales y no
  funcionales.
\item
  \textbf{Especificación de diseño:} se describe la fase de diseño; el
  ámbito del software, el diseño de datos, el diseño procedimental y el
  diseño arquitectónico.
\item
  \textbf{Manual del programador:} recoge los aspectos más relevantes
  relacionados con el código fuente (estructura, compilación,
  instalación, ejecución, pruebas, etc.).
\item
  \textbf{Manual de usuario:} guía de usuario para el correcto manejo de
  la aplicación.
\end{itemize}

\section{Materiales adjuntos}\label{materiales-adjuntos}

Los materiales que se adjuntan con la memoria son: 

\begin{itemize}
\item
	\textbf{Herramienta web}:
	Este proyecto está especialmente pensado para ser utilizado en un entorno local, pero se ha dispuesto una página web para que el usuario pueda hacer pruebas antes de instalarlo en su equipo. Visitar \href{http://takeiteasy.pythonanywhere.com/}{FAT: Financial Analysis Tool}.
\item	
	\textbf{Vídeos de demostración}.
\end{itemize}

Además, los siguientes recursos están accesibles a través de internet:

\begin{itemize}
\item
  \textbf{Repositorio}: 
  Visitar \href{https://github.com/rmt0009alu/FAT}{FAT. GitHub}.
\item
  \textbf{Documentación del código}: 
  Visitar \href{https://fat.readthedocs.io/es/latest/intro.html#}{FAT. Read The Docs} 
\end{itemize}


