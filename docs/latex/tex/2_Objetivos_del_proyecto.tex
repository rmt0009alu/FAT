\capitulo{2}{Objetivos del proyecto}

A continuación se detallan los objetivos que se persiguen con la realización de este
 proyecto:


\section{Objetivos generales}\label{objetivos_generales}

\begin{itemize}
\tightlist
\item
Desarrollar una aplicación \emph{web} que permita a un usuario la composición
 de una cartera de valores cotizados bien diversificada. 
\item
Ofrecer información agregada sobre la evolución de un valor y su sector
 de referencia. 
\item
Permitir la comparación relativa entre valores cotizados. 
\item
Aportar valor añadido a través del análisis de correlaciones entre valores. 
\item
Facilitar la interpretación de los datos recogidos mediante
 representaciones gráficas.
\item
Dar acceso a información extra mediante el análisis de series temporales con
 modelos y redes neuronales. 
\end{itemize}

\section{Objetivos de carácter técnico}\label{objetivos_tecnicos}

\begin{itemize}
\tightlist
\item
Desarrollar una plataforma \emph{web} con \emph{Django} que permita
 realizar el seguimiento de valores cotizados en algunos de los principales
 índices de referencia mundiales. 
\item
Crear bases de datos \emph{SQLite} cuya actualización sea automática a través
 de un administrador de procesos como \emph{cron} en un servidor \emph{web} remoto
 o de forma semiautomática en un servidor local. 
\item
Aplicar la arquitectura MVC (\emph{Model-View-Controler}), más conocida en
 \emph{Django} como MVT (\emph{Model-View-Template}).
\item
Diseñar formularios que permitan la interacción con el usuario para realizar 
 operaciones \emph{CRUD} en la base de datos principal y operaciones de
 lectura en las bases de datos de los valores cotizados. 
\item
Utilizar Git como sistema de control de versiones distribuido junto
 con la plataforma GitHub.
\item
Hacer uso de herramientas CI/CD integradas en el repositorio con
 \emph{GitHub actions}. Por ejemplo, utilizar \emph{pyllint} como
 herramienta de control de calidad de código, o \emph{coverage}
 para testear de forma continuada el desarrollo del proyecto. 
\item
Aplicar la metodología ágil Scrum junto con TDD (\emph{Test Driven
 Development}) en los apartados que sea posible a lo largo del desarrollo
 del software. 
\item
Realizar test unitarios, de integración y de interfaz.
\item
Utilizar Zube como herramienta de gestión de proyectos.
\item
Utilizar un sistema de documentación como Sphinx con el estilo de Read The Docs
 y con la posibilidad de subir la documentación de forma continua. 
\end{itemize}

\section{Objetivos personales}\label{objetivos-personales}

\begin{itemize}
\tightlist
\item
Crear una herramienta sencilla - y no habitual - para el público general en el 
 ecosistema de las \emph{webs} de los mercados de valores. 
\item  
Abarcar el máximo número posible de conocimientos adquiridos durante el grado.
\item
Explorar metodologías, herramientas y estándares utilizados en el mercado 
 laboral.
\item
Introducirme en el mundo del análisis y el \emph{forecasting} de series
 temporales de datos. 
\end{itemize}
