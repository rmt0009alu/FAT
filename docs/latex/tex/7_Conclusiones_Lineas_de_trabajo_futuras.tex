\capitulo{7}{Conclusiones y Líneas de trabajo futuras}

\section{Conclusiones}

El análisis técnico de datos de valores cotizados es un asunto especialmente complejo, como ha quedado patente en este trabajo con el uso (y fracaso) de modelos ARIMA para realizar \emph{forecasting} de series temporales. Sin embargo, cada vez se van viendo mayor número de artículos - más o menos acertados - que tratan de avanzar en el campo del \emph{trading} automatizado o que buscan incorporar nuevas técnicas de análisis para favorecer a sus inversiones. Hoy día no es raro encontrar \emph{bots} que hacen operaciones de compra y venta semiautomatizadas o algoritmos que tratan de competir por obtener las mejores rentabilidades. Pero el objetivo de este trabajo no era obtener un muy buen resultado una única vez - que es lo que suele ocurrir en los casos mencionados previamente -, sino tratar de crear herramientas poco comunes que permitan hacer inversiones rentables a largo plazo y que tengan detrás una base de conocimiento justificable, como lo propuesto en el algoritmo del cruce de medias o en las técnicas de \emph{trading} basado en \emph{machine learning}. 

De entre todas las webs más famosas sobre bolsa e inversión, en ninguna de ellas he encontrado una herramienta que permita hacer un análisis de rentabilidad-riesgo como el que se propone en este trabajo. Y tampoco está disponible el uso de técnicas basadas e \emph{machine learning} para hacer predicciones de tendencias. Seguramente se deba a que los esfuerzos de las grandes plataformas están concentrados en la presentación agradable de datos en tiempo real y en mantener al inversor entretenido navegando por las noticias de esos portales; así que en este sentido mi propuesta podría llegar a tener una muy pequeña ventaja competitiva. 

En cuanto al desarrollo técnico, cabe destacar que el proceso de aprendizaje de \emph{Django} puede ser complicado, pero a cambio se obtiene un elevado control sobre la información y se dispone de una serie de herramientas ya integradas que facilitan el despliegue en pocos pasos, sobre todo, en lo que concierne a la gestión de las bases de datos.

He de reconocer que este trabajo ha supuesto todo un reto, especialmente por las técnicas utilizadas y por el elevado número de nuevas utilidades que he añadido a mi \emph{caja de herramientas} personal. Hace sólo unos meses no había trabajado con \emph{Django} ni con \LaTeX , tampoco había utilizado \emph{Github actions} ni había generado documentación automática con \emph{Read the docs} y \emph{Sphinx}. Sin duda guardaré buen recuerdo de todas las herramientas utilizadas, porque me han servido para salir de mi zona de confort y fortalecer algunas ideas. 

\section{Líneas de trabajo futuras} 

Además de los contenidos comentados a lo largo de esta memoria, queda pendiente una mención al uso de redes LSTM\citep{wiki:lstm} para \emph{forecasting} de series temporales. Uno de mis propósitos iniciales era utilizar este tipo de redes para realizar predicciones, pero tras varias implementaciones detecté que los resultados no eran del todo satisfactorios. 

En algunas fases del trabajo se desarrolló toda la estructura de formularios y dependencias necesarias para poder trabajar con redes LSTM. Esta parte del código está prácticamente completa - con tests incluidos - pero mi falta de confianza en los resultados obtenidos me llevó a tomar la decisión de no incorporarlo al proyecto final. En la página web no está habilitado este apartado, pero en el repositorio de \emph{GitHub} se ha dejado disponible para que se vea cómo se podría utilizar, como una herramienta adicional dentro del \emph{Lab}. Es más, aunque la presentación de resultados no está depurada, es posible realizar un \emph{forecasting} con una red que disponga de una capa oculta de cinco neuronas. La dinámica de interacción con el usuario es la misma que en otras aplicaciones de la web y no resultará extraña la interpretación de los datos de salida. 

Una de las ventajas de cómo se planteó inicialmente la aplicación del \emph{Lab} es que permitiría ir acoplando nuevas utilidades según evolucionaran los requerimientos de los usuarios o clientes. Es decir, se pueden añadir nuevas funcionalidades como la de las redes LSTM, pero también, por ejemplo, se podría utilizar \emph{fbprophet}\citep{online:fbprophet} u otras técnicas de \emph{forecasting} que pudieran resultar interesantes. 

En el laboratorio virtual, de forma adicional, se podría haber realizado algún tipo de técnica de validación cruzada para series temporales \citep{wiki:cross_val_series_temporales} al usar las estrategias de \emph{trading} basadas en \emph{machine learning}. El rendimiento podría disminuir y, por ello, no se utilizó este enfoque inicialmente, pero sería recomendable seguir esta idea en futuras implementaciones.

Por otro lado, hay un conjunto importante de datos que me habría gustado recabar, que son todos los relativos a análisis fundamental\citep{wiki:analisis_fundamental}. Para la inclusión de este tipo de información también se puede utilizar la API de \texttt{yfinance}, aunque requeriría de nuevas bases de datos o, en el mejor de los casos, de una severa adaptación de las existentes. La inclusión de dichos datos podría ofrecer una visión mucho más completa a un posible inversor y facilitaría la toma de decisiones.

Otra característica relevante que hay que señalar es que a lo largo de este trabajo se han utilizado bases de datos \emph{SQLite} por la fácil integración que hay con \emph{Django} pero, seguramente, en un entorno más profesional sería conveniente hacer una migración hacia un sistema gestor de bases de datos como \emph{PostgreSQL}. 

Finalmente, no puedo dejar pasar la ocasión de comentar que la página web está alojada en un servidor de \emph{pythonanywhere} y, tal vez, éste no sea el mejor \emph{hosting}  posible. Si un día se empezara a tener un elevado número de usuarios es más que probable que el rendimiento cayera considerablemente. Por tanto, en caso de realizar mejoras, sería altamente recomendable ampliar los servicios de alojamiento y habría que comprar un nombre de dominio adecuado. 