\apendice{Documentación técnica de programación}

\section{Introducción}

En este anexo se recoge la documentación técnica de programación, que incluye recomendaciones para el entorno de desarrollo, la estructura de directorio, los procesos de compilación, la configuración de integración e instalación de dependencias y las baterías de tests realizadas. 

\section{Estructura de directorios}

El repositorio del proyecto tiene la siguiente distribución de directorios:

\begin{itemize}
\tightlist
\item
\texttt{/}: directorio raíz. Además de todas las rutas de aplicaciones y utilidades, que se detallan a continuación, hay una serie de archivos que cumplen las siguientes funciones:
	\begin{itemize}
	\tightlist
	\item
	\texttt{manage.py}: archivo de utilidad de línea de comandos para tareas 
	administrativas de \emph{Django}.
	\item
	\texttt{.gitignore}: configuración de los archivos que no se encuentran en el 
	repositorio de \emph{GitHub}. Queda a criterio del desarrollador modificar este archivo 
	según las necesidades. 
	\item
	\texttt{.pylintrc}: archivo con configuración general de la herramienta de medición de calidad de código \texttt{pylint}.
	\texttt{README.md}: archivo con recomendaciones de instalación y licencia para mostrar en la portada del repositorio de \emph{GitHub}.
	\texttt{requirements.txt}: dependencias del proyecto. 
	\end{itemize}

\item 
\texttt{/.github}: archivos de configuración de \emph{GitHub actions}.

\item 
\texttt{/Analysis}: archivos de la \emph{app Analysis}. Entre otros, aquí se pueden ver los modelos \emph{StockBase}, \emph{Sectores} y \emph{CambioMoneda}.

\item 
\texttt{/Analysis/migrations}: archivos, con migraciones de modelos a bases de datos, generados automáticamente por \emph{Django}.

\item 
\texttt{/Analysis/templates}: plantillas HTML que permiten ver múltiple información de un valor y su sector asociado.  

\item 
\texttt{/Analysis/templates/includes}: plantillas HTML auxiliares que aportan mejor organización al proyecto. Todas las plantillas tienen nombres con referencias a las funciones que cumplen para mejorar la mantenibilidad.

\item 
\texttt{/DashBoard}: archivos de la \emph{app} que gestiona el área de usuario, donde se puede realizar, por ejemplo, el control de una cartera y ver la distribución de pesos más adecuada. 

\item 
\texttt{/DashBoard/migrations}: archivos, con migraciones de modelos a bases de datos, generados automáticamente por \emph{Django}.

\item 
\texttt{/DashBoard/templates}: plantillas HTML que permiten ver información agregada sobre la cartera del usuario. También están disponibles los formularios de interacción con el usuario para las funcionalidades de esta \emph{app}.

\item 
\texttt{/DashBoard/templates/includes}: plantillas HTML auxiliares.

\item 
\texttt{/databases}: directorio en el que se encuentran las bases de datos de la aplicación. 

\item 
\texttt{/docs}: carpeta de documentación. 

\item 
\texttt{/docs/burndowns}: donde se almacenan las imágenes que corresponden a la evolución de los diferentes \emph{sprints}.

\item 
\texttt{/docs/latex}: documentación de la memoria y los anexos del Trabajo de Fin de Grado que ha llevado al desarrollo de este proyecto. 

\item 
\texttt{/docs/sphinx/_build/html}: documentación en formato HTML de todo el código del proyecto siguiendo el estilo de \emph{ReadTheDocs}.

\item 
\texttt{/FAT}: aplicación principal para gestión del proyecto. Aquí se encuentra el archivo de \texttt{settings.py} junto con otros archivos relevantes para el correcto funcionamiento del resto de aplicaciones. También se puede ver la configuración del enrutamiento a las bases de datos. Además, aquí se encuentra el archivo \texttt{.env} donde se guardan las claves secretas de \emph{Django} y \emph{NewsAPI}.

\item 
\texttt{/htmlcov}: informe de tests realizado con \texttt{coverage}.

\item 
\texttt{/Lab}: archivos de la \emph{app} que gestiona el laboratorio virtual. Aquí se encuentran los archivos de control de modelos ARIMA, estrategias basadas en \emph{machine learning}, etc. 

\item 
\texttt{/Lab/migrations}: directorio vacío porque esta \emph{app} utiliza modelos ya creados para otras aplicaciones del mismo proyecto; se crea automáticamente por \emph{Django}. Aquí podemos ver los formularios \emph{Django} que se utilizan para interactuar con el usuario. 

\item 
\texttt{/Lab/templates}: plantillas HTML que permiten ver información y mostrar con los formularios con los campos predefinidos en \texttt{forms.py}.

\item 
\texttt{/Lab/templates/includes}: plantillas HTML auxiliares.

\item 
\texttt{/log}: configuración del \emph{logger} y archivos de \emph{log} de los tests realizados. Estos archivos pueden ser útiles para futuros desarrolladores.

\item 
\texttt{/News}: archivos de la \emph{app} que gestiona las noticias y la información de valores de la portada web. 

\item 
\texttt{/News/migrations}: directorio vacío, igual que en \texttt{/Lab/migrations}.

\item 
\texttt{/Lab/templates}: plantillas HTML que permiten ver información y mostrar con los formularios con los campos predefinidos en \texttt{forms.py}.

\item 
\texttt{/Lab/templates/includes}: plantillas HTML auxiliares.

\item 
\texttt{/htmlcov}:

\item 
\texttt{/htmlcov}:

\item 
\texttt{/htmlcov}:

\item 
\texttt{/htmlcov}:





\end{itemize}
\section{Manual del programador}

\section{Compilación, instalación y ejecución del proyecto}

\section{Pruebas del sistema}
