\capitulo{4}{Técnicas y herramientas}

\section{Metodologías}\label{metodologias}

\subsection{Scrum}\label{scrum}

Scrum es un marco de trabajo para el desarrollo de \emph{software} que se
engloba dentro de las metodologías ágiles. Aplica una estrategia de
trabajo iterativa e incremental a través de iteraciones (\emph{sprints})
y revisiones \citep{wiki:scrum}.

\subsection{\emph{Test-Driven Development} (TDD)}\label{test-driven-development-tdd}

Es una práctica de desarrollo de \emph{software} que se basa en la repetición
de un ciclo corto de desarrollo: transformar requerimientos a test,
desarrollar el código necesario para pasar los test y posteriormente
refactorizar el código. Esta práctica obliga a los desarrolladores a
analizar cuidadosamente las especificaciones antes de empezar a escribir
código, fomenta la escritura de test, la simplicidad del código y
aumenta la productividad. Como resultado se obtiene \emph{software} más seguro
y de mayor calidad \citep{wiki:tdd}