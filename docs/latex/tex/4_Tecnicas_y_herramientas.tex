\capitulo{4}{Técnicas y herramientas}

\section{Técnicas metodológicas}\label{metodologias}

\subsection{Scrum}\label{scrum}

Scrum \citep{wiki:Scrum} es un marco de trabajo relativamente estructurado y con roles específicos 
dentro de la metodología Agile (roles principales: Product Owner, Scrum Master y desarrollador) . Se 
puede utilizar tanto para la gestión de proyectos como para el desarrollo de productos, especialmente 
en el despliegue de \emph{software}. 

Con Scrum los proyectos se dividen en iteraciones cortas llamadas \emph{sprints}. Al final de cada 
\emph{sprint} se debe presentar un producto mínimo viable y evaluar lo que se ha hecho bien y lo 
que se puede mejorar. 

Se ha optado por esta metodología, frente a otras como \emph{Waterfall}, porque ofrece una alta 
adaptabilidad y genera entrega temprana de valor, con productos viables y valorables por el usuario 
final desde las primeras fases. 

\subsection{\emph{Test-Driven Development} (TDD)}\label{test_driven_development_tdd}

TDD \citep{wiki:TDD} es una metodología de desarrollo de software que se enfoca en escribir una batería 
de tests automatizados antes de iniciar la implementación del código fuente del propio software. Posteriormente, 
se hace un proceso de refactorización para mejorar o solucionar los defectos encontrados. 

Mis conocimientos previos de \emph{Django} y \emph{SQLite} no me han permitido utilizar de forma integral 
esta metodología, pero sí que se ha seguido en diferentes etapas del desarrollo, mejorando notablemente la 
calidad del código final. 

\subsection{\emph{Behavior-Driven Development} (BDD)}\label{behavior_driven_development_bdd}

BDD \citep{wiki:BDD} es una metodología que se basa en el comportamiento del software y me ha resultado útil 
en aquellas fases del proyecto en las que no tenía una idea preconcebida del cómo trabajar con \emph{Django} 
pero sí que conocía el resultado final esperado. 

La ventaja de este enfoque es que las pruebas se escriben en un lenguaje natural y es sencillo extrapolarlas a 
un gestor de tareas con un sistema Kanban. 

\subsection{\emph{Kanban}}\label{kanban}

Kanban \citep{wiki:Kanban} es un método visual de gestión de proyectos a través de la utilización de un tablero, 
en el que se disponen una serie de tarjetas con las tareas pendientes, en curso o finalizadas. Esto permite crear 
un flujo de trabajo que prioriza aquellas tareas más urgentes o que aportan antes valor a un producto.  



\section{Patrones de diseño}\label{patrones_diseno}

\subsection{\emph{Model-View-Template} (MVT)}\label{model_view_template}

Es el patrón de diseño de \emph{Django}, Modelo-Vista-Plantilla \citep{online:django_MVT_1,online:django_MVT_2}, que 
es similar al Modelo-Vista-Controlador (MVC) \citep{wiki:modelo_MVC}. En \emph{Django}, el Modelo representa la 
estructura de los datos, la Vista maneja la lógica de la aplicación (el controlador en MVC) y la Plantilla se encarga 
de la presentación de los datos (la vista en MVC). 

Una de las ventajas de \emph{Django} es que este modelo está plenamente integrado y promueve un acoplamiento 
débil, lo que facilita el mantenimiento y la escalabilidad de una aplicación. 

\imagen{img_01_django_MVT.png}{Patrón MVT. Fuente: realización propia}{1}



\section{Control de versiones}\label{control_versiones}

\begin{itemize}
\tightlist
\item  
Herramientas consideradas: Git \citep{online:git}, Apache Subversion \citep{online:apache_subversion} 
y Mercurial \citep{online:mercurial}.
\item
  Herramienta elegida: Git.
\end{itemize}

Git y Mercurial son sistemas de control de versiones distribuidos (DVCS),
mientras que Subversion - o SVN - es centralizado (VCS). 

Una de las ventajas de Git es que permite a cada desarrollador tener una copia en local del repositorio 
completo y, aunque es menos eficiente para proyectos muy grandes, es más sencillo de utilizar para proyectos 
pequeños. Además, el sistema de ramificación de Git es más intuitivo y facilita la tarea de los desarrolladores. 

\section{Alojamiento del repositorio}\label{alojamiento_repositorio}

\begin{itemize}
\tightlist
\item
  Herramientas consideradas: GitHub \citep{online:github}, GitLab \citep{online:gitlab} y Gitea \citep{online:gitea}.
\item
  Herramienta elegida: GitHub. 
\end{itemize}

Me he decantado por GitHub porque ya lo conocía, porque se utiliza en 
algunas asignaturas del Grado de Ingeniería Informática y porque es muy 
popular, lo que facilita la resolución de problemas gracias a su mayor 
comunidad. 

GitHub puede ofrecer menor control sobre proyectos grandes - Gitea y GitLab 
permiten auto hospedaje con la configuración que más nos interese -, pero en 
proyectos medios o pequeños es una herramienta práctica y sencilla de 
utilizar, con diferentes integraciones y que facilita el uso de flujos 
de trabajo CI/CD. 

\section{Gestión del proyecto}\label{gestion-del-proyecto}

\begin{itemize}
\tightlist
\item
  Herramientas consideradas: Zube, ZenHub, Trello y Jira. 
\item
  Herramienta elegida: Zube.
\end{itemize}

Zube es una plataforma de gestión de proyectos que se integra muy bien con GitHub. Además,
permite la sincronización en tiempo real con el repositorio de referencia que se esté 
utilizando y ofrece una interfaz fácil de utilizar con posibilidad de seguimiento
a través de \emph{burndonws} , \emph{burnups} y \emph{throughput} del equipo de desarrollo 
o de los desarrolladores de forma individual. 

Frente a las alternativas valoradas, Zube ha sido la más intuitiva,
permitiendo hacer seguimiento y planificación del proyecto en pocos pasos. 


\section{Comunicación}\label{comunicacion}

\begin{itemize}
\tightlist
\item
  Herramientas consideradas: email, GitHub y Microsoft Teams \citep{online:ms_teams}.
\item
  Herramientas elegidas: todas las anteriores. 
\end{itemize}

La comunicación en tiempo real, con llamadas o vídeo llamadas a través de Teams, 
 aporta soluciones rápidas por el continuo flujo de preguntas-respuestas. Pero no 
 siempre se pueden utilizar estos medios y es preferible hacer uso de email 
 o de \emph{requests} de \emph{GitHub}. Además, recientemente, existe la posibilidad
 de integrar MS Teams con GitHub \citep{online:integrar_teams_github} para enviar notificaciones 
 a un grupo de trabajo. 
