\apendice{Plan de Proyecto Software}

\section{Introducción}

En un proyecto con un equipo completo de personas trabajando, podríamos definir la planificación de un proyecto software como una etapa que implica detallar los objetivos y el alcance del proyecto, identificar los requisitos y recursos necesarios, establecer un cronograma con hitos clave, asignar tareas a los miembros del equipo y prever los posibles riesgos. 

Dentro del contexto de este trabajo académico, se va a detallar en qué ha consistido la planificación temporal y se tratará de hacer un estudio de viabilidad realista.


\section{Planificación temporal}

En la memoria se ha indicado que se ha utilizado una metodología ágil de gestión de proyectos, con clara fundamentación en Scrum\citep{wiki:scrum}. Algunos de los aspectos más relevantes que han cubierto esta filosofía han sido:

\begin{itemize}
\tightlist
\item  
Desarrollo incremental a través de iteraciones llamadas \emph{sprints}.
\item
Utilización de repositorio \emph{Git} para solicitud de mejoras y para realizar un seguimiento de la evolución, con acceso para los tutores desde las fases iniciales. 
\item
Control temporal a través de los \emph{sprints}, valorando al inicio de cada uno 
cuál sería la duración adecuada basándonos en las tareas que se iban a realizar y el producto que se esperaba al final de esa iteración. 
\item
Seguimiento de tareas en un panel \emph{Kanban}.
\end{itemize}

Además de los \emph{sprints} se diseñaron hitos relevantes denominados \emph{Milestones} en \emph{GitHub} que sirven como referencia de la producción realizada. 

\subsection{Sprints}

A continuación se muestra un resumen de los \emph{sprints} que han tenido lugar en las diferentes fases de este trabajo:


\subsubsection{Sprint 0. Presentación inicial a tutores del TFG}

\begin{itemize}
\item  
\textbf{Duración:} 19/10/2023 a 23/10/2023

\item
\textbf{Objetivo:} Presentar una idea de TFG a los posibles tutores. 

\item
\textbf{Contexto:} Inicio del último curso del Grado de Ingeniería Informática. Se trata de contactar con los posibles tutores que controlen el desarrollo de un TFG. Este TFG estará basado en la idea de desarrollar una herramienta, escrita en Python, para realizar análisis financiero. 

\item
\textbf{Tareas:}
	\begin{itemize}
	\tightlist
	\item 
	Contactar con los posibles tutores por email.
	\item
	Contactar por Teams con los tutores que muestren interés en el TFG e intercambiar primeras impresiones.
	\item
	Determinar las herramientas necesarias para desarrollar un TFG.
	\item
	Preguntar por consejos y opiniones que puedan ser relevantes. 
	\item
	Concertar cita para una futura reunión cuando se haya avanzado con los primeros pasos del TFG.
	\end{itemize}
\end{itemize}


\subsubsection{Sprint 1. Empezar TFG con selección de herramientas adecuadas}

\begin{itemize}
\item  
\textbf{Duración:} 24/10/2023 a 08/11/2023

\item
\textbf{Objetivo:} Empezar el desarrollo del TFG y su correspondiente documentación. Hacerlo utilizando las herramientas adecuadas sugeridas por los tutores en el trascurso del sprint 0. 

\item
\textbf{Contexto:} Tras haber mantenido una primera reunión con los tutores, se detectan ciertas necesidades en cuanto a la utilización de herramientas adecuadas para el desarrollo de un TFG. 

\item
\textbf{Tareas:}
	\begin{itemize}
	\tightlist
	\item 
	Lectura completa de documentación de TFGs disponible en la plataforma UBUVirtual. 
	\item 	
	Rellenar el formulario de \emph{Oferta de TFG Grado de Informática Online}. Buscar referencias bibliográficas teóricas más adecuadas. 
	\item 
	Utilización de \emph{Zube}\citep{online:zube} para gestionar un proyecto con metodología ágil. Para ello, hay que buscar la documentación adecuada que permita entender su utilización.
  	\item 
  	Crear un repositorio en \emph{GitHub} que permita llevar un seguimiento de las acciones realizadas. 
  	\item 
  	En el repositorio, añadir a los tutores e integrar con \emph{Zube}. 
	\item 
	Determinar la mejor herramienta para la documentación del TFG. Elegir entre \LaTeX y \emph{Word}. Justificar la elección y documentar. 
  	\item 
  	Buscar un gestor de referencias bibliográficas: \emph{Zotero vs Mendeley}. 
  	\item 
  	Concertar cita para una futura reunión cuando se haya avanzado con los primeros pasos del TFG.
  	\end{itemize}
\end{itemize}


\subsubsection{Sprint 2. Mostrar primeras tablas y gráficos en servidor web}

\begin{itemize}
\item  
\textbf{Duración:} 08/11/2023 a 29/11/2023

\item
\textbf{Objetivo:} Comenzar a desarrollar una aplicación web, mostrando información de precios cotizados y de gráficos. Almacenar información en una base de datos SQLite3, no en DataFrames ni en archivos .csv. 

\item
\textbf{Contexto:} Una de las primeras recomendaciones de los tutores fue no utilizar archivos .csv para almacenar la información, por lo tanto, se va a utilizar un sistema de almacenamiento en base de datos. La idea general es que dichos datos se descargan con una API, se almacenan y, posteriormente, se procesan para mostrar la información relevante al usuario en archivos HTML. 

\item
\textbf{Tareas:}
	\begin{itemize}
	\tightlist
	\item 
	Preparar entorno local para utilizar una base de datos SQLite3. Instalar una GUI para la BD (no relevante para el usuario final). 
	\item 	
	Preparar clases (Descargador, Vista, Ticker y/o similares) que permitan automatizar la tarea de almacenar información en una base de datos. 
	\item 
	Mostrar un primer producto - no necesariamente visualmente atractivo - en un servidor web con tablas extraídas de la BD. 
  	\item 
  	Documentar, en el capítulo 4 de la memoria, las herramientas utilizadas. 
  	\item 
  	Concertar cita para una futura reunión con los tutores. 
  	\end{itemize}
\end{itemize}


\subsubsection{Sprint 3. Mejorar lógica de web y proteger enlaces con registro. Mostrar gráficos y dar primeros estilos}

\begin{itemize}
\item  
\textbf{Duración:} 29/11/2023 a 13/12/2023

\item
\textbf{Objetivo:} Mejorar la lógica de acceso a la web. Proteger los enlaces no públicos mediante registro de usuarios. Mostrar gráficos y no sólo tablas en un estilo visualmente atractivo. 

\item
\textbf{Contexto:} Una vez llevado el proyecto a \emph{producción} (subido a un servidor web) hay que mejorar y asegurar los enlaces para que no sean accesibles en un mal uso de los mismos. Además, es necesario hacer un registro de usuarios que permita mostrar información relevante dependiendo del rol (invitado / registrado). 

\item
\textbf{Tareas:}
	\begin{itemize}
	\tightlist
	\item 
	Crear lógica de control de usuarios registrados.
	\item 	
	Asignar permiso de acceso a enlaces dependiendo del tipo de usuario. 
	\item 
	Crear portada y dar primeros estilos visuales. 
  	\item 
  	Preparar un índice o zona de \emph{breadcrumbs} de la página para mejorar la navegación.
  	\item 
  	Diseñar una vista que permita mostrar gráficos según el valor cotizado escogido. 
  	\end{itemize}
\end{itemize}


\subsubsection{Sprint 4. Manejar imágenes estáticas. Añadir base de datos de DJ30 y buscar método para auto-actualizar las BDs}

\begin{itemize}
\item  
\textbf{Duración:} 14/12/2023 a 28/12/2023

\item
\textbf{Objetivo:} Mejorar estética y mostrar imágenes estáticas. Crear y añadir una nueva base de datos. Hacer las bases de datos auto-actualizables.

\item
\textbf{Contexto:} El proyecto en local permite mostrar imágenes estáticas en la navbar y en portada, pero en \emph{producción} (subido a un servidor web) no. Hay que mejorar este aspecto para que resulte visualmente atractivo. Además, hay que crear y añadir una nueva base de datos para el índice DJ30 y hay que buscar un método para auto-actualizar las BD en producción: con \emph{cron}\citep{wiki:cron} o con \emph{tasks}. 

\item
\textbf{Tareas:}
	\begin{itemize}
	\tightlist
	\item 
	Mostrar imágenes estáticas en producción (en servidor web).
	\item 	
	Crear una nueva BD y hacerla accesible. 
	\item 
	Investigar e implantar un método de auto-actualización de BDs en producción. 
  	\end{itemize}
\end{itemize}


\subsubsection{Sprint 5. Añadir un carrusel de noticias y cambiar el acceso a las BDs. Incorporar tests y mejorar los logs}

\begin{itemize}
\item  
\textbf{Duración:} 04/01/2024 a 18/01/2024

\item
\textbf{Objetivo:} Añadir noticias relacionadas con los mercados financieros, para hacer un interfaz más útil y agradable para el usuario. Cambiar la lógica de acceso a las bases de datos, para evitar las sentencias en SQL y aprovechar las capacidades del patrón MVT\citep{online:django_MVT_2} de Django. Aplicar primeros tests para mejorar el comportamiento y aumentar la seguridad. 

\item
\textbf{Contexto:} Ahora mismo la portada principal de la página es una imagen estática que permite hacer \emph{login}, pero que no aporta la utilidad que se espera de este tipo de aplicaciones. Por ello, parece razonable añadir una zona de noticias relacionadas con los mercados financieros, que hagan la página más dinámica y agradable de usar. Además, aunque tengo diseñados unos pequeños tests, se tiene que mejorar toda la estructura de \emph{testing} y ampliar el \emph{logger}. 

\item
\textbf{Tareas:}
	\begin{itemize}
	\tightlist
	\item 
	Añadir una nueva aplicación al proyecto, que permita controlar lo que ocurre en la portada:
		\begin{itemize}
		\tightlist
		\item
		Mostrar noticas en portada.
		\item
		Mostrar información sobre los mejores y peores stocks en portada.
		\item
		Mejorar el logging para lo que ocurra en esta nueva aplicación.
		\end{itemize}
	\item
	Mejorar la lógica de acceso a las BDs. Aplicar MVC y evitar consultas directas de SQL en la medida de lo posible.
	\item 	
	Ampliar los tests y mejorar la estructura de testing. 
  	\end{itemize}
\end{itemize}


\subsubsection{Sprint 6. Añadir un \emph{dashboard} para almacenar y controlar una cartera por parte del usuario. Operaciones CRUD en BDs}

\begin{itemize}
\item  
\textbf{Duración:} 22/01/2024 a 05/02/2024

\item
\textbf{Objetivo:} Añadir un área personal que permita realizar el control de una cartera de acciones, con posiciones abiertas y posiciones objetivo. Poder crear informes con la composición de la cartera y mostrar porcentajes de inversión de forma agradable al usuario.   

\item
\textbf{Contexto:} Aunque un usuario puede consultar los datos de un valor concreto y de su índice de referencia, ahora mismo no puede tener una lista de seguimiento que le permita controlar sus inversiones. Se pretende integrar una nueva aplicación que permita ofrecer un \emph{dashboard} a los usuarios, para que puedan hacer un seguimiento de sus inversiones. 

\item
\textbf{Tareas:}
	\begin{itemize}
	\tightlist
	\item 
	Añadir una nueva aplicación al proyecto, que permita controlar un dashboard. 
		\begin{itemize}
		\tightlist
		\item
		Permitir al usuario almacenar información sobre sus inversiones actuales y futuras. 
		\item
		Mostrar acciones compradas, i.e., posiciones abiertas.
		\item
		Mostrar una lista de seguimiento. 
		\item
		Mejorar el \emph{logging} para lo que ocurra en esta nueva aplicación.
		\item
		Preparar los tests más relevantes para el control de esta nueva aplicación. 
		\item
		Integrar esta aplicación con las actuales. 
		\end{itemize}
	\item
	Aprovechar la lógica mejorada de acceso a las BDs para facilitar todas las operaciones CRUD. 
	\item 	
	Ampliar los tests.
  	\end{itemize}
\end{itemize}


\subsubsection{Sprint 7. Refactoring y testing. Poder eliminar stocks en seguimiento desde el DashBoard.}

\begin{itemize}
\item  
\textbf{Duración:} 07/02/2024 a 14/02/2024

\item
\textbf{Objetivo:} Ahora mismo hay bastantes casos sin cubrir con los tests, especialmente en las \emph{views} de las \emph{apps} incorporadas al proyecto. Por ello, se hace necesario, antes de continuar, realizar un testeo exhaustivo para asegurar una buena base en desarrollos posteriores. 

Además, voy a incorporar una nueva funcionalidad dentro del DashBoard, que permitirá al usuario eliminar valores en seguimiento (de manera similar a como se eliminan actualmente los valores en cartera).

\item
\textbf{Contexto:} Tras intentar seguir un enfoque TDD en la creación del DashBoard en el sprint 6, se detecta la necesidad de resolver algunos fallos y cubrir el código, al 100\%, con tests. Además, hay que mejorar el estilo y la calidad del código, así como medir dicha calidad con alguna herramienta que ofrezca información final que justifique las mejoras. 

\item
\textbf{Tareas:}
	\begin{itemize}
	\tightlist
	\item 
	Cubrir todas las normas de estilo de \emph{PEP-8} enhancement
	\item
	Solucionar problema formularios \emph{DashBoard} con precios de más de 2 decimales.
	\item
	Incluir funcionalidad de eliminación de stocks en seguimiento.
	\item
	Completar tests unitarios de \texttt{News.views}
	\item
	Completar tests unitarios \texttt{DashBoard.views}
	\item
	Completar tests unitarios de \texttt{Analysis.views}
	\item
	Medir con \emph{flake8} y \emph{pylint} la calidad del código.
  	\end{itemize}
\end{itemize}


\subsubsection{Sprint 8. Correlaciones entre valores con NetworkX}

\begin{itemize}
\item  
\textbf{Duración:} 17/02/2024 a 02/03/2024

\item
\textbf{Objetivo:} Permitir al usuario comparar la evolución de un valor con respecto al resto, en un tiempo determinado. La idea es mostrar al usuario un grafo creado con \emph{NetworkX}\citep{online:networkx} que permita entender las correlaciones (positiva y negativa) que haya entre diferentes valores. De esta manera se podrá hacer un análisis de diversificación de cartera, ya que tener valores muy correlacionados puede ser indicativo de no tener la cartera bien balanceada. 

\item
\textbf{Contexto:} El usuario puede acceder a la información de un valor pero no puede compararlo con otros. Además, aunque tenga valores comprados o en seguimiento, realmente, ahora mismo, no se le está ofreciendo capacidad analítica, así que hay que proporcionársela. 


\item
\textbf{Tareas:}
	\begin{itemize}
	\tightlist
	\item 
	Añadir dos índices adicionales, con sus valores, para ampliar la información disponible y ofrecer un mayor abanico comparativo. 
	\item
	Agregar una funcionalidad que permita ver la correlación de un valor con todos los demás disponibles. 
		\begin{itemize}
		\tightlist
		\item 
		Como la correlación puede ser positiva o negativa, hacer diferenciación entre ambas. 
		\item
		Además, puede que interese obtener información sobre las mayores correlaciones (positivas o directas; y negativas o inversas)
		\item
		Una vez obtenidas las correlaciones, sobre los precios de cierre, ver la información en grafos de \emph{NetworkX}. 
		\end{itemize}
	\item
	Adicionalmente, intentar permitir al usuario ver los gráficos de la mayor correlación de forma conjunta. 
	\item
	Ampliar los tests a la nueva funcionalidad. 
	\item
	Integrar los grafos en la plataforma web de forma temporal para cada usuario (sin almacenar en el servidor).
  	\end{itemize}
\end{itemize}


\subsubsection{Sprint 9. Aplicación de modelos. \emph{Forecasting} para series temporales}

\begin{itemize}
\item  
\textbf{Duración:} 13/03/2024 a 27/03/2024

\item
\textbf{Objetivo:} Aportar una visión diferenciadora al análisis de valores cotizados, permitiendo que un usuario pueda aplicar un modelo que le permita tener una idea intuitiva de la futura evolución de la cotización. En cualquier caso, se informará al usuario de que los datos aportados por el modelo no pueden ser considerados como una fuente única para tomar decisiones de inversión. 


\item
\textbf{Contexto:} En las webs que ofrecen información de productos cotizados no se ofrece un apartado de experimentación con modelos de \emph{forecasting}. Puede ser un factor diferenciador para aquellos usuarios que entiendan los riesgos y las ventajas de aplicar estos modelos.  


\item
\textbf{Tareas:}
	\begin{itemize}
	\tightlist
	\item 
	Añadir el análisis a través de un modelo ARIMA. 
	\item
	Añadir el análisis a través de una red LSTM. 
	\item
	Permitir al usuario ver alguna información gráfica que facilite la comprensión de los resultados. 
	\item
	Ampliar los tests de las nuevas funcionalidades. 
	\item
	Integrar las gráficas o diagramas en la plataforma web de forma temporal para cada usuario (sin almacenar en el servidor).
  	\end{itemize}
\end{itemize}


\subsubsection{Sprint 10. Documentación y limpieza de código y avanzar considerablemente en la memoria.}

\begin{itemize}
\item  
\textbf{Duración:} 10/04/2024 a 24/04/2024

\item
\textbf{Objetivo:} Documentación de código y de la memoria. 


\item
\textbf{Contexto:} Actualmente el proyecto está documentado con notas internas del autor y debe empezarse a documentar utilizando las herramientas y guías oficiales. 


\item
\textbf{Tareas:}
	\begin{itemize}
	\tightlist
	\item 
	Configurar \emph{MikTex}.
	\item
	Descargar y preparar plantilla oficial de Trabajos de Fin de Grado. 
	\item
	Añadir resumen / abstract. 
	\item
	Documentar introducción. 
	\item
	Describir objetivos del proyecto. 
	\item
	Documentar conceptos teóricos. 
	\item
	Explicar las técnicas y herramientas utilizadas. 
	\item
	Indicar aspectos relevantes del desarrollo del proyecto que se hayan dado hasta ahora. 
	\item
	Detallar trabajos relacionados.
	\item
	Añadir referencias bibliográficas en todos los apartados de la memoria. 
	\item
	Medir la calidad del código e introducir las mejoras necesarias. 
  	\end{itemize}
\end{itemize}


\subsubsection{Sprint 11. Gráfica de Markowitz con frontera efciente y ratio de Sharpe. Mejoras visuales.}

\begin{itemize}
\item  
\textbf{Duración:} 09/05/2024 a 19/05/2024

\item
\textbf{Objetivo:} Favorecer el estudio de una cartera de valores a través de la distribución de pesos de los mismos. Mostrar una gráfica de Markowitz con la cartera ideal encontrada por simulación de Montecarlo y, también, por optimización de funciones. Realizar pequeñas mejoras estéticas y funcionales. 


\item
\textbf{Contexto:} Es necesario mejorar el \emph{DashBoard} de los usuarios para que reciban una información más completa y, de entre las opciones barajadas, una de las más interesantes es la de mostrar cuál sería la mejor distribución de los valores que ya tenga en cartera el usuario, basándome en los retornos y pesos en la propia cartera. 


\item
\textbf{Tareas:}
	\begin{itemize}
	\tightlist
	\item 
	Añadir nuevos métodos que permitan hacer una simulación de Montecarlo con diferentes pesos de valores en una cartera. 
	\item
	Añadir nuevos métodos al \emph{DashBoard} para calcular la frontera eficiente. 
	\item
	Desarrollar las funciones de optimización que permitan calcular la mejor distribución de pesos.
	\item
	Preparar los métodos necesarios para mostrar la gráfica de Markowitz junto con la frontera eficiente y las mejores carteras encontradas. 
	\item
	Implementar los tests necesarios para las nuevas funcionalidades. 
	\item
	Integrar en las plantillas existentes del \emph{DashBoard} para mostrar. 
	\item
	Documentar de forma interna en el apartado de conceptos teóricos (apartado 3 de la memoria del TFG). 
	\item
	Mejoras estéticas en botones y página principal. 
	\item
	Añadir referencias bibliográficas en todos los apartados de la memoria aunque sea a nivel local. 
	\item
	Medir la calidad de nuevo código y testear de forma automática con \emph{GitHub actions}. 
	\item
	Desplegar nueva documentación de código de forma automática con \emph{Sphinx}. 
  	\end{itemize}
\end{itemize}



\subsubsection{Sprint 12. Mejorar gráficas ARIMA. Incluir estrategias basadas en \emph{machine learning}}

\begin{itemize}
\item  
\textbf{Duración:} 22/05/2024 a 26/05/2024

\item
\textbf{Objetivo:} Completar documentación teórica de ARIMA en memoria, añadir apartado de \emph{trading} algorítmico con su documentación y contemplar redes LSTM como posible mejora del proyecto. 


\item
\textbf{Contexto:} Ahora mismo se puede acceder a un apartado de \emph{forecasting} con redes LSTM pero no me convencen los resultados obtenidos, así que se dejará como una de las posibles mejoras del proyecto. Sin embargo, voy a añadir un apartado nuevo al \emph{Lab} sobre \emph{trading} algorítmico. 

Además, tengo que terminar la documentación de ARIMA en la memoria y añadir la nueva documentación de \emph{trading} algorítmico. 


\item
\textbf{Tareas:}
	\begin{itemize}
	\tightlist
	\item 
	Terminar documentación ARIMA. 
	\item
	Mejorar gráficas y plantillas HTML para ARIMA.  
	\item
	Incluir mejoras de estrategia basada en ML, en lugar de redes LSTM, y documentar. 
	\item
	Pasar redes LSTM a apartado de posibles mejoras del proyecto. 
	\item
	Realizar nuevos tests para estrategias basadas en ML.
	\item
	Crear un \emph{release} con el cambio, porque es relevante. 
  	\end{itemize}
\end{itemize}



\subsubsection{Sprint 13. Finalizar memoria. Empezar anexos. Correcciones en plantillas y en \emph{cron} de servidor web}

\begin{itemize}
\item  
\textbf{Duración:} 31/05/2024 a 04/06/2024

\item
\textbf{Objetivo:} Finalizar memoria y empezar anexos. Realizar mejoras estéticas. Corregir tarea \emph{cron} de actualización de BDs en servidor web. Realizar la \emph{release} que está pendiente desde el \emph{sprint} anterior, no realizado por falta de comprobaciones. 


\item
\textbf{Contexto:} En este momento es posible trabajar con todas las herramientas, pero quedan algunos detalles que mejorar antes de realizar un \emph{release}. Además, es necesario avanzar con la documentación de la memoria (que ya tiene cubierta toda la parte teórica). 


\item
\textbf{Tareas:}
	\begin{itemize}
	\tightlist
	\item 
	Finalizar sección 5 de memoria.
	\item
	Finalizar sección 6 de memoria.
	\item
	Finalizar sección 7 de memoria.
	\item
	Empezar anexos. 
	\item
	Mejorar gráficas y plantillas HTML para estrategias basadas en ML. 
	\item
	Ampliar tests para cubrir todos los métodos. 
	\item
	Mejorar la calidad de código. Comprobar con \emph{pylint}.  
	\item
	Mejoras gráficas en grafos de correlaciones de \emph{NetworkX} y adaptación de plantillas en local.
	\item
	Corregir problemas de tarea \emph{cron} en servidor web. 
	\item
	Crear un release con los cambios, siempre y cuando haya terminado todas las correcciones.
  	\end{itemize}
\end{itemize}




\section{Estudio de viabilidad}

\subsection{Viabilidad económica}

Se va a realizar una estimación de costes lo más aproximada a la realidad, como si el proyecto se hubiese realizado en un entorno empresarial. 

\subsubsection{Estudio de costes}

De manera general se pueden considerar costes de personal, de \emph{hardware} y \emph{software} y, en caso de necesidad, costes de infraestructura. La mayor partida económica se tiene que destinar a los recursos humanos:

\tablaSmallSinColores{Costes en RRHH}{l|r|r}{costes_personal}{
\textbf{Concepto} & \textbf{Coste anual} & \textbf{Prorrateo (6 meses)}\\
}{
\textit{Salario bruto} & 25.000,00€ & 12.500,00€\\
\textit{Retención IRPF} & 4.250,00€ & 2.125,00€\\
\textit{Seguridad Social} & 1.587,50€ & 793,75€\\
\textit{Salario neto} & 19.162,50€ & 9.581,25€\\ 
}

Se aplica un 17\% de retención sobre la nómina, considerando ausencia de deducciones tributarias.

A continuación se muestran los costes de \emph{hardware} y \emph{software}. Para el \emph{hardware} se estima una amortización de 5 años y una utilización de 6 meses. El \emph{software}, por su parte, tendrá una amortización estimada de 2 años. 

\tablaSmallSinColores{Costes de \emph{HW} y \emph{SW}}{l|r|r}{costes_hw_sw}{
\textbf{Concepto} & \textbf{Coste} & \textbf{Coste amortizado} \\
}{
\textit{Ordenador portátil} & 1.600,00€ & 160,00€\\
\textit{Licencia MS Windows 10 pro} & 279,00€ & 69,75€\\
}

Por último, se realiza un análisis de costes de infraestructura (solo asimilable en caso de necesidad) de nuevo, el coste amortizado se calcula a 6 meses:

\tablaSmallSinColores{Costes infraestructura}{l|r|r}{costes_infraestructura}{
\textbf{Concepto} & \textbf{Coste anual} & \textbf{Coste amortizado} \\
}{
\textit{Consumo eléctrico} & 350,00€ & 175,00€\\
\textit{Espacio \emph{coworking}} & 4.000,00€ & 2.000,0€\\
\textit{Material de oficina} & 10,00€ & 5,00€\\
\textit{Alojamiento web} & 72,00€ & 36,00€\\
}


Los costes totales estimados son:

\tablaSmallSinColores{Costes totales}{l|r}{coste_total}{
\textbf{Concepto} & \textbf{Coste}\\
}{
\textit{Costes en RRHH} & 12.500,00€\\
\textit{Costes de \emph{HW} y \emph{SW}} & 229.75€\\
\textit{Costes infraestructura} & 2.216,00€\\
\textit{\textbf{TOTAL}} & 14.945,75€\\
}

\subsubsection{Análisis de beneficios}

Este trabajo no se plantea inicialmente como un mecanismo para generar beneficios económicos, sino como una herramienta gratuita para ayudar a los inversores a mejorar la distribución de sus carteras y para que los analistas financieros experimenten con diferentes técnicas. 

En caso de monetizar el proyecto se podría optar por incluir publicidad en la web para generar ingresos que, al menos, cubran los gastos anuales. 

\subsection{Viabilidad legal}

\subsubsection{Descargo de responsabilidades}

El \emph{software} que se proporciona en este trabajo es solo para fines informativos y no debe considerarse como asesoramiento financiero. La información proporcionada no está garantizada como precisa o completa y puede cambiar sin previo aviso.

El usuario es el único responsable de sus decisiones de inversión y de las consecuencias de las mismas. No se debe confiar en este \emph{software} para tomar decisiones de inversión sin realizar una investigación profunda y consultar con un asesor financiero calificado.

\subsubsection{Licencias \emph{software}}

En este trabajo se utilizan múltiples librerías de terceros. Además, aunque los datos son almacenados en bases de datos propias, hay que considerar que se hace uso de información externa a través de la API \emph{yFinance}. 

Las licencias asociadas a las librerías y APIs utilizadas, se detallan en la siguiente tabla:

\tablaSmallSinColores{Licencias de terceros}{l|r}{coste_total}{
\textbf{Librería / API} & \textbf{Licencia}\\
}{
\textit{Python} & OSI-Open Source\\
\textit{Django} & BSD\\
\textit{yFinance} & Apache 2.0\\
\textit{Pandas} & BSD\\
\textit{Numpy} & BSD 3-Clause\\
\textit{Scikit-Learn} & BSD 3-Clause\\
\textit{Scipy} & BSD\\
\textit{NetworkX} & BSD 3-Clause\\
\textit{Matplotlib} & Licencia libre propia\\
\textit{Plotly} & MIT\\
\textit{Statsmodels} & BSD 3-Clause\\
\textit{Feedparser} & BSD 3-Clause\\
\textit{NewsAPI} & MIT\\
\textit{pmdarima} & MIT\\
\textit{Keras} & Apache 2.0\\
}

Todas las librerías y el \emph{framework Django} utilizan licencias de código abierto permisivas - la más restrictiva sería Apache 2.0 - lo que significa que el código de \emph{FAT: Financial Analysis Tool} se puede usar, modificar y distribuir libremente, incluyendo para fines comerciales, siempre que se cumplan los requisitos de atribución y exención de responsabilidad establecidos en cada licencia.

Por tanto, se toma la decisión de publicar este proyecto bajo licencia CC BY-NC-SA 4.0\citep{online:licencia}.

\subsubsection{Consideraciones adicionales}

Dada la condición académica de este proyecto no hay limitación en el uso de los datos financieros almacenados y no se recoge información personal; pero dependiendo del uso que se dé a este \emph{software} es posible que se deba cumplir con regulaciones específicas, como la Ley de Protección de Datos de la Unión Europea (GDPR) o la Ley de Protección de la Privacidad de la Información Financiera (FINRA). 












