\apendice{Especificación de Requisitos}

\section{Introducción}

En este anexo se recogen los requisitos y casos de uso que se han tenido en cuenta para el desarrollo de este trabajo. Se han seguido las recomendaciones del estándar IEEE 830-1998\citep{1998IEEESpecifications} como guía de buenas prácticas.

En este proyecto se ha tratado de crear una especificación de requisitos que cumpla con las siguientes condiciones:

\begin{itemize}
\tightlist
\item
	Completa y consistente
	\begin{itemize}
	\item 
	Incluir todos los requisitos y referencias necesarias.
	\item
	Ser coherente con los propios requisitos y otros documentos de especificación.
	\end{itemize}
\item
	Claridad y accesibilidad
	\begin{itemize}
	\item 
	Redacción clara para evitar malas interpretaciones.
	\item
	Uso de términos y definiciones precisos.
	\end{itemize}
\item
	Verificabilidad y trazabilidad
	\begin{itemize}
	\item 
	Debe de existir un método finito y sin costo para probar los requisitos.
	\item
	Uso de términos y definiciones precisos.
	\end{itemize}
\item
	Modificabilidad y priorización
	\begin{itemize}
	\item 
	Fácilmente modificable.
	\item
	Jerarquía de priorización según relevancia para el negocio (o fin).
	\end{itemize}
\item
	Correctitud y accesibilidad
	\begin{itemize}
	\item 
	El software debe cumplir con los requisitos de la especificación.
	\item
	Accesibilidad y facilidad de comprensión para los usuarios y desarrolladores.
	\end{itemize}
\end{itemize}


\section{Objetivos generales}\label{objetivos-generales}

Este trabajo persigue los siguientes objetivos generales:

\begin{itemize}
\item
Proporcionar una herramienta que otorgue capacidad crítica en inversiones personales.
\item
El sistema tiene que poder controlar la evolución de valores en cartera y en seguimiento. 
\item
Tiene que mostrarse información relevante en cuanto a distribución de carteras y posibles pesos para los valores escogidos. 
\item
Permitir trabajar, de manera experimental, con herramientas de \emph{trading} algorítmico. 
\item
Facilitar el acceso a la información con una página web pública y gratuita. 
\end{itemize}


\section{Catálogo de requisitos}

\subsection{Requisitos funcionales}

\begin{itemize}
\item 
\textbf{RF-1 Mostrar portada con información general de diferentes índices bursátiles:} la web tiene que tener información agregada de todos los índices disponibles. 
    \begin{itemize}
        \item \textbf{RF-1.1 Mostrar carrusel de noticias generales}: en la portada se 
        tiene que disponer de noticias relevantes dentro del mundo bursátil.
        \item \textbf{RF-1.2 Mostrar mejores y peores valores de cada índice}: se tiene que 
        poder consultar, de forma ágil, cuáles han sido los mejores y peores valores de 
        cada índice.
        \begin{itemize}
            \item \textbf{RF-1.2.1 Mostrar gráfica}: tiene que estar disponible una gráfica 
            evolutiva junto a cada valor de los mostrados. 
        \end{itemize}
	\end{itemize}

\item
\textbf{RF-2 Control de usuarios}: los usuarios tendrán acceso a información adicional si
están dados de alta. 
	\begin{itemize}
        \item \textbf{RF-2.1 Permitir registro}: el usuario se podrá registrar. 
        \item \textbf{RF-2.2 Permitir hacer \emph{login}}: el usuario podrá acceder con su 
        cuenta en la web.
        \item \textbf{RF-2.3 Permitir \emph{logout}}: el usuario podrá cerrar su sesión. 
	\end{itemize}

\item
\textbf{RF-3 Gestionar \emph{DashBoard}}: tiene que haber una zona de usuario, denominada \emph{DashBoard}, con información relevante sobre valores seleccionados por el usuario. 
	\begin{itemize}
        \item \textbf{RF-3.1 Crear nuevo valor en cartera}: el usuario podrá añadir un 
        nuevo valor, con fecha y precio de compra, a su cartera.
        \begin{itemize}
            \item \textbf{RF-3.1.1 Mostrar 'donut', sectores y divisas}: si un usuario 
            tiene valores en cartera, éste contará con diagramas de información relevante.
            \item \textbf{RF-3.1.2 Ver gráfica de Markowitz, ratio de Sharpe y pesos}: si 
            un usuario tiene valores en cartera podrá ver lla mejor distribución de pesos 
            para la misma. 
            \item \textbf{RF-3.1.3 Ver posiciones abiertas y evolución de cartera}: si hay 
            valores en cartera se podrá comprobar la evolución de la misma. 
        \end{itemize}
		\item \textbf{RF-3.2 Eliminar valor de cartera}: el usuario podrá eliminar un 
        valor de los que tuviera almacenados. 
        \item \textbf{RF-3.3 Crear un nuevo valor en seguimiento}: el usuario podrá 		
        hacer seguimiento de valores de su interés.
        \item \textbf{RF-3.4 Eliminar valor de seguimiento}: el usuario podrá eliminar un 
        valor de los que tuviera en seguimiento. 
	\end{itemize}

\item
\textbf{RF-4 Consultar índice}: se debe presentar, de forma ordenada, información relevante de cada índice por separado.  
	\begin{itemize}
        \item \textbf{RF-4.1 Mostrar tabla de valores}: al consultar un índice se mostrará 
        una tabla con todos los componentes de ese índice, con información relevante sobre 
        la última sesión disponible. 
        \item \textbf{RF-4.2 Consultar un valor del índice}: se podrá consultar información 
        de un único valor. 
        \begin{itemize}
            \item \textbf{RF-4.2.1 Dar acceso a gráfica interactiva}: se ofrecerá una 
            gráfica interactiva, con medias móviles y volumen. 
            \item \textbf{RF-4.2.2 Mostrar distribución de retornos}: habrá una gráfica con 
            la distribución que hayan seguido los retornos en los últimos meses. 
            \item \textbf{RF-4.2.3 Mostrar datos del último mes}: disponer datos en formato 
            tabular para ver precios de apertura y cierre de los últimos 30 días. 
            \item \textbf{RF-4.2.4 Ver evolución del sector}: facilitar una comparativa con 
            el sector de referencia del valor. 
            \item \textbf{RF-4.2.5 Comparar con otros valores}: existirá la posibilidad de 
            comparar gráficas de precios de cierre con otros valores. 
            \item \textbf{RF-4.2.6 Mostrar grafos de correlación}: mostrar grafos de alta 
            correlación positiva y negativa con otros valores. 
        \end{itemize}
        \item \textbf{RF-4.3 Mostrar noticias relacionadas}: facilitar enlaces de fuentes 
        RSS que estén relacionadas con el índice.
        \item \textbf{RF-4.4 Ver gráfica de evolución}: mostrar gráfica de evolución del
        índice en su conjunto. 
	\end{itemize}	



\item
\textbf{RF-5 Gestionar \emph{Lab}}: dar acceso a un laboratorio virtual. 
	\begin{itemize}
        \item \textbf{RF-5.1 Trabajar con ARIMA}: permitir realizar estimaciones y búsqueda 
        de parámetros con ARIMA. 
        \begin{itemize}
            \item \textbf{RF-5.1.1 Buscar parámetros (p,d,q) con fnuciones ACF y PACF}: se 
            mostrarán gráficas de funciones para interpretación del usuario. 
            \item \textbf{RF-5.1.2 Introducir (p,d,q) de forma manual}: el usuario podrá 
            introducir los parámetros deseados para aplicar un modelo ARIMA. 
            \item \textbf{RF-5.1.3 Buscar (p,d,q) de forma automática}: se proporcionará 
            una funcionalidad de búsqueda automática de los parámetros más adecuados a un 
            valor.
            \item \textbf{RF-5.1.4 Hacer búsqueda por rejilla para (p,d,q)}: se podrá 
            realizar una búsqueda de parámetros (p,d,q) de entre una serie de posibles 
            valores preestablecidos.             
        \end{itemize}
        \item \textbf{RF-5.2 Trabajar con \emph{trading} algorítmico}: facilitar el acceso 
        a herramientas de \emph{trading} algorítmico.
        \begin{itemize}
            \item \textbf{RF-5.2.1 Usar algoritmo de cruce de medias}: se mostrará el 
            resultado de buscar las mejores medias móviles para un valor, en un período de 
            tiempo concreto. 
            \item \textbf{RF-5.2.2 Utilizar estrategia basada en \emph{machine learning}}: 
            permitir interactuar con modelos de regresión y clasificación para estimar la 
            tendencia de la próxima sesión de un índice.
        \end{itemize}
	\end{itemize}

\end{itemize}


\subsection{Requisitos no funcionales}

En este apartado se tratará de dar detalle de aquellas características que no son funcionales pero que aportan un valor añadido al proyecto y a la interacción con la herramienta desarrollada:

\begin{itemize}
\item
\textbf{RNF-1 Escalabilidad}: la web tiene que permitir y favorecer, en la medida de lo posible, la incorporación de nuevas funcionalidades y de bases de datos adicionales. 

\item
\textbf{RNF-2 Privacidad}: los datos de usuario, como nombres o contraseñas, se debe gestionar de forma segura. 

\item
\textbf{RNF-3 Disponibilidad}: la web debe ser compatible con los navegadores más modernos y tendrá alta disponibilidad a través de un servicio de alojamiento fiable. 

\item
\textbf{RNF-4 Usabilidad}: la interfaz de la web será \emph{user friendly}, resultar intuitiva y facilitará comentarios de ayuda adicional en los apartados más especializados. 

\item
\textbf{RNF-5 Mantenibilidad}: se tiene que favorecer un mantenimiento posterior a la puesta en producción. Además, se facilitará información a los desarrolladores para que puedan realizar mejoras incrementales posteriores. 
\end{itemize}


\section{Especificación de requisitos}

\subsection{Casos de uso}\label{casos_uso}

A continuación se muestran todos los casos de uso contemplados. Muchos de estos casos de uso se plantearon en las fases iniciales del proyecto y se fueron mejorando en las sucesivas iteraciones de cada \emph{sprint}. Se trató de realizar mejoras incrementales para satisfacer las expectativas de un inversor o potencial cliente:

\imagenSinMargen{img_anex_01_casos_uso.png}{Diagrama de casos de uso. Fuente: elaboración propia}


\begin{table}[p]
	\centering
	\begin{tabularx}{\linewidth}{ p{0.21\columnwidth} p{0.71\columnwidth} }
		\toprule
		\textbf{CU-1}    & \textbf{Consultar portada}\\
		\toprule
		\textbf{Versión}              & 1.0    \\
		\textbf{Autor}                & Rodrigo Merino Tovar \\
		\textbf{Requisitos asociados} & RF-1, RF-1.1, RF-1.2 y RF-1.2.1 \\
		\textbf{Descripción}          & Permite al usuario ver la página principal\\
		\textbf{Precondición}         & La base de datos se encuentra disponible \\
		\textbf{Acciones}             &
		\begin{enumerate}
			\def\labelenumi{\arabic{enumi}.}
			\tightlist
			\item El usuario accede a la página web.
			\item Se muestran noticias del mundo bursátil. 
			\item Se listan los índices con los mejores y peores valores
		\end{enumerate}\\
		\textbf{Postcondición}        & El número de valores mostrados es múltiplo del 
										número de índices disponibles.  \\
		\textbf{Excepciones}          & 
		\begin{itemize}
			\tightlist
			\item Error al cargar valores (mensaje de error).
			\item Error al cargar una noticia (pasar a siguiente). 
		\end{itemize}\\
		\textbf{Importancia}          & Alta \\
		\bottomrule
	\end{tabularx}
	\caption{CU-1 Consultar portada.}
\end{table}


\begin{table}[p]
	\centering
	\begin{tabularx}{\linewidth}{ p{0.21\columnwidth} p{0.71\columnwidth} }
		\toprule
		\textbf{CU-2}    & \textbf{Mostrar carrusel noticias generales}\\
		\toprule
		\textbf{Versión}              & 1.0    \\
		\textbf{Autor}                & Rodrigo Merino Tovar \\
		\textbf{Requisitos asociados} & RF-1, RF-1.1 \\
		\textbf{Descripción}          & Mostrar carrusel de noticias bursátiles\\
		\textbf{Precondición}         & La API \emph{NewsAPI} está activa \\
		\textbf{Acciones}             &
		\begin{enumerate}
			\def\labelenumi{\arabic{enumi}.}
			\tightlist
			\item El usuario ve texto preliminar de una noticia y su imagen asociada. 
			\item Se \emph{navega} entre noticias con selectores laterales o se dejan pasar automáticamente.
			\item Se muestra un botón para \emph{leer más} en la fuente original.
		\end{enumerate}\\
		\textbf{Postcondición}        & La imagen y el enlace externo se corresponden con el texto de la noticia.  \\
		\textbf{Excepciones}          & 
		\begin{itemize}
			\tightlist
			\item Error al cargar una noticia (pasar a siguiente). 
		\end{itemize}\\
		\textbf{Importancia}          & Media \\
		\bottomrule
	\end{tabularx}
	\caption{CU-2 Mostrar carrusel noticias generales.}
\end{table}


\begin{table}[p]
	\centering
	\begin{tabularx}{\linewidth}{ p{0.21\columnwidth} p{0.71\columnwidth} }
		\toprule
		\textbf{CU-3}    & \textbf{Mostrar mejores y peores valores}\\
		\toprule
		\textbf{Versión}              & 1.0    \\
		\textbf{Autor}                & Rodrigo Merino Tovar \\
		\textbf{Requisitos asociados} & RF-1, RF-1.2 y RF-1.2.1 \\
		\textbf{Descripción}          & Mostrar 3 mejores y 3 peores valores de cada índice en la última sesión bursátil.\\
		\textbf{Precondición}         & Las bases de datos están disponibles. \\
		\textbf{Acciones}             &
		\begin{enumerate}
			\def\labelenumi{\arabic{enumi}.}
			\tightlist
			\item El usuario ve valores de cierre de última sesión de mejores valores de un 			índice. 
			\item El usuario ve valores de cierre de última sesión de peores valores de un 				índice. 
			\item Se muestra gráfica asociada a cada valor.
		\end{enumerate}\\
		\textbf{Postcondición}        & Las gráficas se corresponden con los valores 												mejores y peores y se muestran en orden de mejor a 											peor  \\
		\textbf{Excepciones}          & 
		\begin{itemize}
			\tightlist
			\item Error al cargar un valor (mensaje de error). 
		\end{itemize}\\
		\textbf{Importancia}          & Media \\
		\bottomrule
	\end{tabularx}
	\caption{CU-3 Mostrar mejores y peores valores.}
\end{table}


\begin{table}[p]
	\centering
	\begin{tabularx}{\linewidth}{ p{0.21\columnwidth} p{0.71\columnwidth} }
		\toprule
		\textbf{CU-4}    & \textbf{Mostrar gráfica}\\
		\toprule
		\textbf{Versión}              & 1.0    \\
		\textbf{Autor}                & Rodrigo Merino Tovar \\
		\textbf{Requisitos asociados} & RF-1, RF-1.2, RF-1.2.1 y RF-4.4 \\
		\textbf{Descripción}          & Mostrar gráficas de valores o índices.\\
		\textbf{Precondición}         & Las bases de datos están disponibles. \\
		\textbf{Acciones}             &
		\begin{enumerate}
			\def\labelenumi{\arabic{enumi}.}
			\tightlist
			\item El usuario ve gráficas de última sesión de mejores valores de un 						índice. 
			\item El usuario ve gráficas de última sesión de peores valores de un 						índice. 
			\item Se muestran valores de cierre asociados a cada valor.
			\item El usuario consulta gráfica de índice.
		\end{enumerate}\\
		\textbf{Postcondición}        & Las gráficas se corresponden con los valores 												mejores y peores y se muestran en orden de mejor a peor. O la gráfica del índice consultado está disponible. \\
		\textbf{Excepciones}          & 
		\begin{itemize}
			\tightlist
			\item Error al cargar un valor o índice (mensaje de error). 
		\end{itemize}\\
		\textbf{Importancia}          & Media \\
		\bottomrule
	\end{tabularx}
	\caption{CU-4 Mostrar gráfica.}
\end{table}


\begin{table}[p]
	\centering
	\begin{tabularx}{\linewidth}{ p{0.21\columnwidth} p{0.71\columnwidth} }
		\toprule
		\textbf{CU-5}    & \textbf{Gestionar usuario}\\
		\toprule
		\textbf{Versión}              & 1.0    \\
		\textbf{Autor}                & Rodrigo Merino Tovar \\
		\textbf{Requisitos asociados} & RF-2, RF-2.1, RF-2.2 y RF-2.3 \\
		\textbf{Descripción}          & Controlar sesión de usuario.\\
		\textbf{Precondición}         & La base de datos de usuarios está disponible. \\
		\textbf{Acciones}             &
		\begin{enumerate}
			\def\labelenumi{\arabic{enumi}.}
			\tightlist
			\item El usuario se registra. 
			\item El usuario hace \emph{login}. 
			\item Diferentes acciones dependiendo de la labor a realizar.
			\item El usuario hace \emph{logout} para cerrar sesión.
		\end{enumerate}\\
		\textbf{Postcondición}        & Tras \emph{logout} la cookie de sesión se inhabilita. \\
		\textbf{Excepciones}          & 
		\begin{itemize}
			\tightlist
			\item Error al registrar por contraseña o nombre (mensaje informativo). 
			\item Error en \emph{login} por contraseña o nombre (mensaje informativo). 
		\end{itemize}\\
		\textbf{Importancia}          & Alta \\
		\bottomrule
	\end{tabularx}
	\caption{CU-5 Gestionar usuario.}
\end{table}


\begin{table}[p]
	\centering
	\begin{tabularx}{\linewidth}{ p{0.21\columnwidth} p{0.71\columnwidth} }
		\toprule
		\textbf{CU-6}    & \textbf{Registro}\\
		\toprule
		\textbf{Versión}              & 1.0    \\
		\textbf{Autor}                & Rodrigo Merino Tovar \\
		\textbf{Requisitos asociados} & RF-2 y RF-2.1 \\
		\textbf{Descripción}          & Permitir registro de usuarios.\\
		\textbf{Precondición}         & El usuario no está registrado. \\
		\textbf{Acciones}             &
		\begin{enumerate}
			\def\labelenumi{\arabic{enumi}.}
			\tightlist
			\item El usuario se registra. 
			\item Mostrar un mensaje de bienvenida. 
			\item Dejar usuario ya activo con \emph{login}.
			\item Redirigir a página principal.
		\end{enumerate}\\
		\textbf{Postcondición}        & Nuevo usuario en base de datos. \\
		\textbf{Excepciones}          & 
		\begin{itemize}
			\tightlist
			\item Error al registrar por contraseña o nombre (mensaje informativo). 
			\item Error por usuario duplicado (mensaje informativo). 
		\end{itemize}\\
		\textbf{Importancia}          & Alta \\
		\bottomrule
	\end{tabularx}
	\caption{CU-6 Registro.}
\end{table}


\begin{table}[p]
	\centering
	\begin{tabularx}{\linewidth}{ p{0.21\columnwidth} p{0.71\columnwidth} }
		\toprule
		\textbf{CU-7}    & \textbf{Login}\\
		\toprule
		\textbf{Versión}              & 1.0    \\
		\textbf{Autor}                & Rodrigo Merino Tovar \\
		\textbf{Requisitos asociados} & RF-2 y RF-2.2 \\
		\textbf{Descripción}          & Permitir \emph{login} de usuarios.\\
		\textbf{Precondición}         & El usuario no está logueado. \\
		\textbf{Acciones}             &
		\begin{enumerate}
			\def\labelenumi{\arabic{enumi}.}
			\tightlist
			\item El usuario hace \emph{login}. 
			\item Mostrar \emph{DashBoard}. 
			\item Permitir acceso a funciones adicionales.
		\end{enumerate}\\
		\textbf{Postcondición}        & Usuario logueado y con acceso a funciones adicionales. \\
		\textbf{Excepciones}          & 
		\begin{itemize}
			\tightlist
			\item Error al loguear (mensaje informativo y repetición). 
			\item Error forzado (mensaje informativo). 
		\end{itemize}\\
		\textbf{Importancia}          & Alta \\
		\bottomrule
	\end{tabularx}
	\caption{CU-6 Login.}
\end{table}


\begin{table}[p]
	\centering
	\begin{tabularx}{\linewidth}{ p{0.21\columnwidth} p{0.71\columnwidth} }
		\toprule
		\textbf{CU-8}    & \textbf{Logout}\\
		\toprule
		\textbf{Versión}              & 1.0    \\
		\textbf{Autor}                & Rodrigo Merino Tovar \\
		\textbf{Requisitos asociados} & RF-2 y RF-2.3 \\
		\textbf{Descripción}          & Permitir \emph{logout} de usuarios.\\
		\textbf{Precondición}         & El usuario está logueado. \\
		\textbf{Acciones}             &
		\begin{enumerate}
			\def\labelenumi{\arabic{enumi}.}
			\tightlist
			\item El usuario hace \emph{logout}. 
			\item Mostrar página principal. 
			\item No permitir acceso a funciones adicionales.
		\end{enumerate}\\
		\textbf{Postcondición}        & Usuario no logueado y sin acceso a funciones adicionales. \\
		\textbf{Excepciones}          & \\
		\textbf{Importancia}          & Baja \\
		\bottomrule
	\end{tabularx}
	\caption{CU-8 Logout.}
\end{table}


\begin{table}[p]
	\centering
	\begin{tabularx}{\linewidth}{ p{0.21\columnwidth} p{0.71\columnwidth} }
		\toprule
		\textbf{CU-9}    & \textbf{Gestionar \emph{DashBoard}}\\
		\toprule
		\textbf{Versión}              & 1.0    \\
		\textbf{Autor}                & Rodrigo Merino Tovar \\
		\textbf{Requisitos asociados} & RF-3, RF-3.1, RF-3.1.1, RF-3.1.2, RF-3.1.3, RF-3.2, RF-3.3 y RF-3.4  \\
		\textbf{Descripción}          & Mostrar información agregada de valores en cartera y en seguimiento de un usuario.\\
		\textbf{Precondición}         & El usuario está logueado y las bases de datos están disponibles.  \\
		\textbf{Acciones}             &
		\begin{enumerate}
			\def\labelenumi{\arabic{enumi}.}
			\tightlist
			\item El usuario hace \emph{login}. 
			\item Mostrar página de \emph{DashBoard} con información de usuario.
		\end{enumerate}\\
		\textbf{Postcondición}        & El usuario puede consultar precios, añadir o quitar valores en cartera y añadir o quitar valores en seguimiento. \\
		\textbf{Excepciones}          & 
		\begin{itemize}
			\tightlist
			\item No hay valores previos en cartera (se muestran tablas vacías).
			\item No hay valores previos en seguimiento (se muestran tablas vacías).
		\end{itemize} \\
		\textbf{Importancia}          & Alta \\
		\bottomrule
	\end{tabularx}
	\caption{CU-9 \emph{DashBoard}.}
\end{table}


\begin{table}[p]
	\centering
	\begin{tabularx}{\linewidth}{ p{0.21\columnwidth} p{0.71\columnwidth} }
		\toprule
		\textbf{CU-10}    & \textbf{Crear nuevo valor en cartera}\\
		\toprule
		\textbf{Versión}              & 1.0    \\
		\textbf{Autor}                & Rodrigo Merino Tovar \\
		\textbf{Requisitos asociados} & RF-3, RF-3.1, RF-3.1.1, RF-3.1.2 y RF-3.1.3  \\
		\textbf{Descripción}          & Añadir un valor a la cartera del usuario.\\
		\textbf{Precondición}         & El usuario está logueado y las bases de datos están disponibles.  \\
		\textbf{Acciones}             &
		\begin{enumerate}
			\def\labelenumi{\arabic{enumi}.}
			\tightlist
			\item El usuario accede al \emph{DashBoard}. 
			\item Pinchar en botón \emph{Nueva posición de cartera}.
		\end{enumerate}\\
		\textbf{Postcondición}        & El usuario ve un nuevo valor asociado a su cuenta. \\
		\textbf{Excepciones}          & 
		\begin{itemize}
			\tightlist
			\item Valor no existe (mensaje informativo y reintento).
			\item Fecha no válida (mensaje informativo y reintento).
			\item Precio valor no adecuado para la fecha (mensaje informativo y reintento).
			\item Valores formulario fuera de rango (mensaje informativo y reintento).
		\end{itemize} \\
		\textbf{Importancia}          & Alta \\
		\bottomrule
	\end{tabularx}
	\caption{CU-10 Crear nuevo valor en cartera.}
\end{table}


\begin{table}[p]
	\centering
	\begin{tabularx}{\linewidth}{ p{0.21\columnwidth} p{0.71\columnwidth} }
		\toprule
		\textbf{CU-11}    & \textbf{Mostrar \emph{donut}, sectores y divisas}\\
		\toprule
		\textbf{Versión}              & 1.0    \\
		\textbf{Autor}                & Rodrigo Merino Tovar \\
		\textbf{Requisitos asociados} & RF-3, RF-3.1 y RF-3.1.1 \\
		\textbf{Descripción}          & Mostrar información de composición de cartera en gráficas y diagramas que resulten agradables para el usuario.\\
		\textbf{Precondición}         & El usuario está logueado, las bases de datos están disponibles y el usuario tiene valores guardados en cartera.  \\
		\textbf{Acciones}             &
		\begin{enumerate}
			\def\labelenumi{\arabic{enumi}.}
			\tightlist
			\item El usuario accede al \emph{DashBoard}. 
			\item Mostrar \emph{donut} con valores en cartera.
			\item Mostrar diagrama de sectores en los que se está invertido.  
			\item Mostrar inversión por tipo de divisa. 
			\item Ver cambios de divisa de última sesión disponible.
		\end{enumerate}\\
		\textbf{Postcondición}        & El usuario ve información relevante sobre el estado de su cartera. \\
		\textbf{Excepciones}          & 
		\begin{itemize}
			\tightlist
			\item No hay valores previos en cartera (se muestran tablas vacías).
		\end{itemize} \\
		\textbf{Importancia}          & Media \\
		\bottomrule
	\end{tabularx}
	\caption{CU-11 Mostrar \emph{donut}, sectores y divisas.}
\end{table}


\begin{table}[p]
	\centering
	\begin{tabularx}{\linewidth}{ p{0.21\columnwidth} p{0.71\columnwidth} }
		\toprule
		\textbf{CU-12}    & \textbf{Ver gráfica de Markowitz, ratio de Sharpe
y pesos}\\
		\toprule
		\textbf{Versión}              & 1.0    \\
		\textbf{Autor}                & Rodrigo Merino Tovar \\
		\textbf{Requisitos asociados} & RF-3, RF-3.1 y RF-3.1.2 \\
		\textbf{Descripción}          & Mostrar información de distribución de pesos de los valores en cartera.\\
		\textbf{Precondición}         & El usuario está logueado, las bases de datos están disponibles y el usuario tiene valores guardados en cartera.  \\
		\textbf{Acciones}             &
		\begin{enumerate}
			\def\labelenumi{\arabic{enumi}.}
			\tightlist
			\item El usuario accede al \emph{DashBoard}. 
			\item Mostrar gráfica de Markowitz.
			\item Ver simulación de Montecarlo de 10.000 carteras.  
			\item Indicar frontera eficiente. 
			\item Mostrar ratio de Sharpe. 
			\item Mostrar información de distribución de pesos en tablas.
		\end{enumerate}\\
		\textbf{Postcondición}        & El usuario ve información relevante sobre posibles distribuciones mejores para su cartera. \\
		\textbf{Excepciones}          & 
		\begin{itemize}
			\tightlist
			\item No hay valores previos en cartera (se muestran tablas vacías).
		\end{itemize} \\
		\textbf{Importancia}          & Alta \\
		\bottomrule
	\end{tabularx}
	\caption{CU-12 Ver gráfica de Markowitz, ratio de Sharpe y pesos.}
\end{table}


\begin{table}[p]
	\centering
	\begin{tabularx}{\linewidth}{ p{0.21\columnwidth} p{0.71\columnwidth} }
		\toprule
		\textbf{CU-13}    & \textbf{Ver posiciones abiertas y evolución de cartera}\\
		\toprule
		\textbf{Versión}              & 1.0    \\
		\textbf{Autor}                & Rodrigo Merino Tovar \\
		\textbf{Requisitos asociados} & RF-3, RF-3.1 y RF-3.1.3 \\
		\textbf{Descripción}          & Mostrar información de evolución de inversiones realizadas.\\
		\textbf{Precondición}         & El usuario está logueado, las bases de datos están disponibles y el usuario tiene valores guardados en cartera.  \\
		\textbf{Acciones}             &
		\begin{enumerate}
			\def\labelenumi{\arabic{enumi}.}
			\tightlist
			\item El usuario accede al \emph{DashBoard}. 
			\item Mostrar evolución individual de valores en cartera.
			\item Hacer cálculos de cambio de divisa a euros.  
			\item Mostrar evolución de cartera en euros.
		\end{enumerate}\\
		\textbf{Postcondición}        & El usuario ve la evolución de su cartera. \\
		\textbf{Excepciones}          & 
		\begin{itemize}
			\tightlist
			\item No hay valores previos en cartera (se muestran tablas vacías).
		\end{itemize} \\
		\textbf{Importancia}          & Alta \\
		\bottomrule
	\end{tabularx}
	\caption{CU-13 Ver posiciones abiertas y evolución de cartera.}
\end{table}


\begin{table}[p]
	\centering
	\begin{tabularx}{\linewidth}{ p{0.21\columnwidth} p{0.71\columnwidth} }
		\toprule
		\textbf{CU-14}    & \textbf{Eliminar valor de cartera}\\
		\toprule
		\textbf{Versión}              & 1.0    \\
		\textbf{Autor}                & Rodrigo Merino Tovar \\
		\textbf{Requisitos asociados} & RF-3 y RF-3.2 \\
		\textbf{Descripción}          & Eliminar valor guardado en cartera.\\
		\textbf{Precondición}         & El usuario está logueado, las bases de datos están disponibles y el usuario tiene valores guardados en cartera.  \\
		\textbf{Acciones}             &
		\begin{enumerate}
			\def\labelenumi{\arabic{enumi}.}
			\tightlist
			\item El usuario accede al \emph{DashBoard}. 
			\item Mostrar información relativa a cartera.
			\item El usuario pulsa el botón \emph{Eliminar posición de cartera}.  
			\item Mostrar lista de valores en cartera.
			\item Seleccionar valores a eliminar.
			\item Pulsar \emph{Eliminar seleccionados}.
			\item Volver al \emph{DashBoard}
		\end{enumerate}\\
		\textbf{Postcondición}        & El usuario ve datos actualizados de su cartera. \\
		\textbf{Excepciones}          & 
		\begin{itemize}
			\tightlist
			\item No hay valores posteriores en cartera (se muestran tablas vacías).
		\end{itemize} \\
		\textbf{Importancia}          & Alta \\
		\bottomrule
	\end{tabularx}
	\caption{CU-14 Eliminar valor de cartera.}
\end{table}


\begin{table}[p]
	\centering
	\begin{tabularx}{\linewidth}{ p{0.21\columnwidth} p{0.71\columnwidth} }
		\toprule
		\textbf{CU-15}    & \textbf{Crear un nuevo valor en seguimiento}\\
		\toprule
		\textbf{Versión}              & 1.0    \\
		\textbf{Autor}                & Rodrigo Merino Tovar \\
		\textbf{Requisitos asociados} & RF-3 y RF-3.3 \\
		\textbf{Descripción}          & Guardar un nuevo valor para realizar seguimiento.\\
		\textbf{Precondición}         & El usuario está logueado, las bases de datos están disponibles y el usuario no tiene el valor previamente en seguimiento.  \\
		\textbf{Acciones}             &
		\begin{enumerate}
			\def\labelenumi{\arabic{enumi}.}
			\tightlist
			\item El usuario accede al \emph{DashBoard}. 
			\item Mostrar información relativa a cartera.
			\item El usuario pulsa el botón \emph{Nuevo valor a seguir}.  
			\item Mostrar formulario de búsqueda de valores (\emph{tickers}).
			\item Seleccionar valor a seguir.
			\item Pulsar \emph{Guardar}.
			\item Buscar valores del mismo sector.
			\item Volver al \emph{DashBoard}
		\end{enumerate}\\
		\textbf{Postcondición}        & El usuario ve nuevo \emph{stock} en seguimiento junto con aquellos valores que pertenezcan al mismo sector. \\
		\textbf{Excepciones}          & 
		\begin{itemize}
			\tightlist
			\item No hay valores posteriores en seguimiento (se muestran tablas vacías).
		\end{itemize} \\
		\textbf{Importancia}          & Media \\
		\bottomrule
	\end{tabularx}
	\caption{CU-15 Crear un nuevo valor en seguimiento.}
\end{table}


\begin{table}[p]
	\centering
	\begin{tabularx}{\linewidth}{ p{0.21\columnwidth} p{0.71\columnwidth} }
		\toprule
		\textbf{CU-16}    & \textbf{Eliminar un valor de seguimiento}\\
		\toprule
		\textbf{Versión}              & 1.0    \\
		\textbf{Autor}                & Rodrigo Merino Tovar \\
		\textbf{Requisitos asociados} & RF-3 y RF-3.4 \\
		\textbf{Descripción}          & Para eliminar un valor que estuviera en seguimiento.\\
		\textbf{Precondición}         & El usuario está logueado, las bases de datos están disponibles y el usuario tiene previamente valores en seguimiento.  \\
		\textbf{Acciones}             &
		\begin{enumerate}
			\def\labelenumi{\arabic{enumi}.}
			\tightlist
			\item El usuario accede al \emph{DashBoard}. 
			\item Mostrar información relativa a cartera.
			\item El usuario pulsa el botón \emph{Eliminar valor en seguimiento}.  
			\item Mostrar valores en seguimiento).
			\item Seleccionar valores a eliminar.
			\item Pulsar \emph{Eliminar seleccionados}.
			\item Volver al \emph{DashBoard}
		\end{enumerate}\\
		\textbf{Postcondición}        & El usuario ve valores en seguimiento actualizados. \\
		\textbf{Excepciones}          & 
		\begin{itemize}
			\tightlist
			\item No hay valores previos en seguimiento (se muestran tablas vacías).
			\item No hay valores posteriores en seguimiento (se muestran tablas vacías).
		\end{itemize} \\
		\textbf{Importancia}          & Media \\
		\bottomrule
	\end{tabularx}
	\caption{CU-16 Eliminar un valor de seguimiento.}
\end{table}


\begin{table}[p]
	\centering
	\begin{tabularx}{\linewidth}{ p{0.21\columnwidth} p{0.71\columnwidth} }
		\toprule
		\textbf{CU-17}    & \textbf{Consultar índice}\\
		\toprule
		\textbf{Versión}              & 1.0    \\
		\textbf{Autor}                & Rodrigo Merino Tovar \\
		\textbf{Requisitos asociados} & RF-4, RF-4.1, RF-4.2, RF-4.3, RF-4.4 y RF-1.2.1  \\
		\textbf{Descripción}          & Para mostrar datos de valores de un índice de forma agregada.\\
		\textbf{Precondición}         & Las bases de datos están disponibles.  \\
		\textbf{Acciones}             &
		\begin{enumerate}
			\def\labelenumi{\arabic{enumi}.}
			\tightlist
			\item El usuario selecciona el índice que quiere consultar. 
			\item Mostrar tabla con componentes del índice.
			\item Visualizar gráfica de evolución del índice en su conjunto.  
			\item Facilitar enlaces RSS a noticias relacionadas.
		\end{enumerate}\\
		\textbf{Postcondición}        & El usuario ve información del índice. \\
		\textbf{Excepciones}          & 
		\begin{itemize}
			\tightlist
			\item No hay datos de índice (mensaje de página inexistente).
		\end{itemize} \\
		\textbf{Importancia}          & Alta \\
		\bottomrule
	\end{tabularx}
	\caption{CU-17 Consultar índice.}
\end{table}


\begin{table}[p]
	\centering
	\begin{tabularx}{\linewidth}{ p{0.21\columnwidth} p{0.71\columnwidth} }
		\toprule
		\textbf{CU-18}    & \textbf{Mostrar tabla de valores}\\
		\toprule
		\textbf{Versión}              & 1.0    \\
		\textbf{Autor}                & Rodrigo Merino Tovar \\
		\textbf{Requisitos asociados} & RF-4 y RF-4.1  \\
		\textbf{Descripción}          & Para mostrar tabla de componentes de un índice.\\
		\textbf{Precondición}         & Las bases de datos están disponibles.  \\
		\textbf{Acciones}             &
		\begin{enumerate}
			\def\labelenumi{\arabic{enumi}.}
			\tightlist
			\item El usuario selecciona el índice que quiere consultar. 
			\item Mostrar tabla con componentes del índice.
			\item El usuario ordena según variación diaria.
		\end{enumerate}\\
		\textbf{Postcondición}        & El usuario consulta precios de todos los valores del índice. \\
		\textbf{Excepciones}          & 
		\begin{itemize}
			\tightlist
			\item No hay datos de índice (mensaje de página inexistente).
		\end{itemize} \\
		\textbf{Importancia}          & Alta \\
		\bottomrule
	\end{tabularx}
	\caption{CU-18 Mostrar tabla de valores.}
\end{table}


\begin{table}[p]
	\centering
	\begin{tabularx}{\linewidth}{ p{0.21\columnwidth} p{0.71\columnwidth} }
		\toprule
		\textbf{CU-19}    & \textbf{Consultar un valor del índice}\\
		\toprule
		\textbf{Versión}              & 1.0    \\
		\textbf{Autor}                & Rodrigo Merino Tovar \\
		\textbf{Requisitos asociados} & RF-4, RF-4.1, RF-4.2, RF-4.2.1, RF-4.2.2 y RF-4.2.3, RF-4.2.4, RF-4.2.5 y RF-4.2.6 \\
		\textbf{Descripción}          & Para consultar información detallada de un único valor cotizado.\\
		\textbf{Precondición}         & El usuario está logueado, las bases de datos están disponibles y el valor existe.  \\
		\textbf{Acciones}             &
		\begin{enumerate}
			\def\labelenumi{\arabic{enumi}.}
			\tightlist
			\item El usuario selecciona el índice del valor de su interés. 
			\item El usuario selecciona el valor que quiere consultar. 
			\item Dar acceso a gráfica interactiva.
			\item Mostrar distribución de retornos.
			\item Mostrar datos de último mes.
			\item Ver evolución del sector.
			\item Mostrar formulario de comparación con otros valores.
			\item Mostrar grafos de correlación.
		\end{enumerate}\\
		\textbf{Postcondición}        & El usuario ve toda la información ordenada en forma de tablas o gráficas; es posible interactuar (de forma básica) con la gráfica interactiva. \\
		\textbf{Excepciones}          & 
		\begin{itemize}
			\tightlist
			\item No hay datos del valor (mensaje de página inexistente).
		\end{itemize} \\
		\textbf{Importancia}          & Alta \\
		\bottomrule
	\end{tabularx}
	\caption{CU-19 Consultar un valor del índice.}
\end{table}


\begin{table}[p]
	\centering
	\begin{tabularx}{\linewidth}{ p{0.21\columnwidth} p{0.71\columnwidth} }
		\toprule
		\textbf{CU-20}    & \textbf{Mostrar gráfica interactiva}\\
		\toprule
		\textbf{Versión}              & 1.0    \\
		\textbf{Autor}                & Rodrigo Merino Tovar \\
		\textbf{Requisitos asociados} & RF-4, RF-4.1, RF-4.2 y RF-4.2.1 \\
		\textbf{Descripción}          & Para interactuar con temporalidades e indicadores de un valor cotizado.\\
		\textbf{Precondición}         & El usuario está logueado, las bases de datos están disponibles y el valor existe.  \\
		\textbf{Acciones}             &
		\begin{enumerate}
			\def\labelenumi{\arabic{enumi}.}
			\tightlist 
			\item El usuario selecciona el valor que quiere consultar. 
			\item Dar acceso a gráfica interactiva.
			\item Se varían temporalidades ajustando volumen y medias móviles.
			\item Se comprueban datos diarios.
		\end{enumerate}\\
		\textbf{Postcondición}        & El usuario interactúa con la gráfica del valor. \\
		\textbf{Excepciones}          & 
		\begin{itemize}
			\tightlist
			\item No hay datos del valor (mensaje de página inexistente).
		\end{itemize} \\
		\textbf{Importancia}          & Alta \\
		\bottomrule
	\end{tabularx}
	\caption{CU-20 Mostrar gráfica interactiva.}
\end{table}


\begin{table}[p]
	\centering
	\begin{tabularx}{\linewidth}{ p{0.21\columnwidth} p{0.71\columnwidth} }
		\toprule
		\textbf{CU-21}    & \textbf{Mostrar distribución de retornos}\\
		\toprule
		\textbf{Versión}              & 1.0    \\
		\textbf{Autor}                & Rodrigo Merino Tovar \\
		\textbf{Requisitos asociados} & RF-4, RF-4.1, RF-4.2 y RF-4.2.2 \\
		\textbf{Descripción}          & Para ver la distribución que siguen los retornos de un valor en el último año (252 sesiones).\\
		\textbf{Precondición}         & El usuario está logueado, las bases de datos están disponibles, el valor existe y tiene datos históricos suficientes.  \\
		\textbf{Acciones}             &
		\begin{enumerate}
			\def\labelenumi{\arabic{enumi}.}
			\tightlist 
			\item El usuario selecciona el valor que quiere consultar. 
			\item Dar acceso a gráfica de distribución de retornos.
		\end{enumerate}\\
		\textbf{Postcondición}        & El usuario consulta la distribución de los retornos de un valor. \\
		\textbf{Excepciones}          & 
		\begin{itemize}
			\tightlist
			\item No hay datos del valor (mensaje de página inexistente).
		\end{itemize} \\
		\textbf{Importancia}          & Baja \\
		\bottomrule
	\end{tabularx}
	\caption{CU-21 Mostrar distribución de retornos.}
\end{table}


\begin{table}[p]
	\centering
	\begin{tabularx}{\linewidth}{ p{0.21\columnwidth} p{0.71\columnwidth} }
		\toprule
		\textbf{CU-22}    & \textbf{Mostrar datos del último mes}\\
		\toprule
		\textbf{Versión}              & 1.0    \\
		\textbf{Autor}                & Rodrigo Merino Tovar \\
		\textbf{Requisitos asociados} & RF-4, RF-4.1, RF-4.2 y RF-4.2.3 \\
		\textbf{Descripción}          & Para ver tabla con precios de un valor en el último mes.\\
		\textbf{Precondición}         & El usuario está logueado, las bases de datos están disponibles, el valor existe y tiene datos históricos suficientes.  \\
		\textbf{Acciones}             &
		\begin{enumerate}
			\def\labelenumi{\arabic{enumi}.}
			\tightlist 
			\item El usuario selecciona el valor que quiere consultar. 
			\item Dar acceso a tabla con datos de último mes.
			\item Mostrar tabla con colores verde (alcista) y rojo (bajista).
		\end{enumerate}\\
		\textbf{Postcondición}        & El usuario consulta la tabla de precios del último mes. \\
		\textbf{Excepciones}          & 
		\begin{itemize}
			\tightlist
			\item No hay datos del valor (mensaje de página inexistente).
		\end{itemize} \\
		\textbf{Importancia}          & Media \\
		\bottomrule
	\end{tabularx}
	\caption{CU-22 Mostrar datos del último mes.}
\end{table}


\begin{table}[p]
	\centering
	\begin{tabularx}{\linewidth}{ p{0.21\columnwidth} p{0.71\columnwidth} }
		\toprule
		\textbf{CU-23}    & \textbf{Mostrar evolución del sector}\\
		\toprule
		\textbf{Versión}              & 1.0    \\
		\textbf{Autor}                & Rodrigo Merino Tovar \\
		\textbf{Requisitos asociados} & RF-4, RF-4.1, RF-4.2 y RF-4.2.4 \\
		\textbf{Descripción}          & Para ver comparación - en términos relativos - entre el valor y su sector de referencia.\\
		\textbf{Precondición}         & El usuario está logueado, las bases de datos están disponibles, el valor existe y tiene datos históricos suficientes.  \\
		\textbf{Acciones}             &
		\begin{enumerate}
			\def\labelenumi{\arabic{enumi}.}
			\tightlist 
			\item El usuario selecciona el valor que quiere consultar. 
			\item Obtener el sector de referencia del valor.
			\item Obtener listado de valores del mismo sector.
			\item Calcular media de evolución de todos los valores del mismo sector. 
			\item Ajustar datos de forma relativa (porcentual).
			\item Mostrar gráfica con evolución de valor y evolución de sector.
		\end{enumerate}\\
		\textbf{Postcondición}        & El usuario consulta la gráfica comparativa de valor con sector de referencia. \\
		\textbf{Excepciones}          & 
		\begin{itemize}
			\tightlist
			\item No hay datos del valor (mensaje de página inexistente).
			\item No hay datos de valores del sector (gráfica sin comparación).
		\end{itemize} \\
		\textbf{Importancia}          & Alta \\
		\bottomrule
	\end{tabularx}
	\caption{CU-23 Mostrar evolución del sector.}
\end{table}


\begin{table}[p]
	\centering
	\begin{tabularx}{\linewidth}{ p{0.21\columnwidth} p{0.71\columnwidth} }
		\toprule
		\textbf{CU-24}    & \textbf{Comparar con otros valores}\\
		\toprule
		\textbf{Versión}              & 1.0    \\
		\textbf{Autor}                & Rodrigo Merino Tovar \\
		\textbf{Requisitos asociados} & RF-4, RF-4.1, RF-4.2 y RF-4.2.5 \\
		\textbf{Descripción}          & Para ver comparación - en términos relativos - con otros valores seleccionados por el usuario.\\
		\textbf{Precondición}         & El usuario está logueado, las bases de datos están disponibles, el valor existe y tiene datos históricos suficientes.  \\
		\textbf{Acciones}             &
		\begin{enumerate}
			\def\labelenumi{\arabic{enumi}.}
			\tightlist 
			\item El usuario selecciona el valor que quiere consultar. 
			\item El usuario selecciona en un formulario otro valor con el que comparar. 
			\item Mostrar comparación relativa de precios de cierre. 
			\item Mostrar comparación relativa de rentabilidades diarias. 
		\end{enumerate}\\
		\textbf{Postcondición}        & Recargar página con nueva gráfica comparativa. \\
		\textbf{Excepciones}          & 
		\begin{itemize}
			\tightlist
			\item No hay datos del valor (mensaje de página inexistente).
			\item No hay datos del valor con el que comparar (mensaje de \emph{ticker} no válido).
		\end{itemize} \\
		\textbf{Importancia}          & Media \\
		\bottomrule
	\end{tabularx}
	\caption{CU-24 Comparar con otros valores.}
\end{table}


\begin{table}[p]
	\centering
	\begin{tabularx}{\linewidth}{ p{0.21\columnwidth} p{0.71\columnwidth} }
		\toprule
		\textbf{CU-25}    & \textbf{Mostrar grafos de correlación}\\
		\toprule
		\textbf{Versión}              & 1.0    \\
		\textbf{Autor}                & Rodrigo Merino Tovar \\
		\textbf{Requisitos asociados} & RF-4, RF-4.1, RF-4.2 y RF-4.2.5 \\
		\textbf{Descripción}          & Para ver grafos de correlación positiva y negativa con el resto de valores disponibles.\\
		\textbf{Precondición}         & El usuario está logueado, las bases de datos están disponibles, el valor existe y tiene datos históricos suficientes.  \\
		\textbf{Acciones}             &
		\begin{enumerate}
			\def\labelenumi{\arabic{enumi}.}
			\tightlist 
			\item El usuario selecciona el valor que quiere consultar. 
			\item Crear matriz de correlación entre todos los valores disponibles.
			\item Crear grafo de correlación positiva (correl. mayor a 0,75)
			\item Crear grafo de correlación negativa (correl. mayor a -0,75)
			\item Mostrar grafos de correlaciones.
		\end{enumerate}\\
		\textbf{Postcondición}        & El usuario consulta los grafos de correlaciones. \\
		\textbf{Excepciones}          & 
		\begin{itemize}
			\tightlist
			\item No hay datos del valor (mensaje de página inexistente).
			\item No hay correlaciones que superen los umbrales (mostrar grafo con nodo de valor sin más datos).
		\end{itemize} \\
		\textbf{Importancia}          & Alta \\
		\bottomrule
	\end{tabularx}
	\caption{CU-25 Mostrar grafos de correlación.}
\end{table}


\begin{table}[p]
	\centering
	\begin{tabularx}{\linewidth}{ p{0.21\columnwidth} p{0.71\columnwidth} }
		\toprule
		\textbf{CU-26}    & \textbf{Mostrar noticias relacionadas}\\
		\toprule
		\textbf{Versión}              & 1.0    \\
		\textbf{Autor}                & Rodrigo Merino Tovar \\
		\textbf{Requisitos asociados} & RF-4 y RF-4.3 \\
		\textbf{Descripción}          & Para facilitar enlaces de fuentes RSS que estén relacionadas con el índice.\\
		\textbf{Precondición}         & Las bases de datos están disponibles y las fuentes RSS están activas.  \\
		\textbf{Acciones}             &
		\begin{enumerate}
			\def\labelenumi{\arabic{enumi}.}
			\tightlist 
			\item El usuario selecciona el índice que quiere consultar. 
			\item Mostrar listado de noticias relacionadas de fuentes RSS.
			\item Pulsar el botón \emph{leer más} de una fuente.
			\item Redirigir a web externa. 
		\end{enumerate}\\
		\textbf{Postcondición}        & El usuario ve listado de noticias relacionadas relevantes y puede seleccionar \emph{leer más} en fuente externa. \\
		\textbf{Excepciones}          & 
		\begin{itemize}
			\tightlist
			\item No hay datos de fuentes RSS (mensaje de error).
		\end{itemize} \\
		\textbf{Importancia}          & Alta \\
		\bottomrule
	\end{tabularx}
	\caption{CU-26 Mostrar noticias relacionadas.}
\end{table}


\begin{table}[p]
	\centering
	\begin{tabularx}{\linewidth}{ p{0.21\columnwidth} p{0.71\columnwidth} }
		\toprule
		\textbf{CU-27}    & \textbf{Gestionar \emph{Lab}}\\
		\toprule
		\textbf{Versión}              & 1.0    \\
		\textbf{Autor}                & Rodrigo Merino Tovar \\
		\textbf{Requisitos asociados} & RF-5, RF-5.1, RF-5.1.1, RF-5.1.2, RF-5.1.3, RF-5.1.4, RF-5.2, RF-5.2.1 y RF-5.2.2 \\
		\textbf{Descripción}          & Para controlar una sección de \emph{laboratorio} virtual con el que experimentar sobre los datos disponibles.\\
		\textbf{Precondición}         & El usuario está logueado y las bases de datos están disponibles.  \\
		\textbf{Acciones}             &
		\begin{enumerate}
			\def\labelenumi{\arabic{enumi}.}
			\tightlist 
			\item El usuario selecciona el índice que quiere consultar. 
			\item Mostrar listado de noticias relacionadas de fuentes RSS.
			\item Pulsar el botón \emph{leer más} de una fuente.
			\item Redirigir a web externa. 
		\end{enumerate}\\
		\textbf{Postcondición}        & El usuario ve listado de noticias relacionadas relevantes y puede seleccionar \emph{leer más} en fuente externa. \\
		\textbf{Excepciones}          & 
		\begin{itemize}
			\tightlist
			\item No hay datos de fuentes RSS (mensaje de error).
		\end{itemize} \\
		\textbf{Importancia}          & Alta \\
		\bottomrule
	\end{tabularx}
	\caption{CU-27 Gestionar \emph{Lab}.}
\end{table}