\capitulo{3}{Conceptos teóricos}

Los conceptos teóricos más destacables de este proyecto residen en el estudio del modelo de Markowitz para la formación de una cartera bien diversificada, el análisis de series temporales con modelos ARIMA y en técnicas básicas de trading algorítmico. 


\section{Diversificación de una cartera de valores cotizados}\label{diversificar_cartera}

Esta sección puede empezar con la idea básica de que el riesgo en las inversiones es perjudicial y tener una cartera diversificada reduce el riesgo, por lo tanto, diversificar una cartera es una buena idea. 

La diversificación de una cartera de valores cotizados es una estrategia fundamental para reducir el riesgo y mejorar la rentabilidad a largo plazo. Esta estrategia consiste en distribuir el capital entre diferentes activos, como acciones, bonos y activos de diferentes sectores y regiones geográficas. Al hacerlo, se reduce la dependencia del rendimiento de una sola empresa o sector, lo que protege a la cartera de las fluctuaciones del mercado y minimiza las pérdidas potenciales.

Una manera de caracterizar una cartera es a través del retorno medio de los activos que la componen y su varianza. En esta sección se verá cómo se realiza una optimización de varianza-media, o más conocida como \emph{Modern Portfolio Theory (MPT)} \citep{wiki:mpt}. Es decir, se va a demostrar cómo se busca una cartera con la mejor media y la mejor varianza posibles dados unos valores en dicha cartera y la ponderación de esos valores en la misma. 


\subsection{Disminuir la varianza para minorar el riesgo}

El retorno logarítmico de un valor viene dado por:

\begin{equation} \label{eq:1}
	R = ln(\frac{P_{t}}{P_{t-1}})
\end{equation}

En este trabajo, en algunas ocasiones y por simplicidad de las explicaciones, se hablará de rentabilidad\footnote{En los casos en los que sea conveniente el uso del retorno logarítmico, se indicará expresamente haciendo uso de \ref{eq:1}}:

\begin{equation}
	R_{simple} = \frac{P_{t-1} - P_{t}}{P_{t}} * 100
\end{equation}

Donde $P_{t}$ es el último precio de cierre de mercado disponible y $P_{t-1}$ es el precio de cierre previo. 

Los retornos de un valor son aleatorios y podemos asumir, de forma general, que hay una distribución normal subyacente en ellos:

\imagen{img_01_distribución_retornos.png}{Distribuciones de retornos de diferentes valores. Fuente: elaboración propia}{0.8}

Haciendo esta asunción de distribución normal en los retornos (algo que no siempre se cumple) podemos pensar en una cartera con dos valores, A y B, cuyas distribuciones de retornos son iguales: 

\begin{align} \label{eq:2}
	A: R_{1} \sim  \mathcal{N}(\mu,\,\sigma^{2})\\ 
	B: R_{2} \sim  \mathcal{N}(\mu,\,\sigma^{2})
\end{align}

Entonces, el retorno esperado será el mismo, $\mu$ , tengamos el 100\% de A en cartera, el 100\% de B o con diferentes ponderaciones. Sin embargo, la varianza sí es distinta, porque si calculamos la desviación estándar de los retornos de A y B tenemos lo siguiente: 

\begin{equation}
	sd(R_{1}) = sd(R_{2}) = \sigma
\end{equation}

Es decir, si invertimos todo en A o todo en B, tendremos la misma varianza, pero si hacemos un reparto de, por ejemplo, 50\%/50\%, veremos que la varianza es menor. 

Para ello, dadas las distribuciones de \ref{eq:2}, asumiremos - por ahora - que son independientes. Además, supongamos que existe una variable Y con $1/2$ de $R_{1}$ y $1/2$ de $R_{2}$:

\begin{equation}
	Y = 1/2R_{1} + 1/2R_{2}
\end{equation} 

Entonces, lo que hay que calcular es la varianza de Y:

\begin{equation}
	var(Y) = var(1/2R_{1} + 1/2R_{2})
\end{equation}

Una de las maneras de calcularlo es teniendo en cuenta lo siguiente:

\begin{align}
	var(cX) &= c^{2}var(X)\\
	var(A + B) &= var(A) + var(B)
\end{align}

Y, por tanto:

\begin{equation}
	var(1/2R_{1} + 1/2R_{2}) = (1/2)^{2}\sigma^{2} + (1/2)^{2}\sigma^{2} = 1/2\sigma^{2}
\end{equation}

Es decir: 

\begin{equation}
	var(1/2R_{1} + 1/2R_{2}) = 1/2\sigma^{2} \rightarrow sd = 1/\sqrt{2}\sigma
\end{equation}


Esto nos lleva a pensar que podemos obtener los mismos retornos medios pero disminuyendo la varianza - y la desviación estándar -, es decir, asumiendo menos riesgos en nuestras inversiones porque tendremos menor volatilidad. 

\subsection{Retorno esperado y varianza de una cartera (\emph{portfolio})}

En este apartado se tratará de describir una cartera de valores de forma matemática, con el objetivo de buscar una manera de optimizarla. Para ello, empezaré con una serie de definiciones estadísticas.

Los valores de una cartera tienen unos pesos en la misma. A esos pesos se les puede caracterizar como un vector $w$:

\begin{equation*}
	w = vector\, de\, longitud\, D
\end{equation*}

Donde $D$ es la cantidad de valores que tenemos en cartera\footnote{En mi código, en lugar de $D$ utilizo \textit{num\_valores} para hacerlo más intuitivo y para cumplir con las reglas de estilo de \textit{Python}}. Así, la ponderación de un único valor en cartera vendrá representada por $w_{i}$, donde $i = 1,...,D$.

Los pesos tendrán algunas restricciones relevantes, como que la suma de todos ellos debe ser 1:

\begin{equation}
	\sum_{i=1}^{D}w_{i} = 1
\end{equation}

Además, podríamos tener otras restricciones como que los pesos deben ser positivos, lo que limitaría el uso de posiciones cortas \citep{wiki:posicion_corta} en cartera. En estos casos, se puede limitar indicando la condición de que $w_{i} \geq 0$. 

Por otro lado, hay que tener en cuenta algunas definiciones que se utilizan de forma habitual, como:

\begin{equation*}
	R_{i} = retorno\, del\, valor\, i
\end{equation*}

El retorno medio es el valor esperado de $R_{i}$:

\begin{equation}
	E(R_{i}) = \mu_{i}\;  (forma\, vectorial\, de\, todos\, los\, \mu_{i}\, :\, \mu)
\end{equation}

Hay que considerar que es posible que exista una correlación entre los retornos de los valores y, por tanto, necesitaremos hacer uso de la matriz de covarianza:

\begin{equation}
	E\{(R_{i}-\mu_{i}) (R_{j}-\mu_{j})\} = \sum_{ij}\;  (forma\, matricial\, :\, \sum_{DxD})
\end{equation}

Si tenemos en cuenta lo anterior, ya es posible definir dos conceptos fundamentales que son el retorno medio esperado y la varianza del retorno de una cartera (\emph{portfolio}):

\begin{align} \label{eq:4}
	\mu_{p} &= E(R_{p})\\
	\sigma_{p}^{2} &= var(R_{p})
\end{align}

Para entender cómo se calculan, voy a empezar por el caso más  sencillo, en el supuesto de tener sólo dos valores en cartera:

\begin{equation} \label{eq:5}
	R_{p} = wR_{1} + (1 - w)R_{2}
\end{equation}

$R_{p}$ es una función de variables aleatorias y, por tanto, tendrá una distribución en términos de media y varianza. La media de $R_{p}$ es su valor esperado y recordando que $E$ es un operador lineal podemos operar. Además, podemos sustituir por \ref{eq:3}:

\begin{align} 
    E(R_{p}) &= E(wR_{1} + (1-w)R_{2}) = wE(R_{1}) + (1-w)E(R_{2}) \\
    \mu_{p}  &= w\mu_{1} + (1-w)\mu_{2}
    \label{eq:4}
\end{align}

La varianza de $R_{p}$ requiere de más cálculos; no podemos hacer la suma directa de las dos varianzas porque puede existir correlación entre $R_{1}$ y $R_{2}$. Entonces, los más sencillo es sustituir en la definición de varianza por \ref{eq:4} y \ref{eq:5}:

\begin{equation} \label{eq:6}
\begin{aligned}
    \text{var}(R_{p}) &= E \{(R_{p} - \mu_{p})^{2}\} \\
    &= E\{[wR_{1} + (1-w)R_{2} - w\mu_{1} - (1-w)\mu_{2}]^{2}\} \\
    &= E\{[w(R_{1} - \mu_{1}) + (1-w)(R_{2} - \mu_{2})]^{2}\} \\
    &= E\{w^{2}(R_{1}-\mu_{1})^{2}\} + E\{(1-w)^{2}(R_{2}-\mu_{2})^{2}\} \\
    &\quad + 2E\{w(1-w)(R_{1}-\mu_{1})(R_{2}-\mu_{2})\} \\
    &= w^{2}\text{var}(R_{1}) + (1-w)^{2}\text{var}(R_{2})+2w(1-w)\text{cov}(R_{1}, R_{2})
\end{aligned}
\end{equation}

Lo visto es \ref{eq:6} se puede escribir en términos de la correlación en lugar de la covarianza, $corr_{12} = \rho_{12} = \sigma_{12}/(\sigma_{1}\sigma_{2})$ \citep{wiki:covarianza_correlacion}:

\begin{equation} \label{eq:7}
	\sigma_{p}^{2} = w_{2}\sigma_{1}^{2} + (1-w)^{2}\sigma_{2}^2 + 2w(1-w)\sigma_{12}
\end{equation}

\begin{equation} \label{eq:8}
	\sigma_{p}^{2} = w_{2}\sigma_{1}^{2} + (1-w)^{2}\sigma_{2}^2 + 2w(1-w)\rho_{12} \sigma_{1}\sigma_{2}
\end{equation}
	

Cualquiera de estas dos fórmulas puede ser utilizada para calcular la varianza de una cartera de valores. 

\subsection{Correlación entre valores}

Inicialmente asumía que los retornos de los valores eran totalmente independientes para demostrar que la diversificación disminuye la varianza y, por tanto, el riesgo. Sin embargo, aquí vemos que los retornos no tienen por qué ser independientes. 

Entonces, analizando detenidamente \ref{eq:7} y \ref{eq:8} vemos que, si $0 < w < 1$ y $R_{1}$ y $R_{2}$ están positivamente correlados, la varianza de la cartera aumenta. Y si $R_{1}$ y $R_{2}$ están negativamente correlados disminuimos la varianza del portfolio y, por tanto, tendremos menor riesgo. 

Ahora, para poder adecuar a código de \emph{Python} estas fórmulas, voy a pasarlas a la notación de producto escalar y despejar la matriz de covarianza de $R$, $w^{T}\Sigma w$:

\begin{equation} \label{eq:9}
\begin{aligned}
	var(R_{p}) &= E\{(R_{p}-\mu_{p}^{2})\\
	&= E\{(R^{T}w - \mu^{T}w)^{2}\} \\
	&= E{(R^{T}w - \mu^{T}w)^{T}(R^{T}w - \mu^{T}w)} \\
	&= E\{w^{T}(R^{T}-\mu^{T})^{T}(R^{T}-\mu^{T})w\} \\
	&= w^{T}E\{(R-\mu)(R-\mu)^{T}\}w \\
	&= w^{T}\Sigma w
\end{aligned}
\end{equation}

\subsection{Simulación de Montecarlo}

Es habitual representar las carteras de valores en términos de la relación rentabilidad/riesgo. Para mantener una idea matemática de los conceptos de este trabajo, al riesgo lo voy a denominar volatilidad:

\begin{equation}
	volatilidad\, cartera = volatilidad_{p} = \sqrt(var(R_{p}))
\end{equation}

Una vez tenemos las fórmulas definidas se puede realizar una simulación de Montecarlo \citep{simulacion_montecarlo} con múltiples posibles carteras de inversión y ver la relación rentabilidad/riesgo (retorno esperado/varianza):

\imagen{img_02_simulación_montecarlo.png}{Simulación de 10.000 posibles portfolios con los valores \emph{RED.MC}, \emph{EOAN.DE} y \emph{CSCO}, permitiendo posiciones cortas (covarianza calculada con datos de mayo-2023 a mayo-2024). Fuente: elaboración propia}{1}

De esta gráfica podemos deducir que es posible obtener una mejor rentabilidad sin aumentar el riesgo, i.e., puede mejorarse el rendimiento esperado de nuestra cartera manteniendo la misma volatilidad. Por ejemplo, el punto naranja indica una cartera eficiente, para la que dada una volatilidad, obtenemos el máximo rendimiento esperado posible. Mientras que la cartera representada con un punto rojo obtienen un rendimiento menor para la misma volatilidad. 

Además, podemos intuir que según aumentamos el riesgo que estamos dispuestos a asumir podemos esperar mayores retornos. 

\subsection{Retornos máximo y mínimo posibles}

Para calcular el retorno de una cartera podemos hacer también la representación de producto escalar:

\begin{equation}
	\mu_{p} = \mu^{T}w
\end{equation}

Si utilizo la intuición rápida de maximizar el retorno por sí sólo, estaré cometiendo un error en los cálculos, ya que como es de esperar el retorno no puede crecer indefinidamente (el máximo de la ecuación previa es $\infty$). Entonces, hay que añadir al menos dos restricciones, que son la de que los pesos de los valores en cartera deben sumar 1 y que los pesos deben ser positivos. Por tanto, una representación más adecuada sería:

\begin{equation}
\begin{aligned}
	\max_{w} \mu^{T}w \\
	sujeto\, a: 1_{D}^{T}w = 1 \\
	w_{i} \geq 0
\end{aligned}
\end{equation}

Es decir, estamos ante un problema de optimización con restricciones\footnote{Es habitual añadir otras restricciones como que, por ejemplo, ningún valor tenga un peso mayor al 50\%, i.e., $w_{i} \leq 0.5$ pero en mi caso no utilizaré esta limitación.} Y, más concretamente, se trata de un problema de programación lineal (LP)\citep{programacion_lineal}. 

De manera similar se puede calcular el retorno mínimo:

\begin{equation}
\begin{aligned}
	\min_{w} \mu^{T}w \\
	sujeto\, a: 1_{D}^{T}w = 1 \\
	w_{i} \geq 0
\end{aligned}
\end{equation}

En \emph{Python} estas funciones pueden representarse a través de la librería \emph{Scipy}, concretamente, con \texttt{scipy.optimize.linprog} \footnote{En este proyecto se puede ver cómo se aplica \texttt{scipy.optmize.linprog} en \texttt{DashBoard.views.py}, en el método \texttt{\_rendimientos\_min\_y\_max(retornos\_df)}.}

\subsection{Optimización en términos de retorno y riesgo simultáneamente}

En el apartado anterior se ha visto cómo optimizar los retornos, pero sin tener en cuenta el riesgo (o volatilidad). En este apartado se añade el concepto de riesgo para realizar una optimización simultánea. 

Ya hemos visto, de forma intuitiva, que según aumenta el retorno esperado también aumenta el riesgo. La idea de la optimización de carteras es que no tomemos más riesgos de los necesarios. 

El riesgo lo podemos medir con la desviación estándar. Como la minimización de la varianza también implica la minimización de la desviación estándar (la raíz cuadrada es una función monótona creciente), usaré la varianza calculada en \ref{eq:9} por comodidad en los cálculos.

Si suponemos que queremos un determinado retorno $r$, podríamos representar la minimización del riesgo de la siguiente manera:

\begin{equation} \label{eq:10}
\begin{aligned}
	min_{w} w^{T}\Sigma w \\
	sujeto a: \mu^{T}w = r \\
	1_{D}^{T}w = 1 \\
	w_{i} \geq 0
\end{aligned}
\end{equation}

Como vemos, estamos ante un problema de programación cuadrática (QP), porque la función objetivo es cuadrática en lugar de lineal, aunque las restricciones sí siguen siendo lineales. 

Para simular la optimización de una función cuadrática en \emph{Scipy} hay que utilizar una función genérica llamada \texttt{minimize()} \footnote{En este proyecto se puede ver cómo se aplica \texttt{minimize()} en \texttt{DashBoard.views.\_frontera\_eficiente\_por\_optimizacion()} como un problema de QP y en \texttt{DashBoard.views.\_mejores\_pesos\_por\_optmizacion()} como un problema que no es QP ni LP}. Hay otras librerías más específicas, pero podemos adecuar \emph{Scipy} para problemas QP.


\subsection{Frontera eficiente}

Una vez hemos conocidos los retornos mínimo y máximo posibles, con la función que queremos optimizar, \ref{eq:10}, podemos hacer que el retorno, $r$, sea una sucesión de puntos entre el mínimo y el máximo e ir calculando la varianza mínima con esos retornos objetivos. Esto nos dará el mejor nivel de riesgo posible para cada retorno entre el mínimo y el máximo posibles. 

Esto nos dará una representación en forma de hipérbola que se conoce como frontera eficiente \citep{wiki:frontera_eficiente} y que Harry Markowitz representó en \citep{book:Portfolio_selection} con su forma parabólica de la siguiente manera:

\imagen{img_03_frontera_eficiente}{Frontera eficiente. Fuente: \citep{book:Portfolio_selection}}{0.45}

Si aplicamos estos conceptos a la simulación de Montecarlo que se realizaba previamente, se obtiene lo siguiente:

\imagen{img_04_montecarlo_frontera_eficiente}{Simulación de Montecarlo con frontera eficiente y rentabilidad de cartera con valores \emph{RED.MC}, \emph{EOAN.DE} y \emph{CSCO}, permitiendo posiciones cortas (covarianza calculada con datos de mayo-2023 a mayo-2024) Fuente: Elaboración propia}{0.95}

La parte superior \footnote{Aunque se obtiene toda la curva, por el propio proceso de optimización, la parte inferior no se puede considerar eficiente porque sólo tenemos que proyectar hacia la parte superior de la misma para ver retornos mejores.} de la línea curva negra que rodea la nube de posibles carteras, obtenidas con la simulación de Montecarlo, es lo que se conoce como frontera eficiente. Lo interesante de esta curva es que cualquier punto que seleccionemos de ella nos indica que no hay otra posible cartera con menor riesgo para el mismo retorno - o que no podemos encontrar un retorno esperado mejor para un determinado nivel de riesgo -.


\subsection{Sharpe ratio}

Hasta ahora se ha visto que podemos encontrar diferentes carteras localmente óptimas - distintas distribuciones de pesos de los mismos valores cotizados - a lo largo de la frontera eficiente, pero cabe preguntarse cómo podemos comparar dos carteras, i.e., cuál es mejor si las dos están en la frontera eficiente.

En principio, podemos asumir que el perfil del inversor influirá en una mayor o menor aversión al riesgo, pero lo ideal es hacer un ratio entre el retorno esperado y la volatilidad para tener una medida objetiva. Ese ratio se conoce como \emph{Sharpe ratio} \citep{wiki:sharpe_ratio}:

\begin{equation}
	SR = \frac{E(R_{p}) - r_{f}}{\sigma_{p}}
\end{equation}

Donde $r_{f}$ representa la tasa libre de riesgo \footnote{En este proyecto se considera una tasa libre de riesgo de 0, porque sólo se utilizan acciones y, aunque tienen rentabilidades por dividendos, éstos no están garantizados. Otra perspectiva podría ser tomar las rentabilidades de los bonos del estado como referencia para la tasa libre de riesgo, pero he preferido limitar los cálculos al mercado de acciones cotizadas.}, que es un retorno garantizado que pueden tener determinados activos como depósitos, bonos, letras o similares. 

La obtención del \emph{Sharpe ratio} se puede realizar de dos formas diferentes. Por un lado, podemos optimizar la función del \emph{Sharpe ratio} \footnote{En este trabajo se puede ver cómo se optimiza en \texttt{DashBoard.views.\_mejores\_pesos\_por\_optimizacion()}} pero también podemos aprovechar las múltiples carteras de la simulación de Montecarlo y buscar la de mejor ratio entre todas ellas. En cualquier caso, si se han simulado suficientes carteras, los resultados deben de ser similares - no iguales porque en la simulación puede que no se haya creado la cartera óptima global -. 

Para facilitar la comprensión al usuario se puede realizar una gráfica con toda la información necesaria y acompañarlo de información adicional sobre la distribución de pesos de cada caso:

\imagen{img_05_sharpe_ratio}{Simulación de Montecarlo con frontera eficiente y Sharpe ratio con los valores \emph{RED.MC}, \emph{EOAN.DE} y \emph{CSCO}, permitiendo posiciones cortas (covarianza calculada con datos de mayo-2023 a mayo-2024) Fuente: Elaboración propia}{0.95}



\section{Análisis de series temporales con modelos ARIMA}\label{series_temporales_arima}

Cuando hablamos de \emph{forecasting} (previsión o predicción) de series temporales con productos financieros hay que tener en cuenta que estamos ante un proceso especialmente complicado y que los resultados obtenidos, en muchas ocasiones, son erróneos o decepcionantes. Por ello, en este trabajo se propone un laboratorio virtual para analizar los datos, que le permita al usuario experimentar con algunas herramientas relativas al análisis de series temporales y que sea esa persona la que saque sus propias conclusiones a través de la información obtenida. 

Ya adelanto que "no existe una bola de cristal que nos diga cuál es el precio futuro de un activo". De hecho, al final de esta sección se intenta demostrar que un modelo \emph{naïve forecast} obtiene mejores resultados que un modelo ARIMA cuando hablamos de datos de valores cotizados. 


\subsection{Modelos autorregresivos $(AR_{(p)})$}

En estadística, un modelo autorregresivo (AR) es una representación de un proceso aleatorio en el que la variable de interés depende de sus observaciones pasadas. De forma general podemos decir que estos modelos son, básicamente, modelos de regresión lineal donde los predictores son datos pasados en la serie temporal. En su versión más sencilla lo podemos representar por:

\begin{equation}
	\hat{y} = mx + b
\end{equation}

Donde $\hat{y}$ es el estimador, $x$ son los datos de entrada y $m$ y $b$ son parámetros que se encuentran a través de la minimización del error de las predicciones - error entendido como diferencia entre real y estimado -. 

Si tenemos más de una entrada de datos podemos representarlo de la siguiente manera:

\begin{equation}
	\hat{y} = w_{1}x_{1} + w_{2}x_{2} + b
\end{equation}

Donde, de nuevo, $w_{1}$, $w_{2}$ y $b$ se encuentran a través de un proceso de minimización del error. 

Un modelo autorregresivo es un modelo de regresión lineal múltiple donde las entradas son los valores ya pasados en la serie temporal. Un modelo autorregresivo de orden $p$ ($AR_{p}$) se puede representar de la siguiente manera:

\begin{equation}
	\hat{y}_{t} = b + \phi_{1}y_{t-1} + \phi_{2}y_{t-2} + ... + \phi_{p}y_{t-p}
\end{equation}

Y se puede denotar para un momento temporal concreto como:

\begin{equation}
	y_{t} = b + \phi_{1}y_{t-1} + \phi_{2}y_{t-2} + ... + \phi_{p}y_{t-p} + \epsilon_{t}
\end{equation}

Donde, $\epsilon_{t}$ es el \emph{ruido} en ese instante $t$: 

\begin{equation}
	\epsilon_{t} = \mathcal{N}(0,\,\sigma^{2})
\end{equation}

Es decir, se mide $y_{t}$ como un modelo lineal de las entradas más un \emph{ruido}. Esto nos lleva a pensar - asumiendo que el valor esperado del \emph{ruido} es 0 - que:

\begin{equation}
	\hat{y}_{t} = E(y_{t})
\end{equation}



\subsection{Modelos de media móvil $(MA_{(q)})$}

Es importante empezar destacando que no hay que confundir el nombre de estos modelos con la \emph{media móvil simple} ni la \emph{media móvil exponencialmente ponderada} (EWMA, o MMEP en castellano). 

Este modelo es similar al autorregresivo en el sentido de que es una función lineal, pero en este caso es una función en términos de errores pasados. Es decir, depende de los errores previos, no de los datos anteriores de la serie temporal. Se representa por:

\begin{equation}
	y_{t} = c + \epsilon_{t} + \theta_{1}\epsilon_{t-1} + \theta_{2}\epsilon_{t-2} + ... + \theta_{q}\epsilon_{t-q}
\end{equation}


Si seguimos tratando los errores como una distribución normal, $\epsilon_{t} = \mathcal{N}(0,\,\sigma^{2})$, cuando calculamos el valor esperado veremos que todos esos errores son 0 y, por tanto, el valor esperado sólo dependerá de $c$:

\begin{equation}
\begin{aligned}
	E(y_{t}) &= E(c + \epsilon_{t} + \theta_{1}\epsilon_{t-1} + \theta_{2}\epsilon_{t-2} + ... + \theta_{q}\epsilon_{t-q}) \\ 
	&= c
\end{aligned}
\end{equation}

Entonces, podemos entender el término sesgado $c$ como el valor medio, y los errores como fluctuaciones que hacen que $y_{t}$ vaya arriba o abajo alrededor de $c$. 


\subsection{Combinación de modelos: $(ARMA_{(p, q)})$}

Usaremos este modelo cuando creamos que nuestros datos están linealmente correlados tanto con datos pasados en la serie temporal como con errores pasados del modelo. 

\begin{equation}
	y_{t} = b + \phi_{1}y_{t-1} + \phi_{2}y_{t-2} + ... + \phi_{p}y_{t-p} + \theta_{1}\epsilon_{t-1} + \theta_{2}\epsilon_{t-2} + ... + \theta_{q}\epsilon_{t-q} + \epsilon_{t}
\end{equation}

Cuando entrenamos un modelo ARMA nos interesa que los datos estén cercanos a la estacionariedad \citep{wiki:proceso_estacionario}. La estacionariedad es muy útil en procesos de modelado, porque implica que varios estadísticos, como la media, la varianza, la autocorrelación, etc. se mantendrán constantes con el tiempo. Y en un modelo ARMA estamos tomando ventanas temporales previas para entrenar el modelo, lo cual nos dará buenos resultados si los estadísticos se mantienen a lo largo de otras ventanas temporales. 


\subsection{Datos estacionarios. $I_{(d)}$ y $ARIMA_{(p, d, q)}$}

Los datos de los valores cotizados, normalmente, no son estacionarios. Por ello, es necesario realizar un proceso de diferenciación para intentar aproximarlos a la estacionariedad. Aquí es donde entra en juego la \emph{I} de \emph{ARIMA}. 

Un proceso $I_{(d)}$ es un proceso estacionario después de haber diferenciado $d$ veces. Es decir, $I_{(d)}$ es un proceso integrado en orden $d$. Normalmente necesitaremos diferenciar una o dos veces, pero no es aconsejable hacerlo más de dos veces. De hecho, es posible ver que con una sola diferenciación podemos eliminar la tendencia de los datos:

\imagen{img_06_diferenciacion.png}{Evolución de precios de cierre de \emph{ITX.MC} en los últimos dos años y los mismos datos con diferenciación de orden 1. Fuente: elaboración propia}{0.9}

Pero esto sólo nos da una idea intuitiva de cómo son los datos y, para poder hacer un buen análisis de los mismos, necesitaremos demostrar de alguna forma si nuestros datos son realmente estacionarios. 

En la sección previa se daba una idea informal de qué son datos no estacionarios, haciendo referencia a la tendencia y a la variación de determinados estadísticos a lo largo de una serie temporal pero, para tener una idea más clara, se pueden ver algunos ejemplos de datos no estacionarios:

\imagen{img_07_serie_no_estacionaria.png}{Serie no estacionaria por variación de media. Fuente: elaboración propia}{0.85}

\imagen{img_08_serie_no_estacionaria.png}{Serie no estacionaria por variación de varianza. Fuente: elaboración propia}{0.85}

De nuevo, esto sólo nos da una idea intuitiva. Pero existe un método para testear si una serie temporal es estacionaria o no: la prueba de Dickey-Fuller aumentada (test ADF, por sus siglas en inglés) \citep{wiki:Dickey-Fuller}. 

En esta prueba tenemos una hipótesis nula - la serie temporal es no estacionaria - y una hipótesis alternativa - la serie temporal es estacionaria -. Introduciremos la serie temporal de datos y obtendremos un resultado con un \emph{p-value}. Entonces, podré comprobar si el \emph{p-value} está por encima o por debajo de un umbral significativo, habitualmente 1\% ó 5\%. Si está por debajo del umbral establecido puedo rechazar la hipótesis nula.

Esto en ARIMA tiene bastante utilidad, porque puedo diferenciar $d$ veces los datos hasta conseguir una serie estacionaria, comprobando con un test ADF. Una vez los datos son estacionarios puedo aplicar un modelo $ARMA_{(p, q)}$.

De manera formal se puede hablar de estacionariedad fuerte o débil (SSS o WSS respectivamente, por sus siglas en inglés)\citep{wiki:proceso_estacionario_en}. Una estacionariedad fuerte significa que la distribución de las variables aleatorias de un proceso estocástico no varia con el tiempo, es decir, no varía en diferentes momentos  $\tau$ de la serie temporal:

\begin{equation}
	F_{Y}(y_{t_{1}+\tau}, y_{t_{2}+\tau}, \ldots, y_{t_{n}+\tau}) = F_{Y}(y_{t_{1}}, y_{t_{2}}, \ldots, y_{t_{n}}) \quad \forall \tau, t_{1}, t_{2}, \ldots, t_{n}
\end{equation}

Una estacionariedad débil hace uso de estadísticos de primer y segundo orden, la media y la varianza (o la auto-covarianza). Por tanto, se va a parecer más a la definición informal que había dado previamente. Se puede representa por:

\begin{align}
	\mu{Y}(t) = \mu_{Y}(t + \tau) \quad \forall \tau \\
	K_{YY}(t_{1},t_{2}) = K_{YY}(t_{1} - t_{2}, 0) \quad \forall t_{1}, t_{2}
\end{align}

Lo que significa que no importa en qué momento $\tau$ de la serie temporal miremos, porque tendremos la misma media y que la auto-covarianza no varía con el tiempo.

Entonces, $ARIMA_{(p, d, q)}$ es un modelo en el que hemos diferenciado $d$ veces antes de aplicar un modelo $ARMA_{(p, q)}$. Y hemos visto que $I_{(d)}$ es una operación que hacemos sobre los datos, mientras que en las partes del modelo autorregresivo y de la media móvil tenemos unas fórmulas que nos permiten hacer predicciones. 


\subsection{\emph{p-value} de un test ADF sobre precios de cierre y sobre retornos}

Dada la importancia que tiene la estacionariedad de los datos, voy a mostrar cómo un test ADF nos arroja valores muy diferentes de \emph{p-value} dependiendo del tipo de datos que estemos utilizando. En todos los ejemplos voy a utilizar un umbral del 5\% para el \emph{p-value} a través de una función que me diga si los datos son estacionarios, o no, en base a ese umbral, algo similar a:

\begin{verbatim}
from statsmodels.tsa.stattools import adfuller

def comprobar_serie_temporal(x):
  res = adfuller(x)
  p_value = res[1]
  if res[1] < 0.05:
  	  return p_value, "Serie temporal estacionaria"
  return p_value, "Serie temporal no estacionaria"
\end{verbatim}

Para esta demostración se usan los datos de precios de cierre de los dos últimos años (mayo 2023 a mayo 2024) de la empresa Inditex. En esos datos se puede ver una clara tendencia ascendente y, por tanto, no será una serie estacionaria. 

\imagen{img_09_p_value_test_adf.png}{Serie no estacionaria, con p-value > 0.5. Fuente: elaboración propia}{1}

Además, voy a aplicar un test ADF sobre los retornos - variaciones diarias en \% - de Inditex en el mismo período de tiempo, para ver que el resultado mejora. Y, por último, voy a utilizar \texttt{diff()} para analizar la evolución intradía no porcentual \footnote{Es habitual aplicar una función logarítmica a \texttt{diff()} para suavizar los resultados, pero a efectos gráficos, en este caso, no aporta mejoras significativas.}, que es el tipo de diferenciación que hacen funciones como \texttt{auto.arima} de \texttt{R} o \texttt{pmdarima.arima.auto\_arima} para \texttt{Python}.

De las figuras siguientes podemos deducir que es posible obtener series temporales no estacionarias a partir de precios de cierre de un valor cotizado. Esto permitirá trabajar con esos datos en un modelo ARIMA.

\imagen{img_10_p_value_test_adf.png}{Serie estacionaria, con p-value < 0.5. Fuente: elaboración propia}{0.9}

\imagen{img_11_p_value_test_adf.png}{Serie estacionaria, con p-value < 0.5. Fuente: elaboración propia}{0.9}


\subsection{Selección de hiperparámetros $q$ y $p$. ACF y PACF}

Aunque existen funciones automáticas que obtienen los mejores hiperparámetros (p, d, q) de ARIMA, conviene saber de dónde se sacan éstos. Ya hemos visto cómo deducir $d$ a través de los resultados de un test ADF. Ahora voy a mostrar cómo buscar los mejores valores de $q$ y $p$ a través de funciones que pueden ser útiles para el análisis de los datos.

\subsubsection{Función de autocorrelación (ACF, por sus siglas en inglés) para obtener $q$}

Sabemos que la covarianza se define como el valor esperado entre dos variables aleatorias cualquiera:

\begin{equation}
	cov(X, Y) = E\{(X - \mu_{X})(Y - \mu_{Y})\}
\end{equation}

Y la autocovarianza es la covarianza entre dos variables aleatorias específicas seleccionadas de dos puntos diferentes en una misma serie temporal. 

\begin{equation}
	autocovarianza = cov(Y_{t_{1}}, Y_{t_{2}})
\end{equation}

Entonces, en una relación similar a la que hay entre la covarianza y la correlación, podemos establecer la autocorrelación como:

\begin{equation}
	autocorrelación = \frac{cov(Y_{t_{1}}, Y_{t_{2}})}{\sigma_{Y}(t_{1})\sigma_{Y}(t_{2})}
\end{equation}

La ACF está definida como si los datos de la serie temporal fueran estacionarios, i.e., que la autocorrelación depende sólo de una diferencia temporal. Esa diferencia entre dos momentos de la serie temporal se puede denotar por $\rho(\tau)$:

\begin{equation}
	\rho(Y_{t_{1}}, Y_{t_{2}}) = \rho(t_{1}, t_{2}) = \rho(\tau)
\end{equation}

El porqué del estimador de $\hat(\rho(\tau))$ escapa a los objetivos de explicación de este proyecto, pero si asumimos que la autocorrelación se mantiene constante en el tiempo, podemos usar todas las muestras de la serie para calcular la autocorrelación para cualquier $\tau$ y la fórmula para calcularlo es:

\begin{equation}
	\hat{\rho}(\tau) = \frac{1}{(T-\tau)\hat{\sigma}^{2}} \sum_{t=1}^{T-\tau} (y_{t}-\hat{\mu})(y_{t+\tau}-\hat{\mu})
\end{equation}

La ventaja de establecerlo como una fórmula es que nos permite mostrar una gráfica de la autocorrelación con $\hat{\rho}$ en el eje vertical y $\tau$ en el horizontal. Pero la parte interesante está en que también se muestran los intervalos de confianza, que se pueden entender como un umbral. Cuando veamos un $\tau$ que supere el umbral de los intervalos de confianza podremos decir que $\hat{\rho}$ no se iguala a 0 en ese punto. Aunque habrá que tener en cuenta que si tenemos un $\alpha$ de, por ejemplo, el 5\%, significa que habrá un 5\% de los datos fuera de los intervalos de manera aleatoria - pero suelen ser datos cercanos al umbral límite -. 

\imagen{img_12_acf.png}{ACF con diferenciación logarítmica de orden 1. Fuente: elaboración propia}{1}

En estas gráficas, normalmente se verá que cuando $\tau$ es 0, la autocorrelación es 1. Esto es porque cuando $\tau$ es 0, hacemos $\frac{varianza}{varianza}$ y será 1. 

Si miramos una gráfica de autocorrelación y vemos que en $\tau$, la ACF $\hat{\rho}$ está fuera del intervalo de confianza, pero justo después de ese $\tau$ resulta que $\hat{\rho}$ no está fuera del intervalo, entonces interpretamos que el primer $\tau$ es la $q$ adecuada para el modelo $MA_{(q)}$.


\subsubsection{Función de autocorrelación parcial (PACF, por sus siglas en inglés) para obtener $p$}

En esta función, de manera similar a como se hacía en ACF, se utiliza una representación gráfica de la propia función en el eje vertical y una especie de retraso temporal (como antes $\tau$) en el eje horizontal. También se muestran los intervalos de confianza y su uso es igual que para los modelos $MA_{(q)}$. 

\imagen{img_13_pacf.png}{PACF con diferenciación logarítmica de orden 1. Fuente: elaboración propia}{1}


La definición de la fórmula también está fuera de los objetivos de este trabajo, pero de manera general se podría decir que la PACF en $\tau$ es la autocorrelación entre $Y(t)$ y $Y(t+\tau)$ pero condicionada a $Y(t+1), Y(t+2), ... , Y(t+\tau-1)$.

Se puede representar por:

\begin{equation}
\begin{aligned}
	\phi(\tau, \tau) = PACF \,en \, \tau \\
	\phi(\tau, \tau) = corr(Y_{t+\tau} - \hat{Y}_{t+\tau}, Y_{t} - \hat{Y}_t)
\end{aligned}
\end{equation}

Por ejemplo, $\phi(0, 0) = 1$ porque no hay valores entre $Y(t)$ e $Y(t)$. 



\subsection{Casos especiales de ARIMA}

Algunos casos especiales de ARIMA son:

\begin{align*}
	&ARIMA(p, 0, 0) = ARMA(p, 0) = AR(p) \\
	&ARIMA(0, 0, q) = ARMA(0, q) = MA(q) \\
	&ARIMA(0, d, 0) = I(d)
\end{align*}

Otro caso especial interesante es $ARIMA(0, 1, 0) = I(1)$ que no es más que un paseo aleatorio (RW o random walk)\citep{wiki:paseo_aleatorio} en el que lo único que tenemos es un modelo representado por \emph{ruido}. Este modelo tan particular se puede denotar por:

\begin{equation}
\begin{aligned}
	&\Delta y_{t} = \epsilon_{t} \\
	&y_{t} - t_{t-1} = \epsilon_{t} \\
	&y_{t} = y_{t-1} + \epsilon_{t}
\end{aligned}
\end{equation}

Por ejemplo, suponiendo que tenemos un conjunto de datos a modelar formado por retornos logarítmicos de un valor cotizado: 

\begin{equation}
	r_{t} = p_{t} - p_{t-1} = \epsilon_{t} \sim \mathcal{N}(\mu,\,\sigma^{2})
\end{equation}

Si entrenamos un modelo ARIMA con dichos datos y los mejores parámetros son $p=0, d=1, q=0$, significa que los retornos no son predecibles a través de los valores ni errores previos en la serie temporal. Esto es interesante porque podemos obtener información relevante aunque no estemos haciendo una predicción de la evolución de los precios - o de los retornos -. 


\subsection{Comprobar resultados de ARIMA}

Una vez obtenidos los parámetros deseados de ARIMA y tras realizar un análisis con el modelo, tendremos que comparar nuestros resultados con algún método base de referencia. En este trabajo se propone utilizar como modelo base una predicción naíf, que consistirá en estimar que el precio de cierre de la siguiente sesión será igual al de la sesión actual. 

A priori, se sabe que esto ocurrirá en muy pocas ocasiones, pero como línea de base comparativa puede ser útil, ya que en la mayoría de los casos los usuarios podrán comprobar - por comparación de RMSE - que los resultados de la estrategia naíf son mejores que los de ARIMA. 

\imagen{img_14_arima_sin_walk_forward.png}{Comparación de ARIMA con una estrategia naíf. Fuente: elaboración propia}{1}

Otra posible estrategia es utilizar una validación \emph{walk-forward anchored}\citep{online:walk_forward} para evaluar el MSE y, posteriormente, compararlo de nuevo con una predicción naíf. En este caso, los usuarios podrán ver que los resultados de ARIMA son similares a los de la estrategia naíf y, en muchos casos, peores. 

\imagen{img_15_arima_con_walk_forward.png}{Comparación entre ARIMA, con walk-forward anchored, y una estrategia naíf. Fuente: elaboración propia}{1}

De esto se puede deducir que ARIMA sirve para entender los datos pero, en la gran mayoría de los casos, no funciona bien para predecir precios de valores cotizados\footnote{Estas comprobaciones se pueden hacer con el uso de la web generada en este trabajo, en cualquiera de los apartados de ARIMA(p,d,q) el usuario puede experimentar y sacar sus propias conclusiones a través de resultados similares a los mostrados en esta sección}. 


\section{Trading algorítmico}\label{algo_trading}

\subsection{Estrategia de cruce de medias (o seguimiento de tendencias)}

La idea básica de esta estrategia es crear dos medias móviles simples sobre los precios de cierre de valores cotizados. Una será la media móvil \emph{lenta} y la otra será la media móvil \emph{rápida}. De forma intuitiva podemos deducir que si incluimos más datos en una media móvil, ésta se ajustará de manera más lenta, porque cada dato en la media móvil tendrá menor influencia en el resultado. Es decir, una media móvil de 50 días será más lenta que una media móvil de 20 días. 

La clave de este algoritmo reside en los puntos en los que la media móvil 'rápida' cruza la media móvil \emph{lenta}. Si la media móvil \emph{rápida} cruza desde abajo, entonces consideramos que es un momento de compra; y la cruza desde arriba, será una señal de venta. Mientras no haya cruce de medias, se asume que la posición se mantiene en cartera sin variación. 

En este algoritmo se utilizan las medias ya comentadas pero, además, se implementan señales de compra y venta que sirven para indicar en qué sentido se está realizando el cruce de medias: si la señal previa es \texttt{False} y la señal actual es \texttt{True} significa que la media \emph{rápida} cruza desde abajo; si la señal previa es \texttt{True} y la señal actual es \texttt{False} significa que la media \emph{rápida} cruza desde arriba. 

Aprovechando la información de las señales se guarda una información en otras columnas para saber en qué momentos comprar. 

\tablaSmallAjustadaConPosicion{Cómo almacenar señales para algoritmo de seguimiento de tendencias}
{c|c|c|c|c|c}
{tabla_señales_cruce_medias}
{
 \textbf{MMS\_ráp} & \textbf{MMS\_len} & \textbf{señal} & \textbf{señal\_prev} & \textbf{comprar} & \textbf{vender} \\
}
{
 .. & .. & 1 & .. & .. & .. \\
 .. & .. & 0 & 1 & 0 & 1 \\
 .. & .. & 0 & 0 & 0 & 0 \\
 .. & .. & 1 & 0 & 1 & 0 \\
}
\newpage

El pseudocódigo, suponiendo que se utiliza \emph{Python} y \texttt{pandas.DataFrame} es\footnote{Para calcular los retornos y medias se utilizan precios de cierre, i.e., se entiende que no se pueden hacer compras ni ventas en la apertura del mercado y, por tanto, hay que hacer un shift(1) en las señales, para que tenga sentido}:

\begin{verbatim}
def algoritmo(df):
  df["señal"] = df["mms_rápida"] > df["mms_lenta"]
  df["señal_previa"] = df["señal"].shift(1) 
  df["comprar"] = (df["señal_previa"]==0) and (df["señal"]==1)
  df["vender"] = (df["señal_previa"]==1) and (df["señal"]==0)
  ...
  return	
\end{verbatim}

En la implementación de este algoritmo es habitual asumir que no se permiten posiciones cortas en la cartera. Además, cada vez que se obtiene una señal de compra o venta, el algoritmo entiende que se usa todo el dinero disponible para comprar o que se liquida toda la posición. Esto tiene sentido, porque las señales no se basan en el precio, sino en los momentos de compra y venta, es decir, en el tiempo. 

Adicionalmente, en la tabla de datos habrá una columna \emph{invertido} que indicará si el usuario tiene el valor comprado o no. Esto nos permitirá medir si nuestra inversión se ve afectada por los retornos. Es decir, si estamos invertidos en el valor, nuestro patrimonio se verá afectado por el retorno del día y si no estamos invertidos nuestro capital se mantendrá constante. Esta idea se puede implementar de diversas formas, pero en este trabajo se sigue el siguiente enfoque:

\begin{verbatim}
invertido = False

for fila in df:
  if invertido and fila["vender"]:
    invertido = False
  if not invertido and fila["comprar"]:
    invertido = True
  fila["invertido"] = invertido
\end{verbatim}

Con esta idea, la columna \texttt{df[´invertido´]} va estar retrasada un día, esto es así porque no podemos comprar al final del día - cuando ya ha cerrado el mercado -,i.e., no podemos estar invertidos hasta el siguiente día, que es cuando calculo el retorno. Pero ese retorno ya le habré cambiado con un \emph{shift()}, lo cual, desde mi punto de vista, es más conveniente. 

Una vez que sabemos si se está invertido en un valor, o no, lo que hay que hacer es calcular el retorno del algoritmo. \footnote{El retorno se calcula habitualmente como se indica en \ref{eq:1} pero en este apartado del trabajo, en la web, se mostrará al usuario información de rentabilidad que le pueda ser de fácil comprensión}. Como en \emph{Python} \texttt{True} es 1 y \texttt{False} es 0, se pueden multiplicar las columnas de \emph{invertido} y \emph{retorno\_shift}, por ejemplo:

\begin{verbatim}
  df["retorno_algoritmo"] = df["invertido"] * df["retorno_shift"]
\end{verbatim}

\tablaSmallConPosicion{Cómo almacenar retornos para algoritmo de seguimiento de tendencias}
{c|c|c}
{tabla_retornos_cruce_medias}
{
 \textbf{invertido} & \textbf{retorno\_shift} & \textbf{retorno\_algoritmo}\\
}
{
 0 & 1  & 0  \\
 1 & -1 & -1 \\
 1 & 2  & 2  \\
}\label{tabla:1} 

\subsubsection{Selección apropiada de medias móviles}

Uno de los problemas de este algoritmo puede residir en la correcta selección de las medias móviles simples que se aplicarán. En este trabajo se realiza una búsqueda por rejilla entre algunas de las medias más frecuentes:

\tablaSmallConPosicion{Medias móviles simples frecuentes}
{c|c}
{tabla_mms_frecuentes}
{
 \textbf{MMS\_lenta} & \textbf{MMS\_rápida}\\
}
{
 5  & 30  \\
 10 & 50  \\
 15 & 200 \\
 20 &  -  \\
}

La selección de las medias se realiza por comparación entre la suma de retornos totales. El mejor retorno arrojado por el algoritmo será el de la medias escogidas\footnote{Se puede ver más información en \texttt{Lab.views.\_algoritmo\_cruce\_medias\_automaticas()}}. 


\subsubsection{Comparar con otras estrategias}

Para determinar cómo de bueno es este algoritmo de cruce de medias, se comparan los resultados con una estrategia básica, comúnmente conocida como \emph{buy and hold} (comprar y aguantar)\citep{wiki:buy_and_hold}.


\subsection{Estrategia basada en \emph{Machine Learning}}

En la estrategia anterior no se usan modelos estadísticos ni hay un \emph{aprendizaje} que se pueda aplicar posteriormente. Sin embargo, sí es posible aplicar ideas de regresión o clasificación para ayudarnos a realizar mejores operaciones\footnote{En este trabajo se utilizará la librería \texttt{scikit-learn} con algunnos de sus modelos}. 

Como ya se ha visto en la sección de ARIMA, predecir la evolución de un valor para varios días es realmente complicado. Por ello, en este apartado se seguirá una idea mucho menos ambiciosa, que es la de intentar predecir la evolución para un sólo día. 

Entonces, se intentará entrenar un modelo con todos los datos disponibles hasta la fecha y estimar únicamente la evolución en la fecha siguiente. Con este tipo de estrategias, la decisión de vender, comprar o no hacer nada es algo que se tiene que actualizar diariamente.

La dificultad de esta aproximación radica en la decisión de cómo formar los datos y qué se va a predecir. En mi caso, seguiré la lógica de que hay valores altamente ponderados en un índice y que estos tienen la capacidad de \emph{mover} ese índice \emph{por sí solos} (o, al menos, con mayor influencia que el resto de componentes). Teniendo en cuenta los retornos de dichos valores estimaré si al día siguiente la tendencia será alcista o bajista, de forma general para el índice de referencia. 

Este enfoque se basa en seleccionar los retornos de valores de mayor capitalización de mercado \citep{wiki:market_cap} como si fueran atributos de un modelo de regresión o clasificación. Y la media de los retornos se considerará la clase objetivo. De esta forma, basándome en la hipótesis del aprendizaje inductivo, supondré que es posible que un modelo que aproxime bien una función en los datos de entrenamiento, también lo hará bien en los datos desconocidos. 

La selección de los valores que \emph{representarán} a cada índice serán:

\tablaSmallAjustadaConPosicion{Valores seleccionados de cada índice}
{X|X|X|X}
{tabla_valores_seleccionados}
{
 \textbf{DJ30} & \textbf{IBEX35} & \textbf{FTSE100} & \textbf{DAX40}\\
}
{
 Microsoft (MSFT)  			& Inditex (ITX.MC)  		& SHELL (SHEL.L) 		& SAP (SAP.DE)\\
 Apple (AAPL) 				& Iberdrola (IBE.MC)  		& AstraZeneca (AZN.L) 	& Siemens AG (SIE.DE)\\
 United Health (UNH) 		& Banco Santander (SAN.MC) 	& HSBC (HSBA.L)			& Allianz (ALV.DE)\\
 Johnson and Johnson (JNJ) 	& BBVA (BBVA.MC)			& Unilever (ULVR.L)		& Airbus (AIR.DE)\\
 JP Morgan Chase (JPM) 		& Caixabank (CABK.MC)  		& BP (BP.L)				& Deutsche Telekom\\
}

Cabe preguntarse por qué no se usan los datos de todos los componentes de cada índice y la respuesta es que, en esos casos, los modelos tienden a sobreajustar y no nos interesa. También es adecuado pensar en el porqué de usar retornos en lugar de precios de cierre; esto lo haré así porque los retornos - como la se ha explicado en la sección de ARIMA - siguen, habitualmente, una distribución normal, con una varianza similar a lo largo del tiempo. Entonces, como los modelos de \emph{Machine Learning} no son buenos extrapolando, si los entrenara con precios de cierre y luego en los test aparecieran datos por encima o por debajo, el modelo \emph{no sabría qué hacer}. Sin embargo, con los retornos esto es menos probable que ocurra, porque se comportan como series estacionarias en la mayoría de los casos. 

\subsubsection{Preprocesado de datos}

Para poder utilizar esta estrategia es necesario adaptar los datos para poder trabajar con ellos de manera adecuada. 

\tablaSmallSinColores{Datos con estrategia basada en ML}
{c|c|c|c|c|>{\columncolor{gray}}c}
{tabla_datos_ML}
{
 \textbf{MSFT} & \textbf{AAPL} & \textbf{UNH} & \textbf{JNJ} & \textbf{JPM} & \textbf{DJ30} \\
}
{
1 & 1 & 1 & 1 & 1 & 2\\
2 & 2 & 2 & 2 & 2 & 3\\
3 & 3 & 3 & 3 & 3 & 4\\
.. & .. & .. & .. & .. & .. \\
}

La idea es utilizar los retornos actuales de los valores como entradas del modelo para predecir lo que ocurrirá el día siguiente, por tanto, es adecuado que la columna objetivo esté \emph{adelantada} un día - de nuevo, con un \texttt{shift()} -. A la hora de trabajar con los datos será mucho más sencillo tener toda la información en una sola fila. Así, los datos para usar con los modelos estarán distribuidos de la siguiente manera:

\imagen{img_17_xtrain_ytrain_ml.png}{Conjuntos de entrenamiento y test en estrategia basada en ML. Fuente: elaboración propia}{0.9}


\subsubsection{Algoritmo}

Al utilizar modelos de clasificación o de regresión tendremos algo similar a \texttt{predicción = model.predict(X)}. Entonces, puedo utilizar la predicción para guardar información de si un usuario está invertido en ese momento, o no, con un código similar a este:

\begin{verbatim}
	predicción = model.predict(X)
	...
	if predicción > 0:
		df["invertido"] = 1 
	else:
		df["invertido"] = 0 
\end{verbatim}

Es decir, si la predicción es mayor que 0, entonces se genera una señal de compra o, simplemente, se mantiene la inversión si ya estuviera activa; y al contrario para predicciones menores que 0. Esto permite conocer, con cada predicción, si mantenemos la inversión o no. Y una vez que se conoce el estado de invertido/no invertido se realiza el mismo proceso que en \ref{tabla:1} multiplicando: 

\begin{verbatim}
  df["retorno_algoritmo"] = df["invertido"] * df["retorno_shift"]
\end{verbatim}