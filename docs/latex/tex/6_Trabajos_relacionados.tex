\capitulo{6}{Trabajos relacionados}

Publicaciones sobre inversión y control de la diversificación de las carteras hay muchas, pero por el conocimiento que me han aportado tengo que destacar \emph{El inversor inteligente} \citep{book:Inversor_inteligente} y uno de los libros de Andrea Redondo, \emph{Inversión: Claves para alcanzar la libertad financiera} \citep{book:libertad_financiera}.

Por otro lado, intentar predecir la evolución de los precios de valores cotizados no parece algo nuevo y, de hecho, en los últimos años se han ido publicando múltiples \emph{papers}, artículos y libros relacionados con algunos de los conceptos tratados en este trabajo. No sólo podemos encontrar artículos sencillos con porciones básicas de código, sino que existe una cantidad importante de estudios serios especialmente relacionados con el \emph{forecasting} de series temporales y con la aplicación de técnicas de \emph{machine learning}. Algunos de los más relevantes se listan a continuación:

\begin{itemize}
\tightlist
\item \emph{Machine Learning for Asset Managers} \citep{paper:ML_Asset_Managers}
\item \emph{Financial Time Series Forecasting with Deep Learning: A Systematic Literature Review: 2005-2019} \citep{paper:Deep_Learning_Review}. Aquí se pueden encontrar multitud de referencias a trabajos especializados en este campo. 
\item \emph{Deep Learning for Finance: Deep Portfolios} \citep{paper:Deep_Portfolios}
\item \emph{A systematic review of fundamental and technical analysis of stock market predictions} \citep{paper:fund_and_tech_analysis}
\item \emph{Technical Analysis and Machine Learning: A Systematic Review} \citep{paper:Technical_Analysis_ML}
\end{itemize}

Además de la aplicación de técnicas de \emph{trading} algorítmico, también hay una parte importante de la comunidad analista-inversora que ha dedicado históricamente esfuerzos a aplicar redes neuronales al mundo financiero:

\begin{itemize}
\tightlist
\item \emph{Forecasting Stock Prices with Neural Networks} \citep{paper:Forecasting_NN}
\item \emph{A Survey on LSTM Neural Networks for Time Series Forecasting} \citep{paper:LSTM_Survey}
\end{itemize}

Y otros investigadores han continuado en esa misma línea:

\begin{itemize}
\tightlist
\item \emph{Neural Networks for Algorithmic Trading} \citep{paper:NN_Algorithmic_Trading}
\item \emph{Deep learning for financial applications: A survey} \citep{paper:Deep_Learning_Survey}
\item \emph{Deep Reinforcement Learning for Trading} \citep{paper:Deep_Reinforcement_Trading}
\item \emph{Stock Price Prediction Using Convolutional Neural Networks on a Gridded Time Representation} \citep{paper:CNN_Stock_Prediction}
\end{itemize}


También es casi obligatorio mencionar que es frecuente ver publicaciones sobre la programación de \emph{bots} para realizar \emph{trading} de forma automática. Si bien es cierto que no se suele utilizar \emph{Python} para la programación de estos \emph{bots}, sino lenguajes especializados como MQL4 o MQL5, podemos asumir que hay conceptos extrapolables que resultan de interés:

\begin{itemize}
\tightlist
\item \emph{Automated Trading with Python: Designing and Developing Automated Trading Systems in Python} \citep{paper:Automated_Trading_Python}
\item \emph{Building Machine Learning Powered Trading Bots} \citep{paper:ML_Trading_Bots}
\end{itemize}

Por último, si utilizamos las webs más populares sobre bolsa e inversión, veremos que empieza a ser habitual que integren algún tipo de sección - normalmente de pago - con predicción automática de precios o análisis de señales de compra-venta. Incluso hay entidades que están ofertando a sus clientes herramientas de formación automática de carteras, lo cual lleva a pensar que este tipo de servicios se están popularizando\footnote{No parece adecuado convertir este trabajo en un mecanismo publicitario para webs o entidades y por ello no indico referencias; confío en que cualquiera que esté interesado en este mundo conocerá las opciones \emph{pro} de múltiples webs o las herramientas de tipo \emph{roboadvisor}.}. 