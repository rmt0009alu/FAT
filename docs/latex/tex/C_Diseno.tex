\apendice{Especificación de diseño}

\section{Introducción}

En este anexo se describe la resolución de los requisitos y casos de uso previos. Además, se muestra cómo se almacenan y manejan los datos, detalles procedimentales y la estructura interna del proyecto, entre otros.

\section{Diseño de datos}

En este proyecto de \emph{Django} se han creado cinco aplicaciones, intentado separar la lógica y funcionalidades de cada una de ellas de manera coherente con los resultados esperados. Hay una aplicación principal, denominada \emph{FAT} que se encarga de la gestión y configuración general del proyecto, y otras cuatro aplicaciones que se llaman \emph{News}, \emph{Analysis}, \emph{DashBoard} y \emph{Lab}. 

Por otro lado, es importante destacar que se han utilizado cinco bases de datos. Una de ellas  dedicada a la gestión de usuarios y sus datos asociados, así como a tablas comunes de datos entre valores: divisas, sectores, etc. Las otras cuatro bases de datos almacenan información histórica de los componentes de cada índice:

\imagen{img_anex_02_bases_datos.png}{Diseño de bases de datos. Fuente: elaboración propia}{1}

La decisión de diseño inicial ha implicado una adaptación consecuente de los datos, en donde se crean las siguientes entidades:

\begin{itemize}
\item
\textbf{Usuario} tiene un nombre y un identificador únicos, una marca que indica si es \emph{superusuario} y una contraseña que debe cumplir unas condiciones de complejidad. A su vez puede estar asociado a grupos de usuarios y tener permisos de acceso según su rol. Esta entidad viene predefinida por \emph{Django} y de ella no se utilizan los grupos. 

\item
\textbf{StockComprado}: representa a un valor que tenga el usuario en cartera. Se define a través de: \emph{usuario}, \emph{ticker\_bd}, \emph{bd}, \emph{ticker}, \emph{nombre\_stock}, \emph{fecha\_compra}, \emph{num\_acciones}, \emph{precio\_compra}, \emph{moneda}, \emph{sector}, \emph{ult\_cierre} y \emph{objects} (para \emph{Django}). Los nombres de los \emph{ticker} se diferencian entre el nombre con formato de punto y el nombre con formato de '\_', en todos los casos, para referenciar a las tablas en las bases de datos y evitar conflictos de notación. 

\item
\textbf{StockSeguimiento}: representa a un valor que tenga el usuario en seguimiento. sus atributos son: \emph{usuario}, \emph{ticker\_bd}, \emph{bd}, \emph{ticker}, \emph{nombre\_stock}, \emph{fecha\_inicio\_seguimiento}, \emph{precio\_entrada\_deseado}, \emph{moneda}, \emph{sector} y \emph{objects}.

\item 
\textbf{Sectores}: para almacenar de forma conjunta todos los valores disponibles en todas las bases de datos junto con su sector de referencia. Este modelo se crea única y exclusivamente para mejorar el rendimiento y evitar la búsqueda recurrente de valor-sector. Sus atributos son: \emph{id}, \emph{ticker\_bd}, \emph{bd}, \emph{ticker}, \emph{nombre}, \emph{sector} y \emph{objects}. 

\item
\textbf{CambioMoneda}: se utiliza para guardar la información de la última sesión de las divisas afectadas por las bases de datos y sus cambios de divisa. Es una entidad muy pequeña, pero necesaria para el correcto funcionamiento de las aplicaciones. Su estructura interna viene definida por: \emph{id}, \emph{ticker\_forex}, \emph{date}, \emph{ultimo\_cierre}, \emph{objects}.

\item 
\textbf{StockBase}: representa a cada uno de los valores cotizados. Estos valores tienen los siguientes atributos: \emph{id}, \emph{date}, \emph{open}, \emph{high}, \emph{low}, \emph{close}, \emph{volume}, \emph{dividends}, \emph{stock\_splits}, \emph{ticker}, \emph{previous\_close}, \emph{percent\_variance}, \emph{mm20}, \emph{mm50}, \emph{mm200}, \emph{name}, \emph{sector}, \emph{currency} y \emph{objects}. La mayoría de estos atributos vienen impuestos por la API \emph{yFinance} utilizada para la obtención de los datos. Además, es recomendable no modificar los nombres para facilitar la compatibilidad con otras librerías, como \emph{plotly}.
\end{itemize}

\newpage


\imagenSinMargenSubsection{Diagrama E/R}{img_anex_03_diagrama_ER.png}{Diagrama E/R}


\imagenSinMargenSubsection{Diagrama relacional}{img_anex_04_diagrama_relacional.png}{Diagrama relacional}


% Título de section añadido con la imagen para mejor integración
\imagenSinMargenSeccionYSubseccion{Diseño procedimental}{Diagrama de secuencias \emph{News} y \emph{Analysis}}{img_anex_05_diagrama_secuencia_1.png}{Diagrama de secuencias \emph{News} y \emph{Analysis}}

\imagenSinMargenSubsectionApasiada{Diagrama de secuencias \emph{DashBoard}}{img_anex_06_diagrama_secuencia_2.png}{Diagrama de secuencias \emph{DashBoard}}

\imagenSinMargenSubsection{Diagrama de secuencias \emph{Lab}}{img_anex_07_diagrama_secuencia_3.png}{Diagrama de secuencias \emph{Lab}}

\section{Diseño arquitectónico}

\subsection{Patrón MVT}

Al realizar un proyecto en \emph{Django} hay varias decisiones de diseño que vienen impuestas por el propio \emph{framework}. La característica más importante es que se sigue el patrón MVT (\emph{Model View Template}).

El patrón MVT es una arquitectura de desarrollo web donde \emph{Model} maneja la lógica de la base de datos, \emph{View} procesa las solicitudes y devuelve respuestas y \emph{Template} define la presentación visual. MVT facilita la separación de responsabilidades, permitiendo un desarrollo mejor organizado y un mantenimiento más eficiente. Las comunicaciones entre estos componentes son gestionadas por el \emph{framework}, lo que simplifica la creación de aplicaciones web.

\imagen{img_anex_08_diagrama_MVT.png}{Diagrama de patrón MVT. Fuente: elaboración propia}{1}


\subsection{Patrón \emph{Repository}}

\emph{Django} trabaja con el patrón \emph{Repository} de manera implícita a través de su sistema de modelos y el ORM (Object-Relational Mapping)\citep{wiki:ORM}. Aunque no se implementa este patrón de manera estricta, su ORM y sus \emph{managers} proporcionan una funcionalidad muy similar.

El patrón \emph{Repository} actúa como una capa intermedia entre la lógica de negocio y la capa de acceso a datos. Sus principales objetivos son:

\begin{itemize}
\tightlist
\item 
\textbf{Encapsulamiento} encapsula el acceso a la base de datos, proporcionando un entorno accesible para la lógica de negocio.
\item
\textbf{Desacoplamiento} desacopla la lógica de negocio de los detalles de persistencia.
\item 
\textbf{Consistencia} proporciona una forma de trabajar con colecciones de objetos de dominio sin exponer detalles de la base de datos.
\end{itemize}

En este proyecto, de manera adicional, se siguen ideas basadas en este patrón y se crea un esquema muy similar al de un patrón \emph{Respository}:

\imagen{img_anex_09_patrón_repositorio.png}{Patrón repositorio. Fuente: elaboración propia}{1}

\subsection{Patrón \emph{Factory}}

Para la creación de los objetos \emph{StockBase} se utiliza el patrón \emph{Factory}. La idea subyacente de la aplicación de este patrón al proyecto es que se crea una clase común sin definir completamente y, según la necesidad, se va utilizando de forma dinámica para crear las tablas de los valores cotizados. 

La lógica para crear las clases de modelos se encapsula en un bucle que itera sobre los \emph{tickers} disponibles, abstrayendo el proceso de creación de instancias concretas de modelos.

Este enfoque permite mejorar la mantenibilidad, ya que no será necesario crear nuevas clases por cada valor que queramos añadir, sino que se pueden actualizar las  listas de valores\footnote{Ver \texttt{/util/tickers/Tickers\_BDs.py} para más información.} (llamados \emph{tickers} por sencillez de uso a lo largo del proyecto) y, a través de este patrón, se generarán las clases y tablas oportunas. 

\subsection{Patrón \emph{Strategy}}

Aunque no sigue exactamente este patrón en el diseño del proyecto, he de mencionar que sí se utiliza un enfoque similar a la hora de enrutar la información hacia las bases de datos. 

El \emph{router}\footnote{Ver \texttt{/FAT/routers/router\_bases\_datos.py} para más información} clasifica las bases de datos en función del modelo y define estrategias para:

\begin{itemize}
\tightlist
\item 
Lectura de Datos (\emph{db\_for\_read}).
\item 
Escritura de Datos (\emph{db\_for\_write}).
\item 
Permitir Relaciones (\emph{allow\_relation}).
\item
Migraciones (\emph{allow\_migrate}).
\end{itemize}

La mayor ventaja de seguir ideas basadas en este patrón es que las estrategias se pueden cambiar fácilmente, modificando la implementación de los métodos sin afectar a otras partes del proyecto. 


\subsection{Diagramas de paquetes y directorios}

En el contexto de \emph{Django}, cada aplicación (\emph{app}) puede considerarse un paquete. \emph{Django} está diseñado de manera modular, permitiendo que cada \emph{app} sea una unidad independiente y reutilizable dentro del proyecto y, por tanto, se va a mostrar la estructura de paquetes basada en esta concepción:

\newpage
\subsubsection{Directorio general de todo el proyecto. Distribución de paquetes}

\imagen{img_anex_10_diagrama_paquetes_1.png}{Diagrama de paquetes general.}{1}


\subsubsection{Paquete \emph{FAT}}


\imagen{img_anex_11_diagrama_paquetes_2.png}{Diagrama de paquete \emph{FAT}.}{1}

Como se ve en la figura anterior, este paquete cumple las siguientes funciones:

\begin{itemize}
\tightlist
\item
Configuración general del proyecto (\emph{settings}) y direccionamiento de \emph{URLs}.
\item
Gestión de \emph{router} para las bases de datos.
\item
Almacenamiento de variables de entorno secretas (\emph{.env})
\end{itemize}


\subsubsection{Paquete \emph{News}}

\imagen{img_anex_12_diagrama_paquetes_3.png}{Diagrama de paquete \emph{News}.}{1}

En \emph{News} se gestiona, fundamentalmente, la conexión con la API \emph{NewsAPI}, que permite ofrecer noticias en tiempo real al usuario en portada. Además, permite realizar una búsqueda de los mejores y peores valores de cada índice y dispone de una función de representación gráfica que se integra con otras \emph{apps} del proyecto\footnote{Ver \texttt{/News/views/\_generar\_figura()} para más información.}. 

En este proyecto la cuenta utilizada para la obtención de noticias es gratuita (cualquiera puede adquirir una \emph{API key} gratuita\footnote{Visitar \textbf{https://newsapi.org/}.}) y los contenidos son, en muchos casos, relacionados con India y Estados Unidos, pero como muestra de integración en una web, parece interesante incluirlo al trabajo por los fines didácticos. 


\subsubsection{Paquete \emph{Analysis}}

\imagen{img_anex_13_diagrama_paquetes_4.png}{Diagrama de paquete \emph{Analysis}.}{1}

Este paquete cumple las siguientes funciones:

\begin{itemize}
\tightlist
\item
Creación y gestión de modelos \emph{StockBase}, \emph{Sectores} y \emph{CambioMoneda}.
\item
Controlar las vistas de registro, \emph{login} y \emph{logout}.
\item
Control de gráficas dinámicas de valores cotizados.
\item
Gestión de datos asociados a valores y su sector de referencia.
\item 
Manejo de datos de índices y fuentes RSS asociadas. 
\end{itemize}


\subsubsection{Paquete \emph{DashBoard}}

El paquete \emph{DashBoard} se encarga de:

\begin{itemize}
\tightlist
\item
Controlar la evolución de la cartera de un usuario en la divisa seleccionada (euros). 
\item
Creación y gestión de modelos \emph{StockComprado} y \emph{StockSeguimiento}. 
\item
Control de gráficas de Markowitz y distribución de pesos con ratio de Sharpe.
\item
Manejo de plantillas de zona privada de usuario para ofrecer información agregada. 
\end{itemize}

Además, en esta aplicación e incluyen varios formularios de interacción con el usuario, aprovechando el potencial de \emph{Django} para este tipo de tareas.

\imagen{img_anex_14_diagrama_paquetes_5.png}{Diagrama de paquete \emph{DashBoard}.}{1}


\subsection{Paquete \emph{Lab}}

Este paquete cumple con las siguientes funcionalidades\footnote{Para los usuarios del repositorio y no de la web también está disponible la utilización de redes LSTM con funciones muy básicas.}:

\begin{itemize}
\tightlist
\item
Control de \emph{forecasting} con modelos ARIMA.
\item
Control de funciones de \emph{trading} algorítmico.
\end{itemize}

\imagen{img_anex_15_diagrama_paquetes_6.png}{Diagrama de paquete \emph{Lab}.}{1}


\subsection{Paquete \emph{util}}

\imagen{img_anex_16_diagrama_paquetes_7.png}{Diagrama de paquete \emph{util}.}{1}


Aquí se almacenan los siguientes módulos de utilidad:

\begin{itemize}
\tightlist
\item
Directorio \emph{rss} con las direcciones de los \emph{feeds} asociados a los índices. 
\item
Directorio \emph{tickers} con métodos que permiten acceder, de forma sencilla, a todos los \emph{tickers} disponibles en la aplicación. Si se quiere añadir un nuevo índice, es recomendable empezar por añadir aquí los \emph{tickers} de dicho índice, ya que esto facilitaría mucho los pasos posteriores. 
\item
Archivos para crear y actualizar las bases de datos de los valores. 
\item
Archivo para crear un árbol de directorios del proyecto\footnote{Requiere de la instalación de \texttt{seedir} (y \texttt{emoji} si se desea).}.
\end{itemize}


\subsection{Directorio \emph{tests}}

\imagen{img_anex_17_diagrama_paquetes_8.png}{Diagrama de directorio \emph{tests}.}{0.9}

Como se puede observar en la figura anterior, en \emph{tests} se almacenan todos los tests del proyecto, tratando de seguir la misma estructura de carpetas para mayor mantenibilidad en futuras mejoras. 

Se recomienda el uso de \texttt{coverage} para lanzar los tests.


\subsection{Directorio \emph{log}}

En este paquete se almacenan una serie de archivos de \emph{log} que permiten comprobar lo que ha ocurrido con los tests. Son meramente informativos para el desarrollador. Además, aquí se incluye la configuración del \emph{logger} para todos los tests.

\imagen{img_anex_18_diagrama_paquetes_9.png}{Diagrama de directorio \emph{log}.}{1}


\subsection{Directorio \emph{htmlcov}}

Al utilizar la librería \texttt{coverage} podemos generar un informe HTML que permite visualizar las partes de código que quedan sin cubrir y las que ya están cubiertas. Este informe está accesible en este paquete y es consultable haciendo doble \emph{click} sobre el archivo \emph{index.html}:

\imagen{img_anex_19_diagrama_paquetes_10.png}{Diagrama de directorio \emph{htmlcov}.}{1}


\subsection{Directorio \emph{databases}}

Las bases de datos utilizadas por las aplicaciones se encuentran en este directorio:

\imagen{img_anex_20_diagrama_paquetes_11.png}{Diagrama de directorio \emph{databases}.}{1}


\subsection{Directorio \emph{docs}}

La documentación del proyecto puede ser consultada en este directorio. Los archivos están distribuidos de la siguiente manera:

\begin{itemize}
\tightlist
\item
\emph{/burndowns}: recoge imágenes de la evolución de los diferentes \emph{sprints}.
\item
\emph{/sphinx}: al ejecutar el archivo \texttt{/sphinx/\_build/html/index.html} se puede consultar la documentación del código con el mismo formato que el disponible en \emph{ReadTheDocs}.
\item
\emph{/latex}: almacena la memoria y los anexos, así como los archivos \LaTeX utilizados y las imágenes asociadas.
\end{itemize}

\imagen{img_anex_21_diagrama_paquetes_12.png}{Diagrama de directorio \emph{docs}.}{1}


\subsection{Directorio \emph{.github}}

Al utilizar \emph{GitHub actions} es necesario configurarlas previamente. Los pasos que se deben realizar en las acciones - cada vez que se hace un \emph{push to origin/main} - se describen en los archivos que podemos encontrar en este directorio.

\imagen{img_anex_22_diagrama_paquetes_13.png}{Diagrama de directorio \emph{.github}.}{1}

Las \emph{GitHub actions} están configuradas para generar archivos \emph{.zip}, llamados \emph{Artifacts}, que permitan descargar los informes de resultados tanto de \texttt{pylint} como de \texttt{coverage}. 